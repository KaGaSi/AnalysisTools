\section{DensityAggregates} \label{sec:DensityAggregates}

This utility calculates radial density profiles (RDPs, or radial number
densities) for bead types in an aggregate with specified size (the number
of molecules or aggregation number, $A_{\mathrm{S}}$) from its centre of
mass.

RDP$_i(r)$ of bead type $i$, where $r$ is distance from an aggregate's
centre of mass, is the number of these beads in a spherical shell between
the distances $r$ and $r+\mathrm{d}r$ (in \texttt{DensityAggregates},
$\mathrm{d}r$ is the \texttt{<width>} argument) divided by the
volume of this shell.
%The distance written in the output file is always
%$r+0.5\mathrm{d}r$. Besides RDP, the output file also contains radial
%number profiles (i.e., the number of beads in the spherical shell not
%divided by the shell's volume) and one-sigma errors for both RDP and radial
%number profiles. The output file also contains a header describing what is
%in each column.

Instead of `true' aggregate size, a number of molecules of specific type(s)
can be used (\texttt{-m} option). For example, an aggregate containg 1
\texttt{Mol\_A} molecule and 2 \texttt{Mol\_B} molecules (i.e., three
molecules in all) can be specified in several ways:
\begin{enumerate}[nosep]
  \item  with \texttt{<agg size(s)>} of 3;
  \item  with \texttt{<agg size(s)>} of 3 and \texttt{-m Mol\_A Mol\_B};
  \item  with \texttt{<agg size(s)>} of 1 and \texttt{-m Mol\_A}; or
  \item  with \texttt{<agg size(s)>} of 2 and \texttt{-m Mol\_B}.
\end{enumerate}

Care must be taken when different molecule types share the same bead type.
If one bead type is in more molecule types, the resulting density for that
bead type will be the sum of its densities from all molecule types it
appears in. The \texttt{-x} option can overcome this -- specific molecule
types can be excluded from density calculations, i.e., density of beads in
the excluded molecule types will not be calculated.  For example, assume
two molecule types -- \texttt{Mol\_1} and \texttt{Mol\_2}.  \texttt{Mol\_1}
contains bead types \texttt{A} and \texttt{B}; \texttt{Mol\_2} contains
bead types \texttt{A} and \texttt{C}.  Depending on whether and how the
\texttt{-x} option is used, the utility will calculate:
\begin{enumerate}[nosep]
  \item densities of \texttt{A}, \texttt{B}, and \texttt{C} beads (density of
    \texttt{A} beads is a sum from both molecules), if no \texttt{-x} is used;
  \item densities of only \texttt{A} and \texttt{B} beads (with \texttt{A}
    beads only from \texttt{Mol\_1}), if \texttt{-x Mol\_2} is used;
  \item densities of only \texttt{A} and \texttt{C} beads (with \texttt{A}
    beads only from \texttt{Mol\_2}), if \texttt{-x Mol\_1} is used; or
  \item no densities at all if \texttt{-x Mol\_1 Mol\_2} is used.
\end{enumerate}
Therefore, to be able to plot density of \texttt{A} beads from
\texttt{Mol\_1} and \texttt{Mol\_2} separately, (2) and (3) should be used
(i.e., \texttt{DensityAggregates} should be run twice).

Usage:

\vspace{1em}
\noindent
\texttt{DensityAggregates <input> <output.agg> <width> <output.rho> <agg \\
size(s)> <options>}

\noindent
\begin{longtable}{p{0.24\textwidth}p{0.704\textwidth}}
  \toprule
  \multicolumn{2}{l}{Mandatory arguments} \\
  \midrule
  \texttt{<input>} & input coordinate file (either \texttt{vcf} or
    \texttt{vtf} format) \\
  \texttt{<input.agg>} & input \texttt{.agg} file \\
  \texttt{<width>} & width of each bin of the distribution \\
  \texttt{<output.rho>} & output file(s) (one per aggregate size) with
    automatic \texttt{\#.rho} ending (\texttt{\#} is aggregate size) \\
  \texttt{<agg size(s)>} & aggregate size(s) (the number of molecules in an
    aggregate or the aggregation number, $A_{\mathrm{S}}$) to calculcate
    density for \\
  \toprule
  \multicolumn{2}{l}{Non-standard options} \\
  \midrule
  \texttt{--joined} & specify that \texttt{<input>} contains joined
    coordinates (i.e., periodic boundary conditions for aggregates do not
    have to be removed) \\
  \texttt{-n <int>} & number of bins to average to get smoother density
    (default: 1) \\
  \texttt{-st <int>} & starting timestep for calculation (default: 1) \\
  \texttt{-m <mol name(s)>} & instead of `true' aggregate size, use the number
    of specified molecule type(s) in an aggregate \\
  \texttt{-x <mol name(s)>} & exclude specified molecule type(s) (i.e., do
    not calculate density for beads in molecules \texttt{<mol name(s)>}) \\
  \bottomrule
\end{longtable}

\noindent
Format of output files:
\begin{enumerate}[nosep,leftmargin=20pt]
  \item \texttt{<output.rho>} -- bead densities for one aggregate size
    \begin{itemize}[nosep,leftmargin=5pt]
      \item first line: command used to generate the file
      \item second line: the order of data columns for each bead type --
        \texttt{rdp} is radial density profile, \texttt{rnp} radial number
        profile and \texttt{stderr} are one-$\sigma$ errors for \texttt{rdp}
        and \texttt{rnp}
      \item third line: column headers
        \begin{itemize}[nosep,leftmargin=10pt]
          \item first is the centre of each bin (governed by
            \texttt{<width>}); i.e., if \texttt{<width>} is 0.1,
            then the centre of bin 0 to 0.1 is 0.05, centre of bin 0.1 to
            0.2 is 0.15, etc.
          \item the rest are for the calculated data: each number specifies
            the first column with data for the given bead type (i.e.,
            \texttt{rdp} column)
          \item last line contains the total number of aggregates the
            density was calculated for
        \end{itemize}
    \end{itemize}
\end{enumerate}
