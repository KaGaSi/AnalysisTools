\section{DistrAgg} \label{sec:DistrAgg}

This utility calculates average aggregate mass and aggregation number for
each timestep (i.e., time evolution) and the averages over all timesteps.
It calculates number, weight, and z averages. It also calculates
distribution functions of aggregation sizes and volumes.

For a quantity $\mathcal{O}$, the number, weight, and z averages,
$\langle\mathcal{O}\rangle_{\mathrm{n}}$,
$\langle\mathcal{O}\rangle_{\mathrm{w}}$, and
$\langle\mathcal{O}\rangle_{\mathrm{z}}$, respectively, are defined as
\begin{equation} \label{eq:Avg}
  \langle\mathcal{O}\rangle_{\mathrm{n}} = \frac{\sum_i N_i\mathcal{O}_i     }{N}
  \mbox{, \ \ \ }
  \langle\mathcal{O}\rangle_{\mathrm{w}} = \frac{\sum_i N_im_i\mathcal{O}_i  }{\sum_i N_i m_i}
  \mbox{, and \ \ \ }
  \langle\mathcal{O}\rangle_{\mathrm{z}} = \frac{\sum_i N_im_i^2\mathcal{O}_i}{\sum_i N_i m_i^2},
\end{equation}
where $N$ is the total number of measurements, i.e., the total number of
aggregates for per-aggregate averages (or molecules for per-molecule
averages); $N_i$ is the number of measurements with the value
$\mathcal{O}_i$, and $m_i$ is mass of an aggregate $i$ (or a molecule $i$).

Per-timestep averages are written to the \texttt{<output avg>} and overall
averages are appended as comments (with commented legend) to both
\texttt{<output avg>} and \texttt{<output distr>} files.

Number, weight, and z distribution functions of aggregate sizes,
$F_{\mathrm{n}}(A_{\mathrm{S}})$, $F_{\mathrm{w}}(A_{\mathrm{S}})$, and
$F_{\mathrm{z}}(A_{\mathrm{S}})$ respectively, are defined as
\begin{equation} \label{eq:Fnwz}
  \arraycolsep=1.4pt\def\arraystretch{2.5}
  \begin{array}{>{\displaystyle}rc>{\displaystyle}l}
    F_{\mathrm{n}} & = & \frac{N_{A_{\mathrm{S}} }}{\sum_{A_{\mathrm{S}} } N_i} =
    \frac{N_{A_{\mathrm{S}} }}{N}
  \mbox{,} \\
    F_{\mathrm{w}} & = & \frac{N_{A_{\mathrm{S}} } m_{A_{\mathrm{S}} }}{\sum_{A_{\mathrm{S}} } N_i m_i} =
    \frac{N_{A_{\mathrm{S}} } m_{A_{\mathrm{S}} }}{\sum_{i=1}^N m_i} =
    \frac{N_{A_{\mathrm{S}} } m_{A_{\mathrm{S}} }}{M}
  \mbox{, and} \\
    F_{\mathrm{z}} & = & \frac{N_{A_{\mathrm{S}} } m^2_{A_{\mathrm{S}}
    }}{\sum_{A_{\mathrm{S}} } N_i m_i^2} =
    \frac{N_{A_{\mathrm{S}} } m^2_{A_{\mathrm{S}} }}{\sum_{i=1}^N m_i^2}, \\
  \end{array}
\end{equation}
where $N_{A_{\mathrm{S}}}$ and $m_{A_{\mathrm{S}}}$ stand for the number
and mass, respectively, of aggregates with aggregate size $A_{\mathrm{S}}$;
$M$ is the total mass of all aggregates. The equations are normalised so
that $\sum F_x(A_{\mathrm{S}})=1$.

Distribution of volume fractions of aggregates, $\phi(A_{\mathrm{S}})$, is
defined (assuming all beads have the same volume) as

\begin{equation} \label{eq:Fvol}
  \phi(A_{\mathrm{S}}) = \frac{N_{A_{\mathrm{S}} } m_{A_{\mathrm{S}} }}{\sum_{i=1}^N n_i} =
  \frac{N_{A_{\mathrm{S}} } m_{A_{\mathrm{S}} }}{n},
\end{equation}
where $n_i$ is the number of beads in aggregate $i$ and $n$ is the total
number of beads in all aggregates. If all beads have unit mass (as is often
the case in dissipative particle dynamics), the volume distribution,
$\phi(A_{\mathrm{S}})$, is the equal to the number distribution,
$F_{\mathrm{n}}(A_{\mathrm{S}})$.
These distribution are written into the \texttt{<output distr>} file.

Lastly, \texttt{DistrAgg} can calculate number distribution of composition
for aggregates with specified size(s) (\texttt{-c} option). This is a number
distribution of the numbers of different molecule types in the aggregate.
For example, if the simulation box contains molecule types \texttt{Mol\_A}
and \texttt{Mol\_B}, aggregates with the same size can contain different
numbers of these molecules, or different ratios of the numbers of
\texttt{Mol\_A} to \texttt{Mol\_B} molecules,
$\xi=N_{\mathrm{Mol\_A}}/N_{\mathrm{Mol\_B}}$. For now, \texttt{DistriAgg}
can calculate this distribution only for aggregates containing two molecule
types. The composition distribution is defined as
\begin{equation} \label{eq:CompDistr}
  F_{\mathrm{n}}(\xi) = \frac{N_{\xi,A_{\mathrm{S}} }}{N_{A_{\mathrm{S}}} },
\end{equation}
where $N_{\xi,A_{\mathrm{S}} }$ is the number of aggregate with aggregate
size $A_{\mathrm{S}}$ and ratio $\xi$; $N_{A_{\mathrm{S}}}$ is the total
number of aggregates with aggregate size $A_{\mathrm{S}}$.

The definition of aggregate size is flexible. If none of \texttt{-m},
\texttt{-x}, \texttt{--only} options is used, aggregate size is the `true'
aggregation number, $A_{\mathrm{S}}$, i.e., the number of all molecules in
the aggregate; if \texttt{-m} is used, aggregate size is the sum of only
specified molecule type(s); if \texttt{-x} is used, aggregates containing
only specified molecule type(s) are disregarded; if \texttt{--only} is
used, only aggregates composed of the specified molecule type(s) are taken
into account. For example, consider a system containing three aggregates
composed of various numbers of three different molecule types:
\noindent
\begin{longtable}{l|l}
  \toprule
  Molecule types & Aggregate composition \\
  \midrule
  \texttt{Mol\_A} & \texttt{Agg\_1}: 1 \texttt{Mol\_A} $+2$ \texttt{Mol\_B} $+3$ \texttt{Mol\_C} $=6$ molecules \\
  \texttt{Mol\_B} & \texttt{Agg\_2}: 1 \texttt{Mol\_A} $+2$ \texttt{Mol\_B} $=3$ molecules \\
  \texttt{Mol\_C} & \texttt{Agg\_3}: 1 \texttt{Mol\_A} $=1$ molecule \\
  \bottomrule
\end{longtable}
\noindent
Here is a list of some of the possibilities depending on the option(s)
used:
\begin{enumerate}[nosep]
  \item if none of \texttt{-m}, \texttt{-x}, \texttt{--only} is used, all
    three aggregates are counted and their sizes are their aggregation
    numbers, i.e., $A_{\mathrm{S}}=6$, 3, and 1
  \item if \texttt{-m Mol\_A Mol\_B} is used, all three aggregates are
    counted, but their size is the sum of only \texttt{Mol\_A} and
    \texttt{Mol\_B} molecules: \texttt{Agg\_1} -- 3; \texttt{Agg\_2} -- 3;
    \texttt{Agg\_3} -- 1
  \item if \texttt{-m Mol\_B Mol\_C} is used, \texttt{Agg\_3} is not
    counted, because its size would be zero; \texttt{DistrAgg} would detect
    only two aggregates with sizes: \texttt{Agg\_1} -- 5; \texttt{Agg\_2} --
    2
  \item if \texttt{-x Mol\_A Mol\_B} is used, \texttt{Agg\_2} and
    \texttt{Agg\_3} are not counted, because neither contains anything else
    than \texttt{Mol\_A} and/or \texttt{Mol\_B}; \texttt{DistrAgg} would
    detect only one aggregate with size: \texttt{Agg\_1} -- 6
  \item if \texttt{-x Mol\_A Mol\_B} is combined with \texttt{-m Mol\_A
    Mol\_B}, \texttt{DistrAgg} would again detect only \texttt{Agg\_1}, but
    its size would be taken as 3
  \item if \texttt{--only Mol\_A Mol\_B} is used, \texttt{Agg\_1} is not
    counted, because it contains a molecule not specified by
    \texttt{--only}; \texttt{DistrAgg} would detect only two aggregates
    with sizes: \texttt{Agg\_2} -- 3; \texttt{Agg\_3} -- 1
  \item if \texttt{--only Mol\_A Mol\_B} is combined with \texttt{-m
    Mol\_A}, the two detected aggregates have sizes: \texttt{Agg\_2} -- 1;
    \texttt{Agg\_3} -- 1
  \item if \texttt{--only Mol\_A Mol\_B} is combined with \texttt{-x
    Mol\_A}, only \texttt{Agg\_2} is detected as it is the only composed of
    only \texttt{Mol\_A} and \texttt{Mol\_B} molecules
\end{enumerate}

Usage:

\vspace{1em}
\noindent
\texttt{DistrAgg <input.agg> <distr file> <avg file> <options>}

\noindent
\begin{longtable}{p{0.30\textwidth}p{0.644\textwidth}}
  \toprule
  \multicolumn{2}{l}{Mandatory arguments} \\
  \midrule
  \texttt{<input.agg>} & input \texttt{agg} file \\
  \texttt{<distr file>} & output file with distribution of aggregate
    sizes \\
  \texttt{<avg file>} & output file with per-timestep averages \\
  \toprule
  \multicolumn{2}{l}{Non-standard options} \\
  \midrule
  \texttt{-st <int>} & starting timestep for calculation (default: 1) \\
  \texttt{-n <int> <int>} & use aggregate sizes in a given range \\
  \texttt{-m <mol name(s)>} & use number of specified molecule(s) as
    aggregate size \\
  \texttt{-x <mol name(s)>} & exclude aggregates containing only specified
    mole\-cule(s) \\
  \texttt{--only <mol name(s)>} & use only aggregates composed of specified
    molecule(s) \\
  \texttt{-c <output> <int(s)>} & save composition distribution for
    specified aggregate size(s) to \texttt{<output>} file \\
  \bottomrule
\end{longtable}

\noindent
Format of output files:
\begin{enumerate}[nosep,leftmargin=20pt]
  \item \texttt{<output distr>} -- distributions of aggregate sizes
    \begin{itemize}[nosep,leftmargin=5pt]
      \item first line: command used to generate the file
      \item second line: column headers
        \begin{itemize}[nosep,leftmargin=10pt]
          \item first is the aggregate size, \texttt{As} -- either true aggregation
            number, or the size specified by options
          \item \texttt{F\_n(As)}, \texttt{F\_w(As)}, and \texttt{F\_z(As)} are
            number, weight, and z distribution of aggregate sizes (Equation
            \eqref{eq:Fnwz})
          \item \texttt{<volume distribution>} is distribution according
            to Equation \eqref{eq:Fvol}
          \item next is the total number of aggregates with specified size
          \item the remaining columns show average numbers of every molecule
            type in an aggregate with the specified size
        \end{itemize}
      \item second to last line: column headers for overall averages
        written in the last line
        \begin{itemize}[nosep,leftmargin=10pt]
          \item \texttt{<M>\_n}, \texttt{<M>\_w}, and \texttt{<M>\_z} are
            number, weight, and z averages, respectively, of aggregate
            masses (the averages are defined in Equation~\eqref{eq:Avg})
          \item a column denoted \texttt{<mol name>\_n} shows an average
            number of molecules named \texttt{mol name} in an aggregate
        \end{itemize}
    \end{itemize}
  \item \texttt{<output avg>} -- per-timestep averages
  \begin{itemize}[nosep,leftmargin=5pt]
    \item first line: command used to generate the file
    \item second lines: column headers
      \begin{itemize}[nosep,leftmargin=10pt]
        \item first is simulation timestep
        \item the rest are for the calculated data: number, weight, and z
          average aggregate mass (\texttt{<M>\_n}, \texttt{<M>\_w}, and
          \texttt{<M>\_z}, respectively) and aggregate size
          (\texttt{<As>\_n}, \texttt{<As>\_w}, and \texttt{<As>\_z},
          respectively)
      \end{itemize}
    \item the last two lines are the same as in \texttt{<output distr>}
  \end{itemize}
\item \texttt{-c <name>} -- composition distribution
  \begin{itemize}[nosep,leftmargin=5pt]
    \item first line: command used to generate the file
    \item second lines: column headers
      \begin{itemize}[nosep,leftmargin=10pt]
        \item first is ratio of the two molecule types (i.e., 0 to 1)
        \item the rest are aggregate sizes
        \item in the data, only ratios that are non-zero for at least one
          aggregate size are written; in case of more than one aggregate
          size specified by \texttt{-a} option, if the ratio does not exist
          for some aggregate size(s), `?' is displayed instead of zero
      \end{itemize}
  \end{itemize}
\end{enumerate}
