\section{DistrAgg} \label{sec:DistrAgg}

This utility calculates average aggregate mass and aggregation number for
each timestep (i.e., time evolution) and the averages over all timesteps
from a supplied \tt{agg} file (see Section~\ref{sec:AggFile} for its format).
It calculates number, weight, and z averages. It also calculates
distribution functions of aggregation sizes.

Generally, for a quantity $\mathcal{O}$, the number, weight, and z averages,
$\langle\mathcal{O}\rangle_{\text{n}}$,
$\langle\mathcal{O}\rangle_{\text{w}}$, and
$\langle\mathcal{O}\rangle_{\text{z}}$, respectively, are defined as
\begin{equation} \label{eq:Avg}
  \langle\mathcal{O}\rangle_{\text{n}} = \frac{\sum_i N_i\mathcal{O}_i     }{N}
  \mbox{, \ \ \ }
  \langle\mathcal{O}\rangle_{\text{w}} = \frac{\sum_i N_im_i\mathcal{O}_i  }{\sum_i N_i m_i}
  \mbox{, and \ \ \ }
  \langle\mathcal{O}\rangle_{\text{z}} = \frac{\sum_i N_im_i^2\mathcal{O}_i}{\sum_i N_i m_i^2},
\end{equation}
where $N$ is the total number of measurements, i.e., the total number of
aggregates for per-aggregate averages (or molecules for per-molecule
averages); $N_i$ is the number of measurements with the value
$\mathcal{O}_i$, and $m_i$ is mass of an aggregate $i$ (or a molecule $i$).

Number, weight, and z distribution functions of aggregate sizes,
$F_{\text{n}}(A_{\text{S}})$, $F_{\text{w}}(A_{\text{S}})$, and
$F_{\text{z}}(A_{\text{S}})$, respectively, are defined as
\begin{equation} \label{eq:Fnwz}
  \arraycolsep=1.4pt\def\arraystretch{2.5}
  \begin{array}{>{\displaystyle}rc>{\displaystyle}l}
    F_{\text{n}}(A_\text{S}) & = & \frac{N_{A_{\text{S}} }}{\sum_{A_{\text{S}} } N_i} =
    \frac{N_{A_{\text{S}} }}{N}
  \mbox{,} \\
    F_{\text{w}}(A_\text{S}) & = & \frac{N_{A_{\text{S}} } m_{A_{\text{S}} }}{\sum_{A_{\text{S}} } N_i m_i} =
    \frac{N_{A_{\text{S}} } m_{A_{\text{S}} }}{\sum_{i=1}^N m_i} =
    \frac{N_{A_{\text{S}} } m_{A_{\text{S}} }}{M}
  \mbox{, and} \\
    F_{\text{z}}(A_\text{S}) & = & \frac{N_{A_{\text{S}} } m^2_{A_{\text{S}}
    }}{\sum_{A_{\text{S}} } N_i m_i^2} =
    \frac{N_{A_{\text{S}} } m^2_{A_{\text{S}} }}{\sum_{i=1}^N m_i^2}, \\
  \end{array}
\end{equation}
where $N_{A_{\text{S}}}$ and $m_{A_{\text{S}}}$ stand for the number
and mass, respectively, of aggregates with aggregate size $A_{\text{S}}$;
$M$ is the total mass of all aggregates. The equations are normalized so
that $\sum F_x(A_{\text{S}})=1$.

Per-timestep averages are written to the \tt{<output avg>} and  distributions
into the \tt{<output distr>} file. Overall averages are appended as comments
(with commented legend) to both \tt{<output avg>} and \tt{<output distr>} files.

Lastly, \tt{DistrAgg} can calculate distribution of composition for aggregates
with specified size(s) (\tt{-c} option). Two versions of a \enquote{composition
distribution} are generated. The first is the distribution of numbers of each
molecule type in the aggregates of that size. The second is a distribution of
ratios of all possible molecular pairs in those aggregates The distribution of
numbers of each molecule type is written into \tt{<file>-<size>.txt} file, and
the distribution of all ratios of all possible bead pairs is written into
\tt{<file>-ratio_<size>.txt} file; that is, two files are created for each
aggregate size. In both cases, the number distribution for aggregates with
aggregation number $A_\text{S}$, $F_{A_\text{S}}(i)$, is defined as
%
\begin{equation}
  F_{A_\text{S}}(i) = \frac{N_{A_\text{S},i}}{N_{A_\text{S}}},
\end{equation}
%
where $N_{A_\text{S}}$ is the total number of
aggregates with aggregation number $A_\text{S}$. The
$N_{A_\text{S},i}$ is the number of aggregates with size $A_\text{S}$ that
either contain $i$ molecules of given type (the first distribution type) or has
the ratio of molecules \tt{mol1} and \tt{mol2}; i.e., $i=$ \tt{mol1}$/$\tt{mol2}
(the second distribution type).

The \tt{<avg file>} contains averages for all timesteps regardless of \tt{-st},
\tt{-e}, and \tt{-sk} options. The starting and ending timesteps as well as the
number of skipped timesteps are taken into account for all the distributions and
overall averages.

The definition of aggregate size is flexible. If none of \tt{-m}, \tt{-x}, or
\tt{-only} options is used, aggregate size is the \enquote{true} aggregation
number, i.e., the number of all molecules in the aggregate; if \tt{-m} is used,
aggregate size is the sum of only specified molecule type(s); if \tt{-x} is
used, aggregates containing only specified molecule type(s) are disregarded; if
\tt{-only} is used, only aggregates composed of the specified molecule type(s)
are taken into account. These options can be mixed. For example, consider a
system containing three aggregates composed of various numbers of three
different molecule types:

\begin{longtable}{c|l}
  \toprule
  Molecule types & \multicolumn{1}{c}{Aggregate composition} \\
  \midrule
  \tt{Mol_A} & \tt{Agg_1}: 1 \tt{Mol_A} $+2$ \tt{Mol_B} $+3$ \tt{Mol_C} $=6$ molecules \\
  \tt{Mol_B} & \tt{Agg_2}: 1 \tt{Mol_A} $+2$ \tt{Mol_B} $=3$ molecules \\
  \tt{Mol_C} & \tt{Agg_3}: 1 \tt{Mol_A} $=1$ molecule \\
  \bottomrule
\end{longtable}

\noindent
Here is a list of some of the possibilities depending on the option(s)
used:
\begin{enumerate}[nosep]
  \item if none of \tt{-m}, \tt{-x}, \tt{-only} is used, all three aggregates
    are counted and their sizes are their \enquote{true} aggregation numbers,
    i.e., $A_{\text{S}}=6$, 3, and 1
  \item if \tt{-m Mol_A Mol_B} is used, all three aggregates are
    counted, but their size is the sum of only \tt{Mol_A} and
    \tt{Mol_B} molecules: \tt{Agg_1} -- 3; \tt{Agg_2} -- 3;
    \tt{Agg_3} -- 1
  \item if \tt{-m Mol_B Mol_C} is used, \tt{Agg_3} is not
    counted, because its size would be zero; \tt{DistrAgg} would detect
    only two aggregates with sizes: \tt{Agg_1} -- 5; \tt{Agg_2} --
    2
  \item if \tt{-x Mol_A Mol_B} is used, \tt{Agg_2} and
    \tt{Agg_3} are not counted, because neither contains anything else
    than \tt{Mol_A} and/or \tt{Mol_B}; \tt{DistrAgg} would
    detect only one aggregate with size: \tt{Agg_1} -- 6
  \item if \tt{-x Mol_A Mol_B} is combined with \tt{-m Mol_A
    Mol_B}, \tt{DistrAgg} would again detect only \tt{Agg_1}, but
    its size would be 3
  \item if \tt{-only Mol_A Mol_B} is used, \tt{Agg_1} is not
    counted, because it contains a molecule not specified by
    \tt{-only}; \tt{DistrAgg} would detect two aggregates
    with sizes: \tt{Agg_2} -- 3; \tt{Agg_3} -- 1
  \item if \tt{-only Mol_A Mol_B} is combined with \tt{-m
    Mol_A}, the two detected aggregates have sizes: \tt{Agg_2} -- 1;
    \tt{Agg_3} -- 1
  \item if \tt{-only Mol_A Mol_B} is combined with \tt{-x
    Mol_A}, only \tt{Agg_2} is detected as it is the only one composed of
    only \tt{Mol_A} and \tt{Mol_B} molecules and its size would
    be 3
  \item if \tt{-only Mol_A Mol_B} and \tt{-x Mol_A} are combined
    with \tt{-m Mol_A}, the size of the one detected aggregate would be 1
\end{enumerate}
\vspace{1em}

Note that aggregate mass is always taken as the total mass, e.g., in the
above points 8) and 9), the mass of the one detected aggregate would be the sum
of masses of all the molecules in the aggregate even though the size
is defined differently.

Should the \tt{-c} option be used (without any of the \tt{-x}, \tt{-m}, or
\tt{-only} options), the output \tt{<file>-<size>.txt} file would contain three
data columns, one for each molecule type; the output
\tt{<file>-ratio_<size>.txt} file would contain three columns for the three
ratios, that is \tt{Mol_A}$/$\tt{Mol_B}, \tt{Mol_A}$/$\tt{Mol_C}, and
\tt{Mol_B}$/$\tt{Mol_C}.

Moreover, only a specified range of aggregate sizes can be taken into
account (\tt{-n <int> <int>} option). These sizes are defined by the
\tt{-m}, \tt{-x}, and \tt{-only} options as well.

Usage:

\vspace{1em}
\noindent
\tt{DistrAgg <input.agg> <distr file> <avg file> <options>}

\noindent
\begin{longtable}{p{0.30\textwidth}p{0.644\textwidth}}
  \toprule
  \multicolumn{2}{l}{Mandatory arguments} \\
  \midrule
  \tt{<input>} & input structure file \\
  \tt{<input.agg>} & input \tt{agg} file \\
  \tt{<distr file>} & output file with distribution of aggregate
    sizes \\
  \tt{<avg file>} & output file with per-timestep averages \\
  \toprule
  \multicolumn{2}{l}{Non-standard options} \\
  \midrule
  \tt{-n <int> <int>} & use aggregate sizes in a given range \\
  \tt{-m <mol name(s)>} & use number of specified molecule(s) as
    aggregate size \\
  \tt{-x <mol name(s)>} & exclude aggregates containing only specified
    mole\-cule(s) \\
  \tt{-only <mol name(s)>} & use only aggregates composed of specified
    molecule(s) \\
  \tt{-c <file> <int(s)>} & save composition distribution for
    specified aggregate size(s) to \tt{<output>} file \\
  \midrule
  \multicolumn{2}{l}{Other options (see the beginning of 
                     Chapter~\ref{chap:Utils})}\\
  \midrule
  \multicolumn{2}{l}{\tt{-st},
                     \tt{-e},
                     \tt{-sk},
                     \tt{--help},
                     \tt{--verbose},
                     \tt{--silent},
                     \tt{--version}}\\
  \bottomrule
\end{longtable}

\noindent
Format of output files:
\begin{enumerate}[nosep,leftmargin=20pt]
  \item \tt{<output distr>} -- distributions of aggregate sizes
    \begin{itemize}[nosep,leftmargin=5pt]
      \item first line: AnalysisTools version
      \item second line: command used to generate the file
      \item third line: column headers
        \begin{itemize}[nosep,leftmargin=10pt]
          \item first is the aggregate size, \tt{As} -- either true aggregation
            number or the size specified by options
          \item \tt{F_n(As)}, \tt{F_w(As)}, and \tt{F_z(As)} are
            number, weight, and z distribution of aggregate sizes (Equation
            \eqref{eq:Fnwz})
          \item next is the total number of aggregates with specified size (sum
            from all timesteps)
          \item the remaining columns (\tt{<mol name>_n}) show average
            numbers of every molecule type in an aggregate with the specified
            size
        \end{itemize}
      \item data lines follow
      \item second to last line: column headers for overall averages
        \begin{itemize}[nosep,leftmargin=10pt]
          \item \tt{<As>_n}, \tt{<As>_w}, and \tt{<As>_z} are
            number, weight, and z averages, respectively, of aggregate
            numbers (see Equation~\eqref{eq:Avg} for definitions)
          \item \tt{<M>_n}, \tt{<M>_w}, and \tt{<M>_z} are
            number, weight, and z averages, respectively, of aggregate
            masses (see Equation~\eqref{eq:Avg} for definitions)
          \item next are average numbers of every molecule type in an aggregate
            with the specified size (\tt{<mol name>_n})
          \item average number of aggregates per timestep, \tt{<n_agg>}
        \end{itemize}
      \item last line: the overall averages
    \end{itemize}
  \item \tt{<output avg>} -- per-timestep averages
    \begin{itemize}[nosep,leftmargin=5pt]
      \item first line: AnalysisTools version
      \item second line: command used to generate the file
      \item third line: column headers
      \begin{itemize}[nosep,leftmargin=10pt]
        \item first is simulation timestep
        \item the calculated data follow: number, weight, and z average
          aggregate size (\tt{<As>_n}, \tt{<As>_w}, and \tt{<As>_z},
          respectively) and mass (\tt{<M>_n}, \tt{<M>_w}, and \tt{<M>_z},
          respectively)
        \item the last column is the number of aggregates in the given step
      \end{itemize}
    \item data lines follow
    \item the last two lines are the same as in \tt{<output distr>}
  \end{itemize}
\item \tt{<file>-<size>.txt} from \tt{-c} option -- composition distribution of
  numbers of molecules
  \begin{itemize}[nosep,leftmargin=5pt]
    \item first line: AnalysisTools version
    \item second line: command used to generate the file
    \item third line: number of aggregates of given size (sum from all
      timesteps)
    \item fourth line: column headers
      \begin{itemize}[nosep,leftmargin=10pt]
        \item first is the number of molecules in the aggregate
        \item the rest are the number distributions of for each molecule type in
          the given aggregate size
      \end{itemize}
    \item data lines follow
  \end{itemize}
\item \tt{<file>-ratio_<size>.txt} from \tt{-c} option -- composition
  distribution of ratios of molecular pairs
  \begin{itemize}[nosep,leftmargin=5pt]
    \item first line: AnalysisTools version
    \item second line: command used to generate the file
    \item third line: number of aggregates of given size (sum from all
      timesteps)
    \item fourth line: column headers
      \begin{itemize}[nosep,leftmargin=10pt]
        \item first is the ratio of the two molecules (going from 0 to aggregate
          size with the interval of 0.1)
        \item the rest are the ratios of all molecular pairs
      \end{itemize}
    \item data lines follow
  \end{itemize}
\end{enumerate}
