\section{AddToSystem} \label{sec:AddToSystem}

This utility takes an existing system specified by \texttt{vtf} file(s) and
adds new beads into it, placing them either randomly or using provided
coordinates. The utility requires an input coordinate file containing all
beads (i.e., all beads defined in the structure file must be present in the
coordinate). If an incomplete coordinate file is provided,
\texttt{AddToSystem} exhibits undefined behaviour, i.e., it can run without
problems, exit with an error, or freeze.

The utility has two modes: by default, it reads what is to be added from a
\texttt{FIELD}-like file, placing the new beads and molecules randomly
(Subsection~\ref{ssec:random}); if, however, \texttt{-vtf} option is used,
the utility reads the provided \texttt{vsf} and \texttt{vcf} files, using
coordinates from those (Subsection~\ref{ssec:vtf}).

\subsection{Random placement} \label{ssec:random}

The random coordinates of added beads can be constrained via several
options; for added molecules, either the molecule's first bead (default
behaviour) or its geometric centre (\texttt{-gc} option) obey the
constraints. The coordinates of the remaining beads in the molecule are
governed by the coordinates in the \texttt{FIELD}-like file. Therefore,
not all the molecular beads necessarily obey the constraining rules.
Molecules are added with a random orientation.

There are two types of constraints which can be combined: place new beads
(i) specified distance from other beads or (ii) in a specified interval in
x, y, and/or z directions. In (i), options \texttt{-ld} and/or \texttt{-hd}
specify the distance; if present, these must be accompanied by \texttt{-bt}
option.  Then, the new beads are placed at least \texttt{-ld <float>} and
at most \texttt{-hd <float>} distance from beads specified by the
\texttt{-bt} option. In (ii), options \texttt{-cx}, \texttt{-cy}, and
\texttt{-cz} basically change the box size for the added beads; for
example, \texttt{-cx 5 9} would generate x coordinates only in the interval
$\langle5,9)$. All these options can be combined, but note that
\texttt{AddToSystem} does not perform any sanity checks, that is, if any
combination of the provided options is impossible to achieve, the utility
will run forever.

As stated above, the structure and number of added molecules and monomeric
beads are read from a \texttt{FIELD}-like file. This file must contain a
\texttt{species} section followed by a \texttt{molecule} section as
described in the DL\_MESO simulation package. Following is a short
description of the two sections.

The \texttt{species} section contains the number of bead types and their
properties:
\begin{verbatim}
  species <int>
  <name>  <mass> <charge> <number of unbonded beads>
\end{verbatim}
The first line must start with \texttt{species} keyword followed by the
number of bead types. For each bead type, a single line must contain the
name of the bead, its mass and charge, and a number of these beads that are
not in a molecule (i.e., monomeric or unbonded beads). No blank lines are
allowed in the section.

The \texttt{molecule} section, which comes after the \texttt{species}
section, contains information about structure and numbers of molecules to
be added:

\noindent
\begin{longtable}{ll}
  \texttt{molecule <int>} & Number of types of molecules \\
  \texttt{<name>} & Name of the first molecule type \\
  \texttt{nummols <int>} & Number of \texttt{<name>} molecules \\
  \texttt{beads <int>} & Number of beads in each \texttt{<name>} \\
  \texttt{<bead name> <x> <y> <z>} & One line for each of the \texttt{<int>} beads, \\
  \texttt{...} & \ \ \ \ specifying bead name and its \\
  \texttt{<bead name> <x> <y> <z>} & \ \ \ \ Cartesian coordinates \\
  \texttt{bonds <int>} & Number of bonds in \texttt{<name>} \\
  \texttt{harm <indices> <length> <strength>} & One line for each of the \texttt{<int>} bonds \\
  \texttt{...} & \ \ \ \ containing two bead indices and \\
  \texttt{harm <indices> <length> <strength>} & \ \ \ \ bond length and strength \\
  \texttt{angles <int>} & Number of angles in \texttt{<name>} (optional) \\
  \texttt{harm <indices> <angle> <strength>} & One line for each of the \texttt{<int>} angles \\
  \texttt{...} & \ \ \ \ containing three bead indices, \\
  \texttt{harm <indices> <angle> <strength>} & \ \ \ \ angle and angle strength \\
  \texttt{...} & Anything beyond here is ignored \\
  \texttt{finish} & End of \texttt{<name>}'s description \\
\end{longtable}

The \texttt{molecule} keyword specifies the number of molecule types, that
is the number of \texttt{finish} keywords that must be present. The
\texttt{<bead name>} must be present in the \texttt{species} section. The
\texttt{harm} string in the \texttt{bond} and \texttt{angle} lines is
ignored as \texttt{AddToSystem} assumes harmonic potential for both. The
indices in \texttt{bond} and \texttt{angle} lines run from 1 to the number
of beads in the molecules and are ordered according to the \texttt{beads}
part of the section. Anything else (such as dihedral angles) is ignored.
While a molecule does not have to have angles, at least one bond per
molecule is required. If no molecules are to be added, the line
\texttt{molecule 0} is required. Examples of \texttt{FIELD}-like files are
in the \texttt{Examples/AddToSystem} directory.

\subsection{Provided coordinates} \label{ssec:vtf}

If \texttt{-vtf} option is used, provided \texttt{vsf} and \texttt{vcf}
files are read instead of a \texttt{FIELD}-like file. \texttt{AddToSystem}
incorporates the beads and molecules from these files, using the provided
coordinates instead of generating new ones.

The \texttt{-vtf} option has priority over default behaviour, that is, if
it is present, the utility ignores a \texttt{FIELD}-like file as well as
the constraining options.

The side lengths of the new simulation box are always taken as the higher
ones from the original system and the system to be added.

\vspace{1em}

In both modes, the new unbonded beads appear in the resulting files after
the unbonded beads from the original system (but before any bonded beads).
If molecules are added, \texttt{AddToSystem} places them at the end.

By default, the new beads are added to the system, increasing the total
number of beads. However, if \texttt{--switch} is used, the new beads
replace beads in the original system. The replaced beads are either neutral
beads with the lowest indices (as per ordering in the provided structure
file) or beads of the type specified using \texttt{-xb} option. In both
cases, the beads to be replaced must be unbonded.

The size of the new simulation box can be changed using the \texttt{-b}
option. There are no constraints on the size of the cuboid box; if the new
box is small, not all beads from the original system are necesserily inside
the box (although the added once always are). The origin of the new box
coincides with the origin of the original box unless \texttt{--centre}
option is used; in that case, the centres of the two boxes coincide.

\texttt{AddToSystem} creates \texttt{vcf} and \texttt{vsf} files containing
the new system; the coordinates can also be written into an \texttt{xyz}
file.

Several examples of using the utility are provided in the
\texttt{Examples/AddToSystem} directory.

Usage:

\vspace{1em}
\noindent
\texttt{AddToSystem <input.vcf> <out.vsf> <out.vcf> <options>}

\vspace{1em}
\noindent
\begin{longtable}{p{0.235\textwidth}p{0.709\textwidth}}
  \toprule
  \multicolumn{2}{l}{Mandatory arguments} \\
  \midrule
  \texttt{<input.vcf>} & input coordinate file (either \texttt{vcf} or
    \texttt{vtf} format) \\
  \texttt{<out.vsf>} & output \texttt{vsf} structure file for the new
    system \\
  \texttt{<out.vcf>} & output \texttt{vcf} coordinate file for the new
    system \\
  \toprule
  \multicolumn{2}{l}{Non-standard options} \\
  \midrule
  \texttt{-st <int>} & \texttt{<input.vcf>} timestep to add new beads
    to (default: 1) \\
  \texttt{-xyz <name>} & also save coordinates to \texttt{xyz} file \\
  \texttt{--switch} & replace original beads instead of increasing the
    total number of beads \\
  \texttt{-b <x> <y> <z>} & side lengths of the new simulation box \\
  \texttt{--centre} & place the original simulation box in the middle of
    the new one \\
  \midrule
  \multicolumn{2}{c}{Random placement} \\
  \midrule
  \texttt{-f <name>} & \texttt{FIELD}-like file specifying additions to the
    system (default: \texttt{FIELD}) \\
  \texttt{-ld <float>} & lowest distance from beads specified by
    \texttt{-bt} option \\
  \texttt{-hd <float>} & highest distance from beads specified by
    \texttt{-bt} option \\
  \texttt{-bt <bead names>} & bead types to use in conjunction with
    \texttt{-ld} and/or \texttt{-hd} options \\
  \texttt{-cx <num> <num2>} & constrain x coordinate to interval
    $\langle num, num2)$ \\
  \texttt{-cy <num> <num2>} & constrain y coordinate to interval
    $\langle num, num2)$ \\
  \texttt{-cz <num> <num2>} & constrain z coordinate to interval
    $\langle num, num2)$ \\
  \texttt{-gc} & use molecule's geometric centre for distance check \\
  \midrule
  \multicolumn{2}{c}{Provided coordinates} \\
  \midrule
  \texttt{-vtf <vsf> <vcf>} & add ready-made system from \texttt{vtf} files
    instead of randomly placing beads read from the \texttt{FIELD}-like file \\
  \bottomrule
\end{longtable}
