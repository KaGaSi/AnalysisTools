\section{AddToSystem} \label{sec:AddToSystem}

This utility takes an existing system specified by \texttt{vcf} coordinate
and \texttt{vsf} structure files and adds new beads into it. The new beads
replace neutral unbonded ones with the lowest indices (as ordered in the
\texttt{vsf} file) from the original system. If molecules are added,
\texttt{AddToSystem} places them at the end (for the sake of
\texttt{DL\_MESO} which requires molecules to be after unbonded beads).

The utility generates \texttt{vcf} and \texttt{vsf} files for the new
system.

If there aren't enough neutral unbonded beads in the original system to be
replaced, \texttt{AddToSystem} exits with an error.

The utility has two modes: by default, it reads what is to be added from a
\texttt{FIELD}-like file (see below) and places the new beads and molecules
randomly; if, however, \texttt{-vtf} option is used, the utility reads
provided \texttt{vtf} file(s) and keeps the coordinates from the coordinate
file.

If the new beads and molecules are added randomly, their coordinates are
either completely random within the simulation box or ruled by the
\texttt{-ld}, \texttt{-hd}, and \texttt{-bt} options (either the first bead
or the geometric centre of the new molecules obey these options). The
coordinates of the remaining beads in a molecule are governed by the
coordinates provided in the \texttt{FIELD}-like file. Molecules are added
with a random orientation.  If \texttt{-ld} and/or \texttt{-hd} options are
used, they must accompanied by the \texttt{-bt} option.

As hinted above, the structure and number of added molecules and monomeric
beads are read from a \texttt{FIELD}-like file. This file must contain
\texttt{species} section followed by \texttt{molecule} section as described
in the DL\_MESO simulation package.

The \texttt{species} section contains the number of bead types and their
properties:
\begin{verbatim}
  species <int>
  <name>  <mass> <charge> <number of unbonded beads>
\end{verbatim}
The first line must start with \texttt{species} keyword followed by the
number of bead types. For each bead type a single line must contain the name of
the bead, its mass and charge, and a number of these beads that are not
in a molecule (i.e., monomeric beads).

The \texttt{molecule} section that must be behind the \texttt{species}
section contains information about structure and numbers of molecules to be
added:

\noindent
\begin{longtable}{ll}
  \texttt{molecule <int>} & Number of types of molecules \\
  \texttt{<name>} & Name of the first molecule type \\
  \texttt{nummols <int>} & Number of these molecules \\
  \texttt{beads <int>} & Number of beads in these molecules \\
  \texttt{<bead name> <float> <float> <float>} & One line for each of the \texttt{<int>} beads \\
  \texttt{...} & \ \ \ \ specifying bead name and its \\
  \texttt{<bead name> <float> <float> <float>} & \ \ \ \ Cartesian coordinates \\
  \texttt{bonds <int>} & Number of bonds in these molecules \\
  \texttt{<string> <int> <int>} & One line for each of the \texttt{<int>} bonds \\
  \texttt{...} & \ \ \ \ containing arbitrary string and \\
  \texttt{<string> <int> <int>} & \ \ \ \ indices connected beads \\
  \texttt{...} & Anything beyond here is ignored \\
  \texttt{finish} & Description of a molecule is finished \\
\end{longtable}

The \texttt{molecule} keyword specifies the number of molecule types, that
is the number of \texttt{finish} keywords that must be present. The
\texttt{<bead name>} must be present in the \texttt{species} section. The
arbitrary \texttt{<string>} in the \texttt{bonds} is ignored by
\texttt{AddToSystem} (it is a relic from the \texttt{DL\_MESO} simulation
package, where the \texttt{<string>} specifies a type of bond). The indices
in \texttt{bond} lines run from 1 to the number of beads in the molecules and are
ordered according to the \texttt{beads} part of the section. Because
\texttt{molecule} section in the \texttt{FIELD} file from \texttt{DL\_MESO}
can also include bond angles and dihedral angles, anything beyond the last
bond line is ignored (until the \texttt{finish} keyword is read).

If no molecules are to be added, the line \texttt{molecule 0} must be still
be present in the file.

The following is an example of the \texttt{FIELD}-like file:
\begin{verbatim}
  species 3
  A   1.0  1.0  0
  B   1.0  0.0  0
  CI  1.0 -1.0 30

  molecule 2
  Dimer
  nummols 10
  beads 2
  A 0.0 0.0 0.0
  A 0.5 0.0 0.0
  bonds 1
  harm 1 2
  finish
  surfact
  nummols 10
  beads 3
  A 0.0 0.0 0.0
  B 0.5 0.0 0.0
  B 1.0 0.0 0.0
  bonds 2
  harm 1 2
  harm 2 3
  angles 1
  harm 1 2 3
  finish
\end{verbatim}
In this example, 30 unbonded (or monomeric) negatively charged beads called
\texttt{CI} are added as well as 20 molecules -- 10 molecules called
\texttt{Dimer} and 10 molecules called \texttt{surfact}. \texttt{Dimer}
molecules contain two \texttt{A} beads and one bond each; \texttt{surfact}
molecules contain three beads and two bonds each. The part starting with
\texttt{angles} and ending with \texttt{finish} is ignored. All in all, 80
beads are added -- 30 \texttt{CI}, 30 \texttt{A}, and 20 \texttt{B} beads.

If \texttt{-vtf} option is used, provided \texttt{vsf} and \texttt{vcf}
files are read instead of a \texttt{FIELD}-like file. \texttt{AddToSystem}
incorporates the beads and molecules from these files, using the provided
coordinates instead of generating new ones. Therefore options \texttt{-ld},
\texttt{-hd}, and \texttt{-bt} also have no effect.

The \texttt{-vtf} option has priority over default behaviour, that is, if
it is present, the utility ignores any \texttt{FIELD}-like file (either
\texttt{FIELD} itself or a file provided via \texttt{-f} option).

The utility creates the \texttt{vcf} and \texttt{vsf} files with the new
system and can also write the coordinates into a \texttt{xyz} file.

Usage:

\vspace{1em}
\noindent
\texttt{AddToSystem <input.vcf> <out.vsf> <out.vcf> <options>}

\vspace{1em}
\noindent
\begin{longtable}{p{0.235\textwidth}p{0.709\textwidth}}
  \toprule
  \multicolumn{2}{l}{Mandatory arguments} \\
  \midrule
  \texttt{<input>} & input coordinate file (either \texttt{vcf} or
    \texttt{vtf} format) \\
  \texttt{<out.vsf>} & output \texttt{vsf} structure file for the new system \\
  \texttt{<out.vcf>} & output \texttt{vcf} coordinate file for the new
  system \\
  \toprule
  \multicolumn{2}{l}{Non-standard options} \\
  \midrule
  \texttt{-f <name>} & \texttt{FIELD}-like file specifying additions to the
    system (default: \texttt{FIELD}) \\
  \texttt{-vtf <vsf> <vcf>} & add ready-made system from \texttt{vtf} files
    instead of randomly placing beads read from \texttt{FIELD}-like file \\
  \texttt{-st <int>} & timestep to add new beads to (default: 1) \\
  \texttt{-xyz <name>} & also save coordinates to \texttt{xyz} file \\
  \texttt{-ld <float>} & lowest distance from beads specified by
    \texttt{-bt} option \\
  \texttt{-hd <float>} & highest distance from beads specified by
    \texttt{-bt} option \\
  \texttt{-bt <bead names>} & bead types to use in conjunction with
    \texttt{-ld} and/or \texttt{-hd} options \\
  \bottomrule
\end{longtable}
