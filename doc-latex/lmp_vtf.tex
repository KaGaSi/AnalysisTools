\section{lmp\_vtf} \label{sec:lmp_vtf}

This utility generates \texttt{vsf} and \texttt{vcf} files from a lammps
\texttt{data} file (see
\href{https://lammps.sandia.gov/doc/read_data.html}{lammps manual page} for
details on its format). Note that while lammps uses the word
\enquote{atom}, in this manual, \enquote{bead} is used instead.
\texttt{lmp\_vtf} reads the \texttt{data} file header (it registers only
keywords \texttt{atoms}, \texttt{bonds}, \texttt{angles}, \texttt{atom
types}, \texttt{bond types}, \texttt{angle types}, \texttt{xlo xhi},
\texttt{ylo yho}, and \texttt{zlo zhi}) and \texttt{Masses},
\texttt{Atoms}, \texttt{Bonds}, and \texttt{Angles} section which must be
in this order; everything else (e.g., dihedrals) is ignored.

The utility requires lines in the \texttt{Atoms} section to have the
following format: \texttt{<bead index> <molecule index> <bead type>
<charge> <x> <y> <z>} (i.e., \texttt{atom\_style full} in \texttt{lammps}
terminology).

If any line of the \texttt{Masses} section ends with a comment, its first
string is taken as the name of the bead type. Otherwise, the bead types are
called \texttt{beadN}, where \texttt{N} is their type number in the
\texttt{data} file.

Charge of every bead type is taken as the charge of the last bead of this
type in the \texttt{Atoms} section.

The utility assumes bonds and angles with simple harmonic potential,
$u^{\text{b}}$ and $u^{\text{a}}$, respectively:
\begin{equation}
  u^{\text{b}} = k_{\text{b}}\left(r-r_0\right)^2 \text{ and }
  u^{\text{b}} = k_{\text{a}}\left(\theta-\theta_0\right)^2,
\end{equation}
where $k_{text{b}}$ and $k_{\text{a}}$ are strengths of the potentials and
$r_0$ and $\theta_0$ represent equilibrium distance and angle,
respectively.

Molecule types are determined according to three criteria: (i) the order of
bead types in molecules, (ii) bead connectivity (and bond types), and (iii)
angles between bonds (and angle types). Bead order in every molecule is
considered from the lowest to the highest bead index in the \texttt{data}
file. If \texttt{<molecule index>} in a bead line is 0, this bead is
unbonded (i.e., not part of any molecule). If there's only one bead in a
molecule, this bead is considered unbonded as well. The types of molecules
differing in bead order are called \texttt{molI}, where \texttt{I} goes
from 1 to the total number of molecule types with different bead order. If
two molecules have the same bead order but differ in connectivity (or bond
types), \texttt{-bJ} is appended to the name (\texttt{J} stands for the
number of molecule types differing from the original, i.e., it goes from 1
to the number of molecule types differing in connectivity). Similarly, if
angles are different, \texttt{-aK} is appended with \texttt{K} from 1 to
the number of molecule types differing in the angles.

Note that lines in both \texttt{Atoms} and \texttt{Bonds} sections do not
have to be ordered in any way.

A relatively complex example of a \texttt{data} file and the resulting
\texttt{vsf} and \texttt{vcf} files are in the \texttt{Examples/lmp\_vtf}
directory.

Usage (this utility does not use standard options):

\vspace{1em}
\noindent
\texttt{lmp\_vtf <input> <out.vsf> <out.vcf> <options>}

\noindent
\begin{longtable}{p{0.15\textwidth}p{0.794\textwidth}}
  \toprule
  \multicolumn{2}{l}{Mandatory arguments} \\
  \midrule
  \texttt{<input>} & input lammps \texttt{data} file \\
  \texttt{<out.vsf>} & output \texttt{vsf} structure file \\
  \texttt{<out.vcf>} & output \texttt{vcf} coordinate file \\
  \toprule
  \multicolumn{2}{l}{Options}\\
  \midrule
  \texttt{-v} & verbose output\\
  \texttt{-h} & print this help and exit\\
  \bottomrule
\end{longtable}
