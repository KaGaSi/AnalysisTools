\section{lmp\_vtf} \label{sec:lmp_vtf}

This utility generates \texttt{vsf} and \texttt{vcf} files from lammps
\texttt{data} file (see
\href{https://lammps.sandia.gov/doc/read_data.html}{lammps manual page} for
details on its format). Note that while lammps uses the word
\enquote{atom}, in this manual, \enquote{bead} is used instead.

\texttt{lmp\_vtf} reads the \texttt{data} file header (\texttt{atoms},
\texttt{bonds}, \texttt{atom types}, \texttt{xlo xhi}, \texttt{ylo yhi},
and \texttt{zlo zhi} keywords) and \texttt{Masses}, \texttt{Atoms}, and
\texttt{Bonds} sections; everything else is ignored. The utility requires
lines in the \texttt{Atoms} section to have the following format:
\texttt{<bead index> <molecule index> <bead type> <charge> <x> <y> <z>} (i.e.,
\texttt{atom\_style full} in \texttt{lammps} terminology).

If any line of the \texttt{Masses} section ends with a comment, its first
string is taken as the name of the bead type. Otherwise, the bead types are
called \texttt{beadN}, where \texttt{N} is their type number in the
\texttt{data} file.

Charge of every bead type is taken as the charge of the last bead of this
type in the \texttt{Atoms} section.

Molecule types are determined according to the order of bead types in
molecules as well as according to bead connectivity. Bead order in every
molecule is considered from the lowest to the highest bead index in the
\texttt{data} file. If \texttt{<molecule index>} in a bead line is 0, this
bead is unbonded (i.e., not part of any molecule). If there's only one bead
in a molecule, this bead is considered unbonded as well. The types of
molecules are called \texttt{molN}, where \texttt{N} goes from 1 to the
total number of molecule types.

Following is an example of the \texttt{data} file:

\begin{verbatim}
7 atoms
2 bonds
3 atom types
0 10 xlo xhi
0 10 ylo yhi
0 10 zlo zhi

Masses
1 1.0
2 1.1 # some comment
3 1.2

Atoms
1   0   1   0.0   5.0   5.1   5.2
2   0   1   0.0   5.1   5.2   5.3
3   1   2   0.0   5.2   5.3   5.4
4   1   3   0.0   5.3   5.5   5.6
5   2   3   0.0   5.3   5.5   5.6
6   2   2   0.0   5.3   5.5   5.6
7   3   1   0.0   5.3   5.5   5.6

Bonds
1   1   3   4
2   1   5   6
\end{verbatim}
In this example, there are 3 bead types called \texttt{bead1},
\texttt{some}, and \texttt{bead3}. There are 7 beads in all, 3 are unbonded
(indices 1, 2, and 7; all called \texttt{bead3}), 4 are in two different
molecules. The molecules are considered as different types even though they
have the same numbers bonds and the same connectivity; while the first
molecule has the order of beads \texttt{some}, \texttt{bead3} (according to
the beads' ascending indices), the second one has the order reversed. These
molecule types are called \texttt{mol1} and \texttt{mol2}.

Note that lines in both \texttt{Atoms} and \texttt{Bonds} sections do not
have to be ordered in any way.

Usage (this utility does not use standard options):

\vspace{1em}
\noindent
\texttt{lmp\_vtf <input> <out.vsf> <out.vcf> <options>}

\noindent
\begin{longtable}{p{0.15\textwidth}p{0.794\textwidth}}
  \toprule
  \multicolumn{2}{l}{Mandatory arguments} \\
  \midrule
  \texttt{<input>} & input lammps \texttt{data} file \\
  \texttt{<out.vsf>} & output \texttt{vsf} structure file \\
  \texttt{<out.vcf>} & output \texttt{vcf} coordinate file \\
  \toprule
  \multicolumn{2}{l}{Options}\\
  \midrule
  \texttt{-v} & verbose output\\
  \texttt{-h} & print this help and exit\\
  \bottomrule
\end{longtable}
