\section{SelectedVcf} \label{sec:SelectedVcf}

This utility creates a new \texttt{vcf} coordinate file containing only
beads of specified types. The selected bead
types are printed as comments at the beginning of the output \texttt{vcf}
file.
An \texttt{xyz} coordinate file can also be created from the selected
bead type(s).

Using \texttt{-r} switch, the bead types can also be specified in reverse,
i.e., bead types to be excluded from the output file(s).

Which timesteps are saved can be controlled using \texttt{-st} option for
the first timestep to save and \texttt{-e} option for the last timestep to
save; using \texttt{-sk <int>}, \texttt{<int>} steps are ignored per one
step saved. If the option to save only specified timesteps (\texttt{-n}) is
used, \texttt{-sk} is ignored. A maximum of 100 steps can be explicitly
specified using the \texttt{-n} option.

There is also an option to remove periodic boundary conditions (i.e., to join
molecules). Conversely, the simualation box can be wrapped (i.e., the
periodic boundary conditions applied). If both \texttt{--join} and
\texttt{-w} options are used, the simuation box is first wrapped and then
the molecules are joined.

Also, specified molecules can be excluded which is useful when the same
bead type is shared between more molecule types. However, no utilities can
read a \texttt{vcf} file that does not contain all beads of a given type,
so the \texttt{-x} option is useful only for visualization, e.g., using
vmd.

Usage:

\vspace{1em}
\noindent
\texttt{SelectedVcf <input> <output> <bead type(s)> <options>}

\vspace{1em}
\noindent
\begin{longtable}{p{0.22\textwidth}p{0.724\textwidth}}
  \toprule
  \multicolumn{2}{l}{Mandatory arguments} \\
  \midrule
  \texttt{<input>} & input coordinate file (either \texttt{vcf} or
    \texttt{vtf} format) \\
  \texttt{<output.vcf>} & output \texttt{vcf} coordinate file with indexed
    coordinates \\
  \texttt{<bead type(s)>} & bead type names to save (can be omitted if
    \texttt{-r} is used) \\
  \toprule
  \multicolumn{2}{l}{Non-standard options} \\
  \midrule
  \texttt{-r} & reverse function, i.e., exclude \texttt{<bead type(s)>}
    instead of including them; if no \texttt{<bead type(s)>} are specified,
    all bead types are used (requires \texttt{<input>} with all bead types) \\
  \texttt{--join} & join molecules by removing periodic boundary conditions \\
  \texttt{-w} & wrap simulation box (i.e., apply periodic boundary conditions) \\
  \texttt{-st <int>} & starting timestep for calculation (default: 1) \\
  \texttt{-e <int>} & ending timestep for calculation (default: none) \\
  \texttt{-sk <int>} & number of steps skip per one used (default: 0) \\
  \texttt{-n <int(s)>} & save only specified timesteps \\
  \texttt{-x <name(s)>} & exclude molecules of specified name(s) -- do not
    use if \texttt{<output.vcf>} is further analysed \\
  \texttt{-xyz <name>} & save coordinates to \texttt{xyz} file -- does not
    take into account \texttt{-x} option \\
  \bottomrule
\end{longtable}
