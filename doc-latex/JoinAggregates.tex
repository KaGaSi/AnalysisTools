\section{JoinAggregates} \label{sec:JoinAggregates}

This utility is meant for cases when non-standard option \texttt{-j} is
omitted from \texttt{Aggrega}-\texttt{tes} (or \texttt{Aggregates-NotSameBeads})
command. \texttt{JoinAggregates} uses the provided \texttt{vcf} and
\texttt{agg} files to join aggregates, i.e., to remove their periodic boundary
conditions and save the new coordinates into a \texttt{vcf} file. The
utility reads \texttt{Aggregates} command from the \texttt{agg} file to
determine distance and number of contact pairs for aggregate check (see
Section \ref{sec:Aggregates} for details on \texttt{Aggregates} utility).

Be warned that if \texttt{-sk}, \texttt{-st}, or \texttt{-e} options are
used, \texttt{<output.vcf>} will not be in sync with \texttt{<input.agg>}
and therefore these two files cannot be used in tandem for further analysis
using any utilities calculating aggregate properties.

Usage:

\vspace{1em}
\noindent
\texttt{JoinAggregates <input> <input.agg> <output.vcf> <options>}

\vspace{1em}
\noindent
\begin{longtable}{p{0.22\textwidth}p{0.724\textwidth}}
  \toprule
  \multicolumn{2}{l}{Mandatory arguments} \\
  \midrule
  \texttt{<input>} & input coordinate file (either \texttt{vcf} or
    \texttt{vtf} format) \\
  \texttt{<input.agg>} & input \texttt{agg} file \\
  \texttt{<output.vcf>} & output \texttt{vcf} coordinate file with indexed
    coordinates \\
  \toprule
  \multicolumn{2}{l}{Non-standard options} \\
  \midrule
  \texttt{-sk <int>} & number of steps skip per one used (default: 0) \\
  \texttt{-st <int>} & starting timestep for calculation (default: 1) \\
  \texttt{-e <int>} & ending timestep for calculation (default: none) \\
  \bottomrule
\end{longtable}
