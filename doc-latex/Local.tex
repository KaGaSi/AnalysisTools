\section{Local} \label{sec:Local}

This utility calculates various properties both locally (i.e., average profiles
along the three principal axes) and globally (i.e., per-timestep averages from
the whole system), producing four output files (one file per axis with profiles
along that axis and one file for global averages).

Primarily, the utility calculates kinetic energy and temperature (which requires
an input coordinate file with velocities and bead types with specified mass;
otherwise, both observables will be 0) and density profiles, but using the
\tt{--pot} switch, pair-wise potential can be included to calculate pressure.
For now, only Sutton-Chen and shifted Sutton-Chen potentials are implemented
(see below).

Unlike the DensityBox utility (Section~\ref{sec:DensityBox}), Local takes into
account changing size of an orthogonal simulation box, calculating the profiles
along the three axis with respect to the centre of the box; with option
\tt{--cog} or \tt{--com}, geometric centre of all beads or their centre of mass,
respectively, is used instead. The number of bins (governed by the \tt{<width>}
parameter) is calculated at the first used timestep (which can be specified via
the \tt{-st} option), and this number remains constant; i.e., if the box gets
bigger, beads in the outlying regions are not included in the profiles. The
resulting profiles are written with 0 at the centre of this initial box.

\TODO per-timestep option

An extra file can be provided via the \tt{-fx <file>} option. This file defines
the inter-bead potential to calculate and may contain quantities for reduced
units or coarse-graining level. The section of potential parameters must start
with the line \tt{potential <potential>} and end with the \tt{finish} keyword.
Information for every bead type (named according to the used structure file)
should be present in this section (extra data for bead types not present in the
system are ignored). The following is full structure of the \tt{potential}
section:

\begin{longtable}{ll}
  \tt{potential <potential>} & \# potential type to use\\
  \tt{<global_param_1>}      & \# parameter(s) identical to all beads (see
                               below\\
  \tt{...}                   & \ \ \ \ for their description)\\
  \tt{<global_param_N>}      & \\
  \tt{bead <name_1>}         & \# name of the first bead type\\
  \tt{<param_1> <value(s)>}  & \# first parameter of \tt{<potential>} for
                               bead type\\
  \tt{...}                   & \ \ \ \ \tt{<name_1>} (see below for parameter
                               description)\\
  \tt{<param_N> <value(s)>}  & \# last parameter for \tt{<name_1>}\\
  \tt{bead <name_2>}         & \# name of the second bead type\\
  \tt{<param_1> <value(s)>}  & \\
  \tt{...}                   & \# parameters for the second bead type\\
  \tt{<param_N> <value(s)>}  & \\
  \tt{finish}                & \# end of the potential section \\
\end{longtable}

Besides the potential, self-explanatory lines with keyword-value pairs may
specify reduced units---\tt{energy <value>}, \tt{temperature <value>},
\tt{length <value>}, \tt{mass <value>}, and \tt{pressure <value>}. Provided
potential parameters are divided by these values to be in reduced units while
results are multiplied by these values to provide data in real units; input
coordinates and velocities are assumed to be in reduced units.

Comments (\#-initiated lines) and blank lines in the file are ignored.

\subsection{Sutton-Chen (SC) and shifted SC potentials}

The original density-dependent SC potential based on embedded atom model was
developed to study metals\cite{sutton1990} and is defined as
\begin{equation}\label{eq:SuttonChen}
  U = u^\text{rep} + u^\text{dd} = \epsilon\sum_i\sum_{j>i}
    \left(\frac{a}{r_{ij}}\right)^m-c\epsilon\sum_i\sqrt{\rho_i}
    \mbox{, where }
    \rho_i=\sum_{j\neq i}\left(\frac{a}{r_{ij}}\right)^n,
\end{equation}
where $\epsilon$ is an energy parameter, $a$ is the lattice constant, $c$ is
dimensionless parameter, $m$ and $n$ are integer parameters ($m>n$), $\rho_i$ is
local density for bead $i$, $r_{ij}$ is the distance between beads $i$ and $j$.
Note that variants of the SC potential with non-integer $m$ and $n$ parameters
exist, so the Local utility accepts real values of the parameters.
The pair-wise repulsive part representing Van der Waals repulsion gives rise to
force
\begin{equation}\label{eq:SCfrep}
f^\text{rep}_{ij} = -\frac{\text{d}u^\text{rep}}{\text{d}r_{ij}} =
  \frac{\epsilon_{ij}m_{ij}}{r_{ij}}\left(\frac{a_{ij}}{r_{ij}}\right)^{m_{ij}}
\end{equation}
and the density-dependent part representing metal bonds to force
\begin{equation}\label{eq:SCfdd}
f^\text{dd}_{ij} = -\frac{n_{ij}}{2}
  \left(\frac{c_i\epsilon_{ii}}{\sqrt{\rho_i}}+
        \frac{c_j\epsilon_{jj}}{\sqrt{\rho_j}}\right)
  \left(\frac{a_{ij}}{r_{ij}}\right)^{n_{ij}}.
\end{equation}

To get potential that is smooth at the cut-off distance, $r_\text{c}$, a
shifted variant can be used:
\begin{equation}
U^\text{sf}=u(r)-u(r_\text{c})
  -\left.\frac{\text{d}u}{\text{d}r}\right|_{r_\text{c}}(r-r_\text{c}),
\end{equation}
where $u(r)$ is the standard SC in Equation\eqref{eq:SuttonChen}. The repulsive
part is
\begin{equation}
u^\text{sf-rep} = \epsilon_{ij}\left(\frac{a_{ij}}{r_{ij}}\right)^{m_{ij}}
  -\left(\frac{a_{ij}}{r_\text{c}}\right)^{m_{ij}}
  +m_{ij}\left(\frac{a_{ij}}{r_\text{c}}\right)^{m_{ij}}\frac{r_{ij}-r_\text{c}}{r_\text{c}}
\end{equation}
and the local density, $\rho_{ij}$, for the density-dependent part,
$u^\text{sf-dd}=c\epsilon\sum_i\sqrt{\sum_{j\neq i}\rho_{ij}}$, is
\begin{equation}
\rho_{ij} = \left(\frac{a_{ij}}{r_{ij}}\right)^{n_{ij}}
  -\left(\frac{a_{ij}}{r_\text{c}}\right)^{n_{ij}}
  +n_{ij}\left(\frac{a_{ij}}{r_\text{c}}\right)^{n_{ij}}\frac{r_{ij}-r_\text{c}}{r_\text{c}}
\end{equation}
The repulsive and density-dependent forces can then be written as
\begin{equation}
f^\text{sf-rep}_{ij} =
  \frac{\epsilon_{ij}m_{ij}}{r_{ij}}\left(\frac{a_{ij}}{r_{ij}}\right)^{m_{ij}}
  +m_{ij}\epsilon_{ij}\left(\frac{a_{ij}}{r_\text{c}}\right)^{m_{ij}}\frac{r_{ij}}{r_\text{c}}
\end{equation}
and
\begin{equation}
f^\text{sf-dd}_{ij} = 
  \frac{1}{2}\left(\frac{c_i\epsilon_{ii}}{\sqrt{\rho_i}}+\frac{c_j\epsilon_{jj}}{\sqrt{\rho_j}}\right)
\left[n_{ij}\left(\frac{a_{ij}}{r_{ij}}\right)^{n_{ij}}
-n_{ij}\left(\frac{a_{ij}}{r_\text{c}}\right)^{n_{ij}}\frac{r_{ij}}{r_\text{c}}\right]
\end{equation}

The cross-terms for different bead types are
\begin{equation}
  m_{ij} = \frac{m_{ii} + m_{jj}}{2}\text{, }
  n_{ij} = \frac{n_{ii} + n_{jj}}{2}\text{, and }
  \epsilon_{ij} = \sqrt{\epsilon_{ii}\epsilon_{jj}}.
\end{equation}

Finally, the quasi-coarse-grained dynamics scheme by Dongare\cite{dongare2014}
is used, where the coarse-graining level (CG) represents the number of crystal
unit cells that are reduced into one in each direction; i.e., CG level $i$ means
$i\times i\times i$ unit cells are represented by a single CG unit cell. The
potential parameters are then coarse-grained via the following equations, where
subscript CG represents the coarse-grained parameter and $i$ the CG level,
\begin{equation}
  \begin{array}{rcl}
    n_\text{CG}&\!\!\!=\!\!\!&n\\
    m_\text{CG}&\!\!\!=\!\!\!&m\\
    \epsilon_\text{CG}&\!\!\!=\!\!\!&i^3\epsilon\\
    c_\text{CG}&\!\!\!=\!\!\!&c\\
    T_\text{CG}&\!\!\!=\!\!\!&i^3T\\
    a_\text{CG}&\!\!\!=\!\!\!&ia\\
    r_\text{c,CG}&\!\!\!=\!\!\!&ir_\text{c}\\
  \end{array}
\end{equation}

The following table shows the form of the potential section in the input file
(\tt{-fx <file>}). Note that there are variants of the SC potential that have
real $m$ and $n$ parameters, so the \tt{mn} line accepts real numbers.
\begin{longtable}{ll}
  \toprule
  \multicolumn{2}{c}{Sutton-Chen potential (use \tt{SuttonChen} as
    \tt{<potential>})}\\
  \multicolumn{2}{c}{Shifted Sutton-Chen potential (use \tt{ShiftedSC} as
    \tt{<potential>})}\\
  \midrule
  \multicolumn{2}{c}{2 global parameters:\vspace{-0.7em}}\\
  \multicolumn{2}{c}{\rule{0.3\textwidth}{0.5pt}}\\
  \tt{CG <int>}           & coarse-graining level\\
  \tt{cutoff <float>}     & cut-off distance, $r_\text{c}$\vspace{-0.7em}\\
  \multicolumn{2}{c}{\rule{0.3\textwidth}{0.5pt}}\\
  \multicolumn{2}{c}{4 lines per bead type:\vspace{-0.7em}}\\
  \multicolumn{2}{c}{\rule{0.3\textwidth}{0.5pt}}\\
  \tt{mn <float> <float>} & $m$ and $n$ parameters;
                            lower number is assigned $n$\\
  \tt{c <float>}          & dimensionless $c$ parameter\\
  \tt{eps <float>}        & energy parameter $\epsilon^\star$\\
  \tt{a <float>}          & lattice constant, $a$\\
  \bottomrule
  \multicolumn{2}{l}{$^\star$if in real units, the value must be in
  Kelvins}\\
\end{longtable}

\printbibliography[heading=subbibintoc]
