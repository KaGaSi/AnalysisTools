\section{Angle Molecules} \label{sec:AngleMolecules}

This utility calculates angles (distribution and overall averages) between beads
in each molecule of specified molecule type(s); the input structure file must
contain angles (such as LAMMPS \data file). It can also calculate angles between
any three beads in the molecule type(s) via the \tt{-n} option. See
\tt{Examples/AngleMolecules} for an example.

By default, the angles are calculated for every molecule type, but the \tt{-m}
option can be used to specify only some of them.

The utility groups angles according to bead types; e.g., in a linear 6-bead
molecule (such as in the \tt{Example/AngleMolecules} directory) with bead order
A-A-A-B-A-A with all possible angles defined (i.e., 1-2-3, 2-3-4, 3-4-5, and
4-5-6 where the numbers specify bead indices in the molecule), there are three
angle types: A-A-A, A-A-B, and A-B-A. However, using the \tt{--all} option, all
molecular angles are averaged separately as well; i.e., beside for the three
angles based on bead types, the distributions and averages are calculated for
all five angles in the molecule too.

Using \texttt{-n} option, extra angles are specified by three bead indices taken
from the bead order in the molecule type(s). These indices go from 1 to $N$,
where $N$ is the number of beads in the molecule type (\tt{Info} utility can be
used to check the indices). Multiple angles may be specified, and the results
are written to the specified file. Angles containing an index number that is
higher than the number of beads in a molecule are ignored only for that
molecule.

\vspace{1em}
\noindent
Usage: \tt{AngleMolecules <input> <width> <output> [options]}
\noindent
\begin{longtable}{p{0.24\textwidth}p{0.704\textwidth}}
  \toprule
  \multicolumn{2}{l}{Mandatory arguments} \\
  \midrule
  \texttt{<input>}       & input coordinate file\\
  \texttt{<width>}       & width of each distribution bin\\
  \texttt{<output>}      & output file for distribution\\
  \midrule
  \multicolumn{2}{l}{Options}\\
  \midrule
  \texttt{-m <name(s)>} & specify molecule type(s) to use\\
  \texttt{--joined}     & specify that \texttt{<input>} contains joined
    coordinates\\
  \texttt{--all}        & write all angles for the molecule types\\
  \texttt{-n <file> <ints>} & multiple of three indices for extra angle\\
  \midrule
  \multicolumn{2}{l}{Other options (see the beginning of
                     Chapter~\ref{chap:Utils})}\\
  \midrule
  \multicolumn{2}{p{0.948\textwidth}}{\tt{-st},
                                      \tt{-e},
                                      \tt{-sk},
                                      \tt{-i},
                                      \tt{--verbose},
                                      \tt{--silent},
                                      \tt{--help},
                                      \tt{--version}}\\
  \bottomrule
\end{longtable}

\noindent
Format of output files:
\begin{enumerate}[nosep,leftmargin=20pt]
  \item \texttt{<output>} -- distribution of angles (grouped by bead types and
    possibly for all molecular angles as well if \tt{--all} option is used)
    \begin{itemize}[nosep,leftmargin=5pt]
      \item first line: AnalysisTools version
      \item second line: command used to generate the file
      \item following lines (number of specified molecules plus 1): column
        headers
        \begin{itemize}[nosep,leftmargin=10pt]
          \item first column is the centre of each bin in angles (governed by
            \texttt{<width>}); i.e., if \texttt{<width>} is 5$^{\circ}$,
            then the centre of bin 0 to 5$^{\circ}$ is 2.5, centre of bin 5
            to 10$^{\circ}$ is 7.5 and so on
          \item the rest are for the calculated data: for each specified
            molecule there are angles grouped by bead types followed by all
            angles in that molecule (if \tt{--all} option is used)
        \end{itemize}
      \item next lines: the calculated data
      \item last several lines (number of specified molecules plus 2): minimum,
        maximum, and arithmetic mean for each calculated angle (last line) and
        headers (preceding lines)
    \end{itemize}
  \item \texttt{-n <file>} -- angles specified by supplied molecular indices
  \begin{itemize}[nosep,leftmargin=5pt]
    \item the file structure is the same as for the default output file
  \end{itemize}
\end{enumerate}
