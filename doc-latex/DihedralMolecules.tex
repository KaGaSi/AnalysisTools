\section{DihedralMolecules} \label{sec:DihedralMolecules}

This utility calculates angles between specified planes in each molecule of
specified molecule type(s). The planes in a molecule are arbitrary, so they
can represent true dihedral angles or improper dihedrals.

The angle is specified by six bead indices (according to the order of beads
in the molecule in \texttt{vsf} -- similarly to the \texttt{-n} option in
\texttt{AngleMolecules}, Section~\ref{sec:AngleMolecules}). The
first three indices specify one plane and the next three the other. For
example, assuming indices \texttt{1 2 3 4 5 6}, the first plane is
specified by the first three beads in the molecule; second plane by the
next three beads (beads \texttt{4 5 6}). The default indices
(i.e., if \texttt{-n} option is not used) are \texttt{1 2 3 2 3 4}. More
than one angle can be specified (i.e., a multiple of six numbers have to be
supplied to the \texttt{-n} option.).

The utility calculates distribution of angles for each specified trio of
bead indices for each molecule type and prints overall averages at the end
of \texttt{<output>} file. If \texttt{-a} option is used, it can also write
all the angles for all individual molecules in each timestep (i.e., time
evolution of the angle for each individual molecule).

Usage:

\vspace{1em}
\noindent
\texttt{DihedralMolecules <input> <mol name(s)> <options>}

\noindent
\begin{longtable}{p{0.235\textwidth}p{0.709\textwidth}}
  \toprule
  \multicolumn{2}{l}{Mandatory arguments} \\
  \midrule
  \texttt{<input>} & input coordinate file (either \texttt{vcf} or
    \texttt{vtf} format) \\
  \texttt{<output>} & output file for distribution \\
  \texttt{<mol name(s)>} & molecule name(s) to calculcate angles for \\
  \toprule
  \multicolumn{2}{l}{Non-standard options} \\
  \midrule
  \texttt{--joined} & specify that \texttt{<input>} contains joined
    coordinates (i.e., periodic boundary conditions for molecules do not
    have to be removed) \\
  \texttt{-a <file>} & write all angles for all molecules in all timesteps
    to \texttt{<file>} \\
  \texttt{-n  <ints>} & multiple of six indices for angle calculation
    (default: 1 2 3 2 3 4) \\
  \texttt{-st <int>} & starting timestep for calculation (default: 1) \\
  \bottomrule
\end{longtable}

\noindent
Format of output files:
\begin{enumerate}[nosep,leftmargin=20pt]
  \item \texttt{<output>} -- distribution of angles
    \begin{itemize}[nosep,leftmargin=5pt]
      \item first line: command used to generate the file
      \item second line: calculated angles (the dash-separated numbers
        correspond to indices inside every molecule and are the same as the
        arguments to the \texttt{-n} option) with the numbers in brackets
        corresponding to $n$th column of data for each molecule type
      \item third line: numbering of columns (i.e., column headers)
        \begin{itemize}[nosep,leftmargin=10pt]
          \item first is the centre of each bin in angles (governed by
            \texttt{<width>}); i.e., if \texttt{<width>} is 5$^{\circ}$,
            then the centre of bin 0 to 5$^{\circ}$ is 2.5, centre of bin 5
            to 10$^{\circ}$ is 7.5 and so on
          \item the rest are for the calculated data: the range for each
            molecule type specifies which column numbers correspond to the
            calculated angles for that particular molecule type and the
            order of angles is given by the second line
        \end{itemize}
      \item last two lines: arithmetic means for each calculated angle
        (last line) and headers (second to last line) that again give range
        of columns in the last line for each molecule type
    \end{itemize}
  \item \texttt{-a <file>} -- all angles for all molecules in all timesteps
  \begin{itemize}[nosep,leftmargin=5pt]
    \item first and second lines are the same as for \texttt{<output>}
    \item third lines: column headers
      \begin{itemize}[nosep,leftmargin=10pt]
        \item first is simulation timestep
        \item the rest are the calculated data: the range for each molecule
          type corresponds to the number of molecules of the given type
          times the number of calculated angles; for each molecule the
          angles are ordered according to the second line
      \end{itemize}
  \end{itemize}
\end{enumerate}
