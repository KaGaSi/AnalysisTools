\section{DensityMolecules} \label{sec:DensityMolecules}

This utility works similarly to \texttt{DensityAggregates}, only instead
for whole aggregates, RDPs are calculated for individual molecules.
Similarly to \texttt{DensityAggregates}, the output file(s) also contain
statistical errors and radial number profiles.

By default, the utility calculates RDPs from the molecule's centre of mass,
but specified bead in the molecule can be used instead (\texttt{-c}
option). The bead is specified by its index inside the molecule as defined
in the \texttt{vsf} file. For example, assume that molecule's beads in
\texttt{vsf} are ordered \texttt{A}, \texttt{B}, \texttt{C}. Then, bead
\texttt{A} is 1, bead \texttt{B} is 2, and \texttt{C} is 3. The index for
the bead RDP should be calculated from is an argument of the \texttt{-c}
option. The number of arguments of \texttt{-c} must be equal to the number
of specified molecule type(s). The order for the arguments is taken from
the \texttt{vsf} file, not from the \texttt{DensityMolecules} command. For
example, assume that RDPs for molecules \texttt{Mol\_A} (with centre at the
third bead) and \texttt{Mol\_B} (with centre at the centre of mass) should
be calculated. The form of the \texttt{-c} option depends on the order of
when the first molecules \texttt{Mol\_A} and \texttt{Mol\_B} appear in the
\texttt{vsf} file:
\begin{enumerate}[nosep]
  \item if \texttt{Mol\_A} is above \texttt{Mol\_B} in \texttt{vsf}, then
    it must be \texttt{-c 3 x} (where \texttt{x} means the centre of mass);
  \item if \texttt{Mol\_B} is above \texttt{Mol\_A} in \texttt{vsf}, then
    it must be \texttt{-c x 3}.
\end{enumerate}

Usage:

\vspace{1em}
\noindent
\texttt{DensityMolecules <input> <width> <output> <mol name(s)> <options>}

\noindent
\begin{longtable}{p{0.20\textwidth}p{0.744\textwidth}}
  \toprule
  \multicolumn{2}{l}{Mandatory arguments} \\
  \midrule
  \texttt{<input>} & input coordinate file (either \texttt{vcf} or
    \texttt{vtf} format) \\
  \texttt{<width>} & width of each bin of the distribution \\
  \texttt{<output>} & output file(s) (one per molecule type) with
    automatic \texttt{<mol\_name>.rho} ending \\
  \texttt{<mol name(s)>} & molecule name(s) to calculcate density for \\
  \toprule
  \multicolumn{2}{l}{Non-standard options} \\
  \midrule
  \texttt{--joined} & specify that \texttt{<input>} contains joined
    coordinates (i.e., periodic boundary conditions for molecules do not
    have to be removed) \\
  \texttt{-n <int>} & number of bins to average to get smoother density
    (default: 1) \\
  \texttt{-st <int>} & starting timestep for calculation (default: 1) \\
  \texttt{-c <int>/x} & use specified bead (\texttt{<int>}) or centre of
    molecule's mass (\texttt{x}) as centre (i.e., as $r=0$) \\
  \bottomrule
\end{longtable}

\noindent
Format of output files:
\begin{enumerate}[nosep,leftmargin=20pt]
  \item \texttt{<output>} -- bead densities for one molecule
    \begin{itemize}[nosep,leftmargin=5pt]
      \item first line: command used to generate the file
      \item second line: the order of data columns for each bead type --
        \texttt{rdp} is radial density profile, \texttt{rnp} radial number
        profile and \texttt{stderr} are one-$\sigma$ errors for \texttt{rdp}
        and \texttt{rnp}
      \item third line: column headers
        \begin{itemize}[nosep,leftmargin=10pt]
          \item first is the centre of each bin (governed by
            \texttt{<width>}); i.e., if \texttt{<width>} is 0.1,
            then the centre of bin 0 to 0.1 is 0.05, centre of bin 0.1 to
            0.2 is 0.15, etc.
          \item the rest are for the calculated data: each number specifies
            the first column with data for the given bead type (i.e.,
            \texttt{rdp} column)
        \end{itemize}
    \end{itemize}
\end{enumerate}
