\section{DensityMolecules} \label{sec:DensityMolecules}

This utility works similarly to \texttt{DensityAggregates}, only instead
for whole aggregates, RDPs are calculated for individual molecules.
Similarly to \texttt{DensityAggregates}, the output file(s) also contain
statistical errors and radial number profiles.

By default, the utility calculates RDPs from the molecule's centre of mass,
but any bead in the molecule (with an index according to \texttt{vsf} --
similarly to \texttt{-n} option in \texttt{AngleMolecules},
Section~\ref{sec:AngleMolecules}) can be used instead (\texttt{-c} option).

Usage:

\vspace{1em}
\noindent
\texttt{DensityMolecules <input> <width> <output> <mol name(s)> <options>}

\noindent
\begin{longtable}{p{0.22\textwidth}p{0.724\textwidth}}
  \toprule
  \multicolumn{2}{l}{Mandatory arguments} \\
  \midrule
  \texttt{<input>} & input coordinate file (either \texttt{vcf} or
    \texttt{vtf} format) \\
  \texttt{<width>} & width of each bin of the distribution \\
  \texttt{<output>} & output file with automatic \texttt{<mol\_name>.rho}
    ending (one file per molecule type is generated) \\
  \texttt{<mol name(s)>} & molecule name(s) to calculcate density for \\
  \toprule
  \multicolumn{2}{l}{Non-standard options} \\
  \midrule
  \texttt{--joined} & specify that \texttt{<input>} contains joined
    coordinates (i.e., periodic boundary conditions for molecules do not
    have to be removed) \\
  \texttt{-n <int>} & number of bins to average for smoother density
    (default: 1) \\
  \texttt{-st <int>} & starting timestep for calculation (default: 1) \\
  \texttt{-c <name> <int>} & use specified bead in a molecule
    \texttt{<name>} instead of its centre of mass \\
  \bottomrule
\end{longtable}

\noindent
Format of output files:
\begin{enumerate}[nosep,leftmargin=20pt]
  \item \texttt{<output><mol\_name>.rho} -- bead densities for one molecule
    \begin{itemize}[nosep,leftmargin=5pt]
      \item first line: command used to generate the file
      \item second line: the order of data columns for each bead type --
        \texttt{rdp} is radial density profile, \texttt{rnp} is radial number
        profile and \texttt{stderr} are one-$\sigma$ errors for \texttt{rdp}
        and \texttt{rnp}
      \item third line: column headers
        \begin{itemize}[nosep,leftmargin=10pt]
          \item first is the centre of each bin (governed by
            \texttt{<width>}); i.e., if \texttt{<width>} is 0.1,
            then the centre of bin 0 to 0.1 is 0.05, centre of bin 0.1 to
            0.2 is 0.15, etc.
          \item the rest are for the calculated data: each number specifies
            the first column with data for the given bead type (i.e.,
            \texttt{rdp} column)
        \end{itemize}
    \end{itemize}
\end{enumerate}
