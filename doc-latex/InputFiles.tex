\chapter{Format of input/output files}\label{chap:input}

This section describes several file types used by many of the
AnalysisTools utilities. Output files for the utilities themselves are
described in their respective sections in Chapter~\ref{chap:Utils}.

\section{System information} %{{{
% initial
Most of the utilities read information about the system from
\href{https://github.com/olenz/vtfplugin/wiki/VTF-format}{vtf files},
sometimes complemented by a \field file (input file for
\href{https://www.scd.stfc.ac.uk/Pages/DL_MESO.aspx}{DL\_MESO
simulation package}) or a \href{https://lammps.sandia.gov/}{LAMMPS}
\data file.

The system information can either be split into a \vsf structure file
(Section~\ref{ssec:StructureVsf}) and a \vcf coordinate file
(Section~\ref{ssec:CoordinateVcf}), or presented in a single \vtf file
containing both the structure and coordinate sections; note that when
individual \vscf files are mentioned in the manual, they can be
viewed as \vscf sections of the combined \vtf file. The advantage of
using separate structure and coordinate files is that the same coordinate
file can be used with different structure files to get different results.

In general, the utilities consider only bead types that are present in a
\vcf file (i.e., bead types present in a \vsf file but missing from a \vcf
file are ignored).
%The \tt{FIELD} or \data file is generally used only
%when bead charge and/or mass is missing from the \vsf file.

All utilities assume cuboid simulation box with dimensions from 0 to $N$,
where $N$ is the side length of the box (which can be different in all
three dimensions).

Chapter~\ref{chap:ReadData} briefly describes how the data are read.

\subsection{\vsf structure file} \label{ssec:StructureVsf} %{{{

The structure file contains all information about all beads and bonds
except for their Cartesian coordinates. The first part of a \vsf
file contains bead definitions. Each line contains the description of a
single bead and follows these rules:

\begin{itemize}[topsep=0pt,itemsep=0pt]
  \item the line starts with \tt{atom} (or just \tt{a})
  \item the second string is a bead index number that starts from 0 and
    increases with every subsequent line (the last bead definition line
    therefore shows the total number of beads in the simulation)
  \item the line contains bead name as \tt{name <char(8)>}
  \item the first bead definition line may contain the \tt{default} keyword instead
    of the index number; every bead that is not explicitly written in the
    bead definition lines is of the default type
  \item if the bead is in a molecule, the line contains molecule name
    (\tt{resname <char(8)>}) and molecule id (\tt{resid <int>})
    that starts from 1
  \item \tt{mass} and \tt{charge} keywords are read if present;
    otherwise, bead mass is 1 and bead charge is 0 unless a \tt{FIELD}
    or \data file is provided
  \item other keywords are ignored
\end{itemize}

The types of beads are identified solely by their different name, i.e.,
should two beads share a name but differ in other characteristics, they
will be considered as belonging to the same type. The mass and charge of
each bead type is equal to the value for the corresponding keyword for the
bead with the highest index.

The second part of a \vsf file contains bond definitions.  Each
bond definition line follows these rules:

\begin{itemize}[topsep=0pt,itemsep=0pt]
  \item the line starts with \tt{bond} (or just \tt{b})
  \item bond between two beads is specified by their indices separated by a
    colon
\end{itemize}

Blank lines and comments (lines beginning with \tt{\#}) are allowed in
both parts of the \vsf file.

See examples of the \vsf file in the \tt{Examples} directory.

\subsection{\vcf coordinate file} \label{ssec:CoordinateVcf}

The coordinate file contains Cartesian coordinates of the beads and the
size of the cuboid simulation box.
Coordinates are read from a \vcf file
containing either ordered timesteps or
indexed timesteps.

An ordered \vcf file must contain all beads defined in the \vsf file, while
an indexed \vcf file can contain only a subset of defined beads. Both
indexed and ordered \vcf files contain a line before every timestep
specifying the file type:
\begin{verbatim}timestep ordered/indexed\end{verbatim}
The keyword \tt{timestep} can be omitted.  In
both ordered and indexed \vcf files, the size of the simulation box is
given by a line: appearing before the first timestep:
\begin{verbatim}pbc <float> <float> <float>\end{verbatim}

The \vcf file may contain comment lines (beginning with
\tt{\#}) and blank lines between timesteps, but the coordinate block
must be continuous.

The coordinate blocks in an ordered \vcf file contain only Cartesian
coordinates -- every line has the following format:
\begin{verbatim}<x> <y> <z>\end{verbatim}
The beads are written in ascending order of their indices as defined in the
\vsf file. All beads defined in the \vsf file must be present in the
ordered \vcf file.

The coordinate blocks in an indexed \vcf file contain not only Cartesian
coordinates but also bead indices -- every line has the following format:
\begin{verbatim}<index> <x> <y> <z>\end{verbatim} An indexed timestep does
not have to contain all beads defined in the \vsf structure file; however,
\tt{AnalysisTools} utilities work with whole sets of beads, that is, when
one bead of a specific type is missing, all beads of that type (or with the
same name) must be omitted as well. Moreover, the same number of beads must
be present in each indexed coordinate block. However, the beads do not have
to be ordered according to their ascending indices.

The \tt{Examples} directory contains several example \vtf files. %}}}

\subsection{Complementary \field file} %{{{

This file can be used to get mass and/or charge of bead types (only if
missing from the \vsf file) as well as bond parameters and angle and angle
parameters for molecules (these informations are not stored in the \vsf
file).

The format of this file is taken directly from the
\href{https://www.scd.stfc.ac.uk/Pages/DL_MESO.aspx}{DL\_MESO}
software. If the \field file is used only to read mass and/or charge
information about bead types, only the \tt{species} section is required.
This section contains a header line
\begin{verbatim}species <int>\end{verbatim}
where \tt{<int>} is the number of bead types (or species as called by the
DL\_MESO software) in the \field file. Every bead type is then describe by
a single line:
\begin{verbatim}<name> <m> <q> <n>\end{verbatim}
where \tt{<name>} is the bead type name (that must correspond to a bead
name in the \vsf file if its mass/charge is to be read from the \field
file), \tt{m} and \tt{q} are the bead's mass and charge, respectively,
and \tt{<n>} is the number unbonded beads of that type (i.e., beads
not present in a molecule). Not all bead types in the \vsf file must be
present in the \field file. Blank lines are not allowed in the \tt{species}
section.

Should bond and angle information be read as well, a \tt{molecules} section
must follow the \tt{species} section. This section starts with a header line:
\begin{verbatim}molecule <int>\end{verbatim}
where \tt{<int>} is the number of types of molecules. The header is
followed by \tt{<int>} blocks, each describing a single molecule type and
ending with a line:
\begin{verbatim}finish\end{verbatim}
Every molecular block starts with two lines:
\begin{verbatim}<name>
nummols <n>\end{verbatim}
where \tt{<name>} is the molecule's name (that must correspond to a
\tt{resname} in the \vsf file if the bond/angle information is to be read)
and \tt{<n>} is the number of molecules of this type (that does not have to
correspond to the number of \tt{<name>} molecules in the \vsf file).

The rest of the molecule's description is divided into several sub-blocks:
(i) bead order and coordinates, (ii) bond connectivity, and (iii) angles
which is optional. Bead order as well as connectivity must correspond to
that of the \tt{<name>} molecule in the \vsf file.

Block (i) starts with a line:
\begin{verbatim}beads <beads>\end{verbatim}
where \tt{<beads>} is the number of beads in the molecule. Following are
\tt{<beads>} lines for each bead:
\begin{verbatim}<name> <x> <y> <z>\end{verbatim}
where \tt{<name>} is a bead name that must be present in the \tt{<species>}
section and \tt{<x>}, \tt{<y>}, and \tt{<z>} are Cartesian coordinates.

Block (ii) starts with a line:
\begin{verbatim}bonds <n>\end{verbatim}
where \tt{<n>} is the number of bonds in the molecule. Following are
\tt{<n>} lines for each bond:
\begin{verbatim}harm <l> <m> <k> <r>\end{verbatim}
where \tt{<l>} and \tt{<m>} are bead indices of connected beads that run
from 1 to \tt{<beads>} and \tt{<k>} and \tt{<r>} are the strength and
equilibrium distance, respectively, of a harmonic oscillator (AnalysisTools
assumes bonds use harmonic potential).

Block (iii) starts with a line:
\begin{verbatim}angles <n>\end{verbatim}
where \tt{<n>} is the number of angles in the molecule. Following are
\tt{<n>} lines for each angle:
\begin{verbatim}harm <l> <m> <n> <k> <alpha>\end{verbatim}
where \tt{<l>}, \tt{<m>}, and \tt{<n>} are bead indices defining the angle
that run from 1 to \tt{<beads>} and \tt{<k>} and \tt{<alpha>} are the
strength and equilibrium angle, respectively, of a harmonic oscillator
(AnalysisTools assumes angles use harmonic potential). This block is optional.

Anything between the last angle line and the \tt{finish} keyword is ignored.

Several examples of a \field file can be found in the \tt{Examples}
directory, e.g., in \tt{Examples/AddToSystem}. %}}}

\subsection{Complementary \data file} %{{{
 %}}}
 %}}}

\section{Aggregate file (\tt{agg})} \label{ssec:AggFile} %{{{

The aggregate file with \tt{agg} extension is generated using
\tt{Aggregates} (or \tt{Aggregates -NotSameBeads}) utility. The
file contains information about the number of aggregates in each timestep
and which molecules and monomeric (i.e., unbonded) beads belong to which
aggregate. It serves as an additional input file for utilities that
calculate aggregate properties; \tt{agg} file is, therefore, linked to
the \vcf file that was used to generate it.

The \tt{agg} file is a simple text file. The first line contains the
command used to generate it -- parts of this command may be necessary for
subsequent analysis of aggregates. The second line is blank, and from the
third line, the data for individual timesteps are shown. It follows these
rules:

\begin{itemize}[topsep=0pt,itemsep=0pt]
  \item each timestep starts with \tt{Step: <int>} (only \tt{Step}
    keyword is read by the utilities)
  \item the second line is the number of aggregates in the given
    timestep and is followed by a blank line
  \item there are two lines for each aggregate:
  \begin{enumerate}[topsep=0pt,itemsep=0pt]
    \item number of molecules in the aggregate followed by their indices
      taken from the \vsf file (\tt{resid} indices)
    \item number of monomeric beads in the aggregate followed by their
      indices taken from the \vsf file (\tt{atom} indices)
  \end{enumerate}
  \item no blank or comment lines are allowed inside the aggregate block
  \item not all molecules present in the \vcf file used to generate
    this file must be present in every timestep
\end{itemize}

Note that the term aggregate also refers to free chains (i.e., fully
dissolved chains).

When the keyword \tt{Last} is present instead of \tt{Step}, it
signalises the end of the \tt{agg} file; no utility will read anything
beyond this keyword.

Besides using this file for further analysis by other utilities, the
indices can be used in \href{http://www.ks.uiuc.edu/Research/vmd/}{vmd} to
visualize, e.g., only a specific aggregate by using \tt{resid 1 2 3} in the
\tt{Selected Atoms} box in vmd.

An example of an \tt{agg} file can be found in the \tt{Examples/DistrAgg}
directory. %}}}
