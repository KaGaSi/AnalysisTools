\chapter{Format of input/output files}\label{chap:input}

% initial %{{{
All utilities read information about the system from
\href{https://github.com/olenz/vtfplugin/wiki/VTF-format}{vsf/vcf files}
(formatted as described below) and \texttt{FIELD} file (input file for
\href{http://www.scd.stfc.ac.uk//research/app/ccg/software/DL_MESO/40694.aspx}{DL\_MESO
simulation package}).
All system information is read from the \texttt{vsf} structure file
(Section~\ref{sec:StructureVsf}) and from the \texttt{vcf} coordinate file
(Section~\ref{sec:CoordinateVcf}). One \texttt{vtf} file containing a
structure and coordinate sections can also be used.
The \texttt{FIELD} file is only used when bead charge
and/or mass is missing from the \texttt{vsf} structure file.
The utilities consider only bead types that are present in the \texttt{vcf}
coordinate file (i.e., bead types present in the \texttt{vsf} file but not
in the \texttt{vcf} file are not seen by the utilities). A description of
how the system data are read is shown in Chapter~\ref{chap:ReadData}.

Both \texttt{vsf} and \texttt{vcf} files can be generated using a
\texttt{traject} utility provided by the DL\_MESO simulation package
(\texttt{traject-v2\_5} and \texttt{traject-v2\_6} provided here are
modified versions from earlier DL\_MESO simulation package versions).  Both
structure and coordinates can be in one file (with \texttt{vtf} extension)
-- this file can be used instead of separate \texttt{vsf} and
\texttt{vcf} files.

All utilities assume cuboid simulation box with dimensions from 0 to $N$,
where $N$ is the side length of the box (which can be different in all
three dimensions).
%}}}

\section{\texttt{vsf} structure file} \label{sec:StructureVsf} %{{{

The structure file contains all information about all beads and bonds
except for their Cartesian coordinates. The utilities are written for a
\texttt{vsf} file created by a \texttt{traject} utility (for DL\_MESO
versions from 2.5 to 2.7), but other \texttt{vsf} files should work as long
as they adhere to the following format.

A \texttt{vsf} file is divided into two parts. The first part contains bead
definitions. Each line contains the description of a single bead and
follows these rules:

\begin{itemize}[topsep=0pt,itemsep=0pt]
  \item the line starts with \texttt{atom} (or just \texttt{a})
  \item the second string is a bead index number that starts from 0 and
    increases with every subsequent line (the last bead definition line
    therefore shows the total number of beads in the simulation)
  \item the line contains bead name as \texttt{name <char(8)>}
  \item the first bead definition line may contain \texttt{default} instead
    of the index number; every bead that is not explicitly written in the
    bead definition lines is of the default type
  \item if the bead is in a molecule, the line contains molecule name
    (\texttt{resname <char(8)>}) and molecule id (\texttt{resid <int>})
    that starts from 1
  \item \texttt{mass} and \texttt{charge} keywords are read if present
    (otherwise the mass and charge of beads is read from \texttt{FIELD})
  \item other keywords are ignored
\end{itemize}

The following is an example of bead definition lines containing all
required data:

\begin{verbatim}
   atom default name Bead_A
   atom 0 name Bead_B
   atom 3 name Bead_C resname Mol_A 1
   atom 4 name Bead_C resname Mol_A 1
   atom 6 name Bead_C resname Mol_A 2
   atom 7 name Bead_C resname Mol_A 2
   atom 8 name Bead_D resname Mol_B 3
   atom 9 name Bead_A resname Mol_B 3
\end{verbatim}

In this example, there are four bead types (named \texttt{Bead\_A},
\texttt{Bead\_B}, and \texttt{Bead\_C}, and \texttt{Bead\_D}) and 10 beads
in all (with indices from 0 to 9). Beads with indices 1, 2, 5, and 9 are of
the default type (\texttt{Bead\_A}); notice that if a bead of a default
type is in a molecule, it must have its own \texttt{atom} line. There are
two molecule types named \texttt{Mol\_A} and \texttt{Mol\_B} with molecule
indices 1 to 3. All molecules with the same name must have the same
structure, i.e., the same number of beads and the same bond connectivity.

The second part of a \texttt{vsf} file contains bond definitions.  Each
bond definition line follows these rules:

\begin{itemize}[topsep=0pt,itemsep=0pt]
  \item the line starts with \texttt{bond} (or just \texttt{b})
  \item bond between two beads is specified by their indices separated by a
    colon
\end{itemize}

The following is an example of bead definition lines that complement the
above-shown bead definition lines:

\begin{verbatim}
   bond 3: 4
   bond 6: 7
   bond 8: 9
\end{verbatim}

In this example, there three bonds between beads with indices 3 and 4, 6
and 7, 8 and 9.

Blank lines and comments (lines beginning with \texttt{\#}) are allowed in
both parts of the \texttt{vsf} file. Instead of a separate \texttt{vsf}
file, a \texttt{vtf} file can be used.

\section{\texttt{vcf} coordinate file} \label{sec:CoordinateVcf}

The coordinate file contains Cartesian coordinates of the beads and the
size of the cuboid simulation box.
Coordinates are read from a \texttt{vcf} file
containing either ordered timesteps (Section~\ref{ssec:OrderedVcf}) or
indexed timesteps (Section~\ref{ssec:IndexedVcf}).

An ordered \texttt{vcf} file must contain all beads defined in the
\texttt{vsf} file, while an indexed \texttt{vcf} file can contain only a
subset of defined beads. Both indexed and ordered \texttt{vcf} files
contain a line before every timestep specifying the file type --
\texttt{timestep ordered} or \texttt{timestep indexed} (the keyword
\texttt{timestep} can be omitted).  In both ordered and indexed
\texttt{vcf} files, the size of the simulation box is given by a line
\texttt{pbc <float> <float> <float>} which is located before the first
coordinate block. Only \texttt{timestep} and \texttt{pbc} lines are read
before the first coordinates (everything else is ignored), so \texttt{vtf}
file can be used instead of a \texttt{vcf} file.

The \texttt{vcf} file may contain comment lines (beginning with
\texttt{\#}) and blank lines between timesteps, but the coordinate block
must be continuous.

\subsection{Ordered coordinate file} \label{ssec:OrderedVcf}

Coordinate lines in ordered \texttt{vcf} file contain only the Cartesian
coordinates of the beads in the form \texttt{<float> <float> <float>}. The
beads are written in ascending order of their indices as defined in the
\texttt{vsf} file. The following is an example of an ordered \texttt{vcf}
file:

\begin{verbatim}
   # any number of comments or blank lines

   timestep ordered
   pbc 10 10 10
   0.0 0.0 0.0
   0.5 0.5 0.5
   ...

   # comments between timesteps
   timestep ordered
   # another comment
   1.0 1.0 1.0
   1.5 1.5 1.5
   ...
\end{verbatim}

In this example, the simulation box is cubic with side length of 10.
Beginnings of two timesteps are represented by coordinates of the first two
beads with indices 0 and 1 (as defined in the \texttt{vsf} file).

\subsection{Indexed coordinate file} \label{ssec:IndexedVcf}

Indexed coordinate files contains not only Cartesian coordinates, but also
bead indices (preceding the coordinates). Therefore an
indexed timestep does not have to contain all beads in the \texttt{vsf}
structure file. Moreover, the beads do not have to be ordered according to
their ascending indices.  The following is an example of an ordered
\texttt{vcf} file:

\begin{verbatim}
   # any number of comments or blank lines

   timestep indexed
   pbc 10 10 10
    2 0.5 0.5 0.5
   21 0.0 0.0 0.0
   ...

   # comments between timesteps
   timestep indexed
   # another comment
   21 1.0 1.0 1.0
    2 1.5 1.5 1.5
   ...
\end{verbatim}

This example is similar to that for the ordored \texttt{vcf} file, but two
beads have indices 2 and 21 instead of 0 and 1. %}}}

\section{Aggregate file (\texttt{agg})} \label{sec:AggFile} %{{{

The aggregate file with \texttt{agg} extension is generated using
\texttt{Aggregates} utility. The file
contains information about the number of aggregates in each timestep and
which molecules and monomeric beads belong to which aggregate. It serves as
an additional input file for utilities that calculate properties of whole
aggregates -- \texttt{agg} file is therefore linked to the \texttt{vcf} file
that was used to generate it.

The \texttt{agg} file is a simple text file. The first line contains the
command used to generate it -- parts of this command can be necessary for
subsequent analysis of aggregates. The second line is blank, and from the
third line, the data for individual timesteps are shown. It follows these
rules:

\begin{itemize}[topsep=0pt,itemsep=0pt]
  \item each timestep starts with \texttt{Step: <int>} (only \texttt{Step}
    keyword is read by the utilities)
  \item the second line contains the number of aggregates in the given
    timestep and is followed by a blank line
  \item there are two lines for each aggregate:
  \begin{enumerate}[topsep=0pt,itemsep=0pt]
    \item number of molecules in the aggregate followed by their indices
      taken from the \texttt{vsf} file
    \item number of monomeric beads in the aggregate followed by their
      indices taken from the \texttt{vsf} file
  \end{enumerate}
  \item no blank or comment lines are allowed inside the aggregate block
  \item all molecules present in the \texttt{vcf} file used to generate
    this file must be present in every timestep; here, aggregate can also
    refer to dissolved molecules (i.e., single chains).
\end{itemize}

When the keyword \texttt{Last} is present instead of \texttt{Step}, it
signalises the end of the \texttt{agg} file; no utility will read anything
beyond this keyword.

Following is an example of an \texttt{agg} file:

\begin{verbatim}
   Aggregates in.vcf 1 1 out.agg A

   Step: 1
   2

   2 : 1 3
     3 : 10 100 1000
   1 : 2
     1 : 20

   Step: 2
   1

   3 : 1 2 3
     4 : 10 20 100 2000

   Last Step: 2
\end{verbatim}

In this example, command \texttt{Aggregates in.vcf 1 1 out.agg A} was used
to generate the file (see Section \ref{sec:Aggregates} for details about
this utility). There are two timesteps here -- the first contains two
aggregates (although one of them is a free, dissolved molecule) and the
second a single aggregate.  As an example, the aggregate in the second step
contains three molecules with indices 1, 2, and 3 (taken from the
\texttt{vsf} file) and four monomeric beads (i.e., solvent or counterions)
with indices 10, 20, 100, and 2000 (again, taken from the \texttt{vsf}
file).

Besides using this file for further analysis using other utilities, the
indices can be used in vmd to visualize, e.g., only a specific aggregate by
using \texttt{resid 1 2 3} in the \texttt{Selected Atoms} box in vmd.
 %}}}
