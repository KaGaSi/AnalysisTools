\chapter{Format of input/output files}\label{chap:input}

This section describes several file types used by many of the
AnalysisTools utilities. Output files for the utilities themselves are
described in their respective sections in Chapter~\ref{chap:Utils}.

\section{System information} %{{{
Most of the utilities read system information from
\href{https://github.com/olenz/vtfplugin/wiki/VTF-format}{vtf files},
sometimes complemented by a \field file (input file for
\href{https://www.scd.stfc.ac.uk/Pages/DL_MESO.aspx}{DL\_MESO simulation
package}) or a \href{https://lammps.sandia.gov/}{LAMMPS} \data file.

The system information can either be split into a \vsf structure file
(Section~\ref{ssec:StructureVsf}) and a \vcf coordinate file
(Section~\ref{ssec:CoordinateVcf}), or presented in a single \vtf file
containing both the structure and coordinate sections; note that when
individual \vscf files are mentioned in the manual, they can usually be
viewed as \vscf sections of the combined \vtf file. The advantage of
using separate structure and coordinate files is that the same coordinate
file can be used with different structure files to get different results.

In general, the utilities consider only bead types that are present in a
\vcf file (i.e., bead types present in a \vsf file but missing from a \vcf
file are ignored).
%The \tt{FIELD} or \data file is generally used only
%when bead charge and/or mass is missing from the \vsf file.

All utilities assume cuboid simulation box with dimensions from 0 to $N$,
where $N$ is the side length of the box (which can be different in all
three dimensions).

Chapter~\ref{chap:ReadData} briefly describes how the data are read.

\subsection{\vsf structure file} \label{ssec:StructureVsf} %{{{

The structure file contains all information about all beads and bonds
except for their Cartesian coordinates in two types of entries. The first
type of entry, an atom line, contains a bead definition. Each atom line is
the description of a single bead following these rules:

\begin{itemize}[topsep=0pt,itemsep=0pt]
  \item the line starts with \tt{a[tom]} (i.e., \tt{atom} or just \tt{a})
  \item the second string is a either a bead index (starting at 0) or the
    \tt{default} keyword; every bead that is not explicitly written in the
    bead definition lines is of the \tt{default} type
  \item the line contains bead name as \tt{n[ame] <char(16)>}
  \item if the bead is in a molecule, the line contains both molecule name
    (\tt{res[name] <char(8)>}) and molecule id (\tt{resid <int>}) starting
    at 1
  \item \tt{m[ass]}, \tt{charge|q} (i.e., \tt{charge} or \tt{q}), and
    \tt{r[adius]} keywords are read if present
  \item other keywords are ignored
\end{itemize}

The types of beads are generally identified solely by their name, i.e.,
should two beads share a name but differ in other characteristics (i.e.,
mass, charge, or radius), they will be considered as belonging to the same
type. The mass, charge, and radius of each bead type is equal to that for
the topmost bead with that name in the \vsf file that contains the
corresponding keyword (if the keyword is not present in any line of the
given bead type, that characteristic is undefined). However in some cases,
beads that share a name but differ in their characteristics, are treated as
different bead types (with \tt{\_<int>} attached to its name); notably,
\tt{TransformVsf} does this to produce a new \vsf file with these names.
For example, the lines
\begin{verbatim}
atom default name A mass 1 q 0
a 0 name B mass 2 r 1
a 2 name B m 1 charge -1
atom 4 name B
\end{verbatim}
would ordinarilly lead to two \tt{A} type beads (both with mass 1, charge
0, and undefined radius) and three \tt{B} type beads (all with mass 2,
charge -1 and radius 1). However, \tt{TransformVsf} would split the \tt{B}
type beads into three distinct types one \tt{B\_1} bead (with mass 2,
undefined charge, and radius 1), one \tt{B\_2} bead (with mass 1, charge
-1, and undefined radius), and one \tt{B\_3} bead (with all three
characteristics undefined).  Note that if the \tt{a 0...} line weren't
there, the \tt{B} beads would be the same even for \tt{TransformVsf}; i.e.,
if there's only one value for any characteristic, all beads sharing the
same name will have that value even if it is not specified on every atom
line.

The second type of entry is a bond line defining connectivity between
beads. Each bond line starts with \tt{b[ond]} followed by two bead indexes
(corresponding to an atom line) separated by a colon. Note that the colon
must be right after the first index, i.e., no white space between the
number and the colon is allowed. The double colon or comma separated
bond entries are not allowed in \tt{AnalysisTools}.

In general, \tt{AnalysistTools} recognizes types of molecules only by their
names (taking the connectivity and bead order of the first molecule of
every type); if two molecules share the same \tt{res[name]}, but differ in
connectivity or number of beads, the utilities will exit with an error.
Again, the \tt{TransformVsf} utility goes to more detail, distinguishing
molecule types by all its characteristics. For example, the lines
\begin{verbatim}
a 0 A m 1 res Mol resid 1
a 1 B m 1 res Mol resid 1
a 2 B res Mol resid 1
a 3 A res Mol resid 2
a 4 B res Mol resid 2
a 5 B m 2 res Mol resid 2
b 0:1
b 1:2
b 3:4
b 4:5
b 3:5
\end{verbatim}
would ordinarilly lead to two linear \tt{Mol} type molecules, each
containing one \tt{A} and two \tt{B} beads and connected \tt{A-B-B}.
However, \tt{TransformVsf} would split the molecules to one linear
\tt{Mol\_1} type molecule (with one \tt{A} and two \tt{B\_1} beads,
connected \tt{A-B\_1-B\_1}) and one ring \tt{Mol\_2} type molecule (with
one \tt{A}, one \tt{B\_1}, and one \tt{B\_2} beads, connected in a
triangle); \tt{B\_1} bead has mass 1 and \tt{B\_2} bead mass 2.

The \vsf file can contain any number of blank lines and comments (lines
beginning with \tt{\#}).

See complete examples of the \vsf file in the \tt{Examples} directory.

\subsection{\vcf coordinate file} \label{ssec:CoordinateVcf}

The coordinate file contains Cartesian coordinates of the beads and the
size of the cuboid simulation box. Coordinates are read from a \vcf file
containing either ordered timesteps or indexed timesteps; all timesteps
must be of the same type and contain the same beads.

An ordered \vcf file must contain all beads defined in the \vsf file, while
an indexed \vcf file can contain only a subset of defined beads. Both
indexed and ordered \vcf files contain a line before every timestep
specifying the file type, \tt{timestep o[rdered]/i[ndexed]}; the
\tt{timestep} keyword can be omitted. In both ordered and indexed \vcf
files, the size of the simulation box is given by a line appearing before
the first timestep \tt{pbc <float> <float> <float>}.

The \vcf file may contain comment lines (beginning with
\tt{\#}) and blank lines between timesteps, but the coordinate block
must be continuous.

The coordinate blocks in an ordered \vcf file contain only Cartesian
coordinates -- every line has the \tt{<x> <y> <z>} format The beads are
written in an ascending order of their indices as defined in the \vsf file.
All beads defined in the \vsf file must be present in the ordered \vcf
file.

The coordinate blocks in an indexed \vcf file contain not only Cartesian
coordinates but also bead indices -- every line has the \tt{<index> <x> <y>
<z>} format. An indexed timestep does not have to contain all beads defined
in the \vsf structure file; however, \tt{AnalysisTools} utilities work with
whole sets of beads, that is, when one bead of a specific type is missing,
all beads of that type (or with the same name) must be omitted as well.
The beads do not have to be ordered in any specific way. %}}}

\subsection{DL\_MESO \field file} %{{{

\begin{comment}
This file can be used to get mass and/or charge of bead types (only if
missing from the \vsf file) as well as bond parameters and angle and angle
parameters for molecules (these informations are not stored in the \vsf
file).

The format of this file is taken directly from the
\href{https://www.scd.stfc.ac.uk/Pages/DL_MESO.aspx}{DL\_MESO}
software. If the \field file is used only to read mass and/or charge
information about bead types, only the \tt{species} section is required.
This section contains a header line
\begin{verbatim}species <int>\end{verbatim}
where \tt{<int>} is the number of bead types (or species as called by the
DL\_MESO software) in the \field file. Every bead type is then describe by
a single line:
\begin{verbatim}<name> <m> <q> <n>\end{verbatim}
where \tt{<name>} is the bead type name (that must correspond to a bead
name in the \vsf file if its mass/charge is to be read from the \field
file), \tt{m} and \tt{q} are the bead's mass and charge, respectively,
and \tt{<n>} is the number unbonded beads of that type (i.e., beads
not present in a molecule). Not all bead types in the \vsf file must be
present in the \field file. Blank lines are not allowed in the \tt{species}
section.

Should bond and angle information be read as well, a \tt{molecules} section
must follow the \tt{species} section. This section starts with a header line:
\begin{verbatim}molecule <int>\end{verbatim}
where \tt{<int>} is the number of types of molecules. The header is
followed by \tt{<int>} blocks, each describing a single molecule type and
ending with a line:
\begin{verbatim}finish\end{verbatim}
Every molecular block starts with two lines:
\begin{verbatim}<name>
nummols <n>\end{verbatim}
where \tt{<name>} is the molecule's name (that must correspond to a
\tt{resname} in the \vsf file if the bond/angle information is to be read)
and \tt{<n>} is the number of molecules of this type (that does not have to
correspond to the number of \tt{<name>} molecules in the \vsf file).

The rest of the molecule's description is divided into several sub-blocks:
(i) bead order and coordinates, (ii) bond connectivity, and (iii) angles
which is optional. Bead order as well as connectivity must correspond to
that of the \tt{<name>} molecule in the \vsf file.

Block (i) starts with a line:
\begin{verbatim}beads <beads>\end{verbatim}
where \tt{<beads>} is the number of beads in the molecule. Following are
\tt{<beads>} lines for each bead:
\begin{verbatim}<name> <x> <y> <z>\end{verbatim}
where \tt{<name>} is a bead name that must be present in the \tt{<species>}
section and \tt{<x>}, \tt{<y>}, and \tt{<z>} are Cartesian coordinates.

Block (ii) starts with a line:
\begin{verbatim}bonds <n>\end{verbatim}
where \tt{<n>} is the number of bonds in the molecule. Following are
\tt{<n>} lines for each bond:
\begin{verbatim}harm <l> <m> <k> <r>\end{verbatim}
where \tt{<l>} and \tt{<m>} are bead indices of connected beads that run
from 1 to \tt{<beads>} and \tt{<k>} and \tt{<r>} are the strength and
equilibrium distance, respectively, of a harmonic oscillator (AnalysisTools
assumes bonds use harmonic potential).

Block (iii) starts with a line:
\begin{verbatim}angles <n>\end{verbatim}
where \tt{<n>} is the number of angles in the molecule. Following are
\tt{<n>} lines for each angle:
\begin{verbatim}harm <l> <m> <n> <k> <alpha>\end{verbatim}
where \tt{<l>}, \tt{<m>}, and \tt{<n>} are bead indices defining the angle
that run from 1 to \tt{<beads>} and \tt{<k>} and \tt{<alpha>} are the
strength and equilibrium angle, respectively, of a harmonic oscillator
(AnalysisTools assumes angles use harmonic potential). This block is optional.

Anything between the last angle line and the \tt{finish} keyword is ignored.

Several examples of a \field file can be found in the \tt{Examples}
directory, e.g., in \tt{Examples/AddToSystem}.
\end{comment}
 %}}}

\subsection{LAMMPS \data file} %{{{
 %}}}
  %}}}

\section{Aggregate file (\tt{agg})} \label{ssec:AggFile} %{{{

A \tt{file.agg} is generated using
\tt{Aggregates} (or \tt{Aggregates-NotSameBeads}) utility. The
file contains information about the number of aggregates in each timestep
and which molecules and monomeric (i.e., unbonded) beads belong to which
aggregate. It serves as an additional input file for utilities that
calculate aggregate properties; \tt{agg} file is, therefore, linked to
the \vcf file that was used to generate it.

The \tt{agg} file is a simple text file. The first line contains the
command used to generate it -- parts of this command may be necessary for
subsequent analysis of aggregates. The second line is blank, and from the
third line, the data for individual timesteps are shown. It follows these
rules:

\begin{itemize}[topsep=0pt,itemsep=0pt]
  \item each timestep starts with \tt{Step: <int>} (only \tt{Step}
    keyword is read by the utilities)
  \item the second line is the number of aggregates in the given
    timestep and is followed by a blank line
  \item there are two lines for each aggregate:
  \begin{enumerate}[topsep=0pt,itemsep=0pt]
    \item number of molecules in the aggregate followed by their indices
      taken from the \vsf file (\tt{resid} indices)
    \item number of monomeric beads in the aggregate followed by their
      indices taken from the \vsf file (\tt{atom} indices)
  \end{enumerate}
  \item no blank or comment lines are allowed inside the aggregate block
  \item not all molecules present in the \vcf file used to generate
    this file must be present in every timestep
\end{itemize}

Note that the term aggregate also refers to free chains (i.e., fully
dissolved chains).

When the keyword \tt{Last} is present instead of \tt{Step}, it
signalises the end of the \tt{agg} file; no utility will read anything
beyond this keyword.

Besides using this file for further analysis by other utilities, the
indices can be used in \href{http://www.ks.uiuc.edu/Research/vmd/}{vmd} to
visualize, e.g., only a specific aggregate by using \tt{resid 1 2 3} in the
\tt{Selected Atoms} box in vmd.

An example of an \tt{agg} file can be found in the \tt{Examples/DistrAgg}
directory. %}}}
