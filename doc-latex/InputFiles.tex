\chapter{Format of input/output files}\label{chap:input}

This section describes several file types used by many of the
AnalysisTools utilities. Output files for the utilities themselves are
described in their respective sections in Chapter~\ref{chap:Utils}.

Generally, the utilities work with
\href{https://github.com/olenz/vtfplugin/wiki/VTF-format}{vtf files} which
are convenient text files containing particle coordinates as well as
particle definitions (name, mass, charge, etc.) and molecule connectivity
(i.e., bonds between particles). Additional information (e.g., bond angles
or dihedrals) can be added from other sources such as input files for
\href{https://www.scd.stfc.ac.uk/Pages/DL_MESO.aspx}{DL\_MESO simulation
package} or \href{https://lammps.sandia.gov/}{LAMMPS molecular dynamics
simulator}.

\section{System description} %{{{
Most of the utilities read system description from the \vtf files
which can be complemented by a DL\_MESO \field or a LAMMPS \data files. Not
all valid \vtf files will work with AnalysisTools; see the following
sections for information their proper formatting

The \vtf can either be split into a \vsf structure file
(Section~\ref{ssec:StructureVsf}) and a \vcf coordinate file
(Section~\ref{ssec:CoordinateVcf}), or presented in a single \vtf file
containing both the structure and coordinate sections; note that when
individual \vscf files are mentioned in the manual, they can usually be
viewed as \vscf sections of the combined \vtf file. The advantage of using
separate structure and coordinate files is that the same coordinate file
can sometime be used with different structure files to get different
results.

In general, the utilities consider only bead types that are present in a
\vcf file (i.e., bead types present in a \vsf file but missing from a \vcf
file are ignored).

Some of the utilities work with orthogonal as well as other paralleloid
boxes (i.e., triclinic cells).

\subsection{\vsf structure file} \label{ssec:StructureVsf} %{{{

The structure file contains all information about all beads and bonds
(except for their Cartesian coordinates) in two types of entries. The first
type, an \enquote{atom line}, contains a bead definition, while the second
one, a \enquote{bond line}, defines which beads are connected.

Each atom line describes a single bead (except for the \tt{atom default}
line, see below) and must follow these rules:

\begin{itemize}[topsep=0pt,itemsep=0pt]
  \item the line starts with \tt{a[tom]} (i.e., \tt{atom} or just \tt{a})
  \item the second string is either a bead index (starting at 0) or the
    \tt{default} keyword; every bead that is not explicitly written in the
    bead definition lines is of the \tt{default} type (in case of multiple
    \tt{default} lines, the first one is used while the subsequent ones are
    ignored)
  \item the line contains bead name as \tt{n[ame] <char(16)>}
  \item if the bead is in a molecule, the line contains both molecule name
    (\tt{res[name] <char(8)>}) and molecule id (\tt{resid <int>}) starting
    at 1
  \item \tt{m[ass]}, \tt{charge|q} (i.e., \tt{charge} or \tt{q}), and
    \tt{r[adius]} keywords are read if present
  \item other keywords are ignored
\end{itemize}

Each bond line starts with \tt{b[ond]} followed by two bead indexes
(corresponding to their indexes from the atom lines) separated by a colon.
Note that the colon must be right after the first index, i.e., no white
space between the number and the colon is allowed.

\tt{AnalysisTools} groups the same beads (and the same molecules) into bead
(and molecule) types. Generally, types of beads are defined solely by their
name, i.e., should two beads share a name but differ in other
characteristics, they will be grouped into the same type with mass, charge,
and radius each equal to that for the bead type's topmost atom line with
the corresponding keyword. If the keyword is missing from all atom lines,
that characteristic is undefined. Similarly, different molecule molecules
are generally defined based on their names. If two molecules share a name,
but differ in bead order and/or connectivity, the utility will exit with an
error.

However, some utilities can distinguish bead and molecule types based on
other characteristics as well. For example, the \tt{--detailed} option
forces the \tt{Info} utility (\ref{sec:Info}) to consider mass,
charge, and radius along with the name for defining bead types and bead
order and connectivity along with the name for defining molecule types. As
an example, consider the following \vsf file that contains 11 beads and
two molecules:

...well, I wish it actually worked.

\begin{lstlisting}
atom default name A mass 1 charge 0
atom 0 name B mass 2 radius 1
atom 2 name B mass 1 charge -1
atom 4 name B
atom 5 name C resname Mol resid 1 mass 1
atom 6 name A resname Mol resid 1 mass 1
atom 7 name A resname Mol resid 1
atom 8 name C resname Mol resid 2
atom 9 name D resname Mol resid 2 radius 1.5
atom 10 name D resname Mol resid 2 mass 2
bond 5:6
bond 6:7
bond 8:9
bond 9:10
bond 8:10
\end{lstlisting}

Using \tt{Info <vsf>}, bead and molecule types are identified by their name
only:

\begin{lstlisting}
Counts = {
  .TypesOfBeads     = 4,
  .Bonded           = 6,
  .Unbonded         = 5,
  .Beads            = 11,
  .TypesOfMolecules = 1,
  .Molecules        = 2
}

BeadType[0] = {.Name = A, .Number = 4, .Charge = 0.0,
                                  .Mass = 1.0, .Radius = n/a}
BeadType[1] = {.Name = B, .Number = 3, .Charge = n/a,
                                  .Mass = 2.0, .Radius = 1.0}
BeadType[2] = {.Name = C, .Number = 2, .Charge = n/a,
                                  .Mass = 1.0, .Radius = n/a}
BeadType[3] = {.Name = D, .Number = 2, .Charge = n/a,
                                  .Mass = n/a, .Radius = 1.5}

MoleculeType[ 0] = {
  .Name       = Mol,
  .Number     = 2,
  .nBeads     = 3,
  .Bead       = {C, A, A},
  .nBonds     = 2,
  .Bond       = {1-2, 2-3},
  .nBTypes    = 2
  .BType      = {C, A},
  .Mass       = 3.0,
  .Charge     = n/a
}
\end{lstlisting}

Here, four bead types and a single molecule type are identified. Looking,
for example, at the mass of the \tt{B} beads, its value is 2 according to
the second line in the \vsf file (the first atom line containing the name
\tt{B}) even though the mass of \tt{B} on the next atom line is 1. Looking
at the single molecule type, it contains bead types \tt{C} and \tt{A}
connected \tt{C-A-A}; these were taken from the first molecule (\vsf lines
5-7). However, the second molecule is in the shape of a triangle (see \vsf
lines 13-15).

The \vsf file can contain any number of blank lines and comments (lines
beginning with \tt{\#}).

See complete examples of the \vsf file (as well as of some errors) in the
\tt{Examples} directory.

\subsection{\vcf coordinate file} \label{ssec:CoordinateVcf}

The coordinate file contains Cartesian coordinates of the beads and the
size of the paralleloid simulation box. Coordinates are read from a \vcf
file containing either ordered timesteps or indexed timesteps; all
timesteps must be of the same type and contain the same beads.

An ordered \vcf file must contain all beads defined in the \vsf file, while
an indexed \vcf file can contain only a subset of defined beads. Both
indexed and ordered \vcf files contain a line before every timestep
specifying the file type, \tt{t[imestep] o[rdered]/i[ndexed]}; the
\tt{timestep} keyword can be omitted.

In both ordered and indexed \vcf files, the size of the simulation box is
given by a line appearing before the first timestep:
%
\begin{lstlisting}
pbc <x> <y> <z> # cuboid simulation box
pbc <a> <b> <c> <alpha> <beta> <gamma> # triclinic box
\end{lstlisting}
%
Should this line also prepend other timesteps, the box dimensions are
adjusted accordingly.

The \vcf file may contain comment lines (beginning with
\tt{\#}) and blank lines between timesteps, but the coordinate block
must be continuous.

The coordinate blocks in an ordered \vcf file contain only Cartesian
coordinates -- every line has the \tt{<x> <y> <z>} format The beads are
written in an ascending order of their indices as defined in the \vsf file.
All beads defined in the \vsf file must be present in the ordered \vcf
file.

The coordinate blocks in an indexed \vcf file contain not only Cartesian
coordinates but also bead indices -- every line has the \tt{<index> <x> <y>
<z>} format. An indexed timestep does not have to contain all beads defined
in the \vsf structure file; however, \tt{AnalysisTools} utilities work with
whole sets of beads, that is, when one bead of a specific type is missing,
all beads of that type (or with the same name) must be omitted as well.
The beads do not have to be ordered in any specific way.

Velocities can also be included following the Cartesian coordinates, i.e.,
\begin{lstlisting}
<index> <x> <y> <z> <vel(x)> <vel(y)> <vel(z)>
\end{lstlisting}
 %}}}

\subsection{DL\_MESO \field file} %{{{

\begin{comment}
This file can be used to get mass and/or charge of bead types (only if
missing from the \vsf file) as well as bond parameters and angle and angle
parameters for molecules (these informations are not stored in the \vsf
file).

The format of this file is taken directly from the
\href{https://www.scd.stfc.ac.uk/Pages/DL_MESO.aspx}{DL\_MESO}
software. If the \field file is used only to read mass and/or charge
information about bead types, only the \tt{species} section is required.
This section contains a header line
\begin{verbatim}species <int>\end{verbatim}
where \tt{<int>} is the number of bead types (or species as called by the
DL\_MESO software) in the \field file. Every bead type is then describe by
a single line:
\begin{verbatim}<name> <m> <q> <n>\end{verbatim}
where \tt{<name>} is the bead type name (that must correspond to a bead
name in the \vsf file if its mass/charge is to be read from the \field
file), \tt{m} and \tt{q} are the bead's mass and charge, respectively,
and \tt{<n>} is the number unbonded beads of that type (i.e., beads
not present in a molecule). Not all bead types in the \vsf file must be
present in the \field file. Blank lines are not allowed in the \tt{species}
section.

Should bond and angle information be read as well, a \tt{molecules} section
must follow the \tt{species} section. This section starts with a header line:
\begin{verbatim}molecule <int>\end{verbatim}
where \tt{<int>} is the number of types of molecules. The header is
followed by \tt{<int>} blocks, each describing a single molecule type and
ending with a line:
\begin{verbatim}finish\end{verbatim}
Every molecular block starts with two lines:
\begin{verbatim}<name>
nummols <n>\end{verbatim}
where \tt{<name>} is the molecule's name (that must correspond to a
\tt{resname} in the \vsf file if the bond/angle information is to be read)
and \tt{<n>} is the number of molecules of this type (that does not have to
correspond to the number of \tt{<name>} molecules in the \vsf file).

The rest of the molecule's description is divided into several sub-blocks:
(i) bead order and coordinates, (ii) bond connectivity, and (iii) angles
which is optional. Bead order as well as connectivity must correspond to
that of the \tt{<name>} molecule in the \vsf file.

Block (i) starts with a line:
\begin{verbatim}beads <beads>\end{verbatim}
where \tt{<beads>} is the number of beads in the molecule. Following are
\tt{<beads>} lines for each bead:
\begin{verbatim}<name> <x> <y> <z>\end{verbatim}
where \tt{<name>} is a bead name that must be present in the \tt{<species>}
section and \tt{<x>}, \tt{<y>}, and \tt{<z>} are Cartesian coordinates.

Block (ii) starts with a line:
\begin{verbatim}bonds <n>\end{verbatim}
where \tt{<n>} is the number of bonds in the molecule. Following are
\tt{<n>} lines for each bond:
\begin{verbatim}harm <l> <m> <k> <r>\end{verbatim}
where \tt{<l>} and \tt{<m>} are bead indices of connected beads that run
from 1 to \tt{<beads>} and \tt{<k>} and \tt{<r>} are the strength and
equilibrium distance, respectively, of a harmonic oscillator (AnalysisTools
assumes bonds use harmonic potential).

Block (iii) starts with a line:
\begin{verbatim}angles <n>\end{verbatim}
where \tt{<n>} is the number of angles in the molecule. Following are
\tt{<n>} lines for each angle:
\begin{verbatim}harm <l> <m> <n> <k> <alpha>\end{verbatim}
where \tt{<l>}, \tt{<m>}, and \tt{<n>} are bead indices defining the angle
that run from 1 to \tt{<beads>} and \tt{<k>} and \tt{<alpha>} are the
strength and equilibrium angle, respectively, of a harmonic oscillator
(AnalysisTools assumes angles use harmonic potential). This block is optional.

Anything between the last angle line and the \tt{finish} keyword is ignored.

Several examples of a \field file can be found in the \tt{Examples}
directory, e.g., in \tt{Examples/AddToSystem}.
\end{comment}
 %}}}

\subsection{LAMMPS \data file} %{{{
 %}}}
  %}}}

\section{Aggregate file (\tt{agg})} \label{ssec:AggFile} %{{{

A \tt{file.agg} is generated using
\tt{Aggregates} (or \tt{Aggregates-NotSameBeads}) utility. The
file contains information about the number of aggregates in each timestep
and which molecules and monomeric (i.e., unbonded) beads belong to which
aggregate. It serves as an additional input file for utilities that
calculate aggregate properties; \tt{agg} file is, therefore, linked to
the \vcf file that was used to generate it.

The \tt{agg} file is a simple text file. The first line contains the
command used to generate it -- parts of this command may be necessary for
subsequent analysis of aggregates. The second line is blank, and from the
third line, the data for individual timesteps are shown. It follows these
rules:

\begin{itemize}[topsep=0pt,itemsep=0pt]
  \item each timestep starts with \tt{Step: <int>} (only \tt{Step}
    keyword is read by the utilities)
  \item the second line is the number of aggregates in the given
    timestep and is followed by a blank line
  \item there are two lines for each aggregate:
  \begin{enumerate}[topsep=0pt,itemsep=0pt]
    \item number of molecules in the aggregate followed by their indices
      taken from the \vsf file (\tt{resid} indices)
    \item number of monomeric beads in the aggregate followed by their
      indices taken from the \vsf file (\tt{atom} indices)
  \end{enumerate}
  \item no blank or comment lines are allowed inside the aggregate block
  \item not all molecules present in the \vcf file used to generate
    this file must be present in every timestep
\end{itemize}

Note that the term aggregate also refers to free chains (i.e., fully
dissolved chains).

When the keyword \tt{Last} is present instead of \tt{Step}, it
signalises the end of the \tt{agg} file; no utility will read anything
beyond this keyword.

Besides using this file for further analysis by other utilities, the
indices can be used in \href{http://www.ks.uiuc.edu/Research/vmd/}{vmd} to
visualize, e.g., only a specific aggregate by using \tt{resid 1 2 3} in the
\tt{Selected Atoms} box in vmd.

An example of an \tt{agg} file can be found in the \tt{Examples/DistrAgg}
directory. %}}}
