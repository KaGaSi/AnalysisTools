\section{Selected} \label{sec:Selected}

This utility creates a new coordinate file of the \xyz, \vtcf, or \tt{lammpstrj}
format. By default, the utility saves all bead and molecules types (basically
transforming one coordinate file format into another), but using \tt{-bt} and/or
\tt{-mt} options specifies which bead and/or molecule types to exclude from the
output file. If \tt{--reverse} option is used, the specified bead and/or
molecule types are instead the only ones that are written into the output file.
Should both \tt{-bt} and \tt{-mt} (as well as \tt{--reverse}) be omitted, the
utility effectively transforms between the supported coordinate file types.

Besides the standard \tt{-st}, \tt{-e}, and \tt{-sk} options, which timesteps to
save can be explicitly specified via the \tt{-n} option that can take a maximum
of 100 arguments (the \tt{-st}, \tt{-e}, and \tt{-sk} are then ignored). Also,
using \tt{--last} option saves only the last valid step from the input
coordinate file (all the previous options are ignored if \tt{--last} is used).
If LAMMPS \data file is used as a coordinate file, all the above options are
ignored as the \data file contains by definition only a single timestep.

There is also an option to remove periodic boundary conditions for molecules
(i.e., to join them) via the \tt{--join} switch. Conversely, the simulation box
can be wrapped (i.e., the periodic boundary conditions applied, putting all
beads inside the box) via the \tt{--wrap} switch. If both \tt{--wrap} and
\tt{--join} options are used, the simulation box is first wrapped and then the
molecules are joined.

Lastly, all coordinates can be scaled, i.e., each coordinate divided by a given
value (\tt{-sc} option) and/or moved by specified vector (\tt{-m} option)
These happen before wrapping/joining of the coordinates, and scaling takes place
before moving.

\vspace{1em}
\noindent
Usage: \tt{Selected <input> <output> <bead type(s)> [options]}
\noindent
\begin{longtable}{p{0.22\textwidth}p{0.724\textwidth}}
  \toprule
  \multicolumn{2}{l}{Mandatory arguments}\\
  \midrule
  \tt{<input>}        & input coordinate file\\
  \tt{<output>}       & output coordinate file\\
  \midrule
  \multicolumn{2}{l}{Options}\\
  \midrule
  \tt{-bt <bead type>} & bead types to exclude\\
  \tt{-mt <mol type>}  & molecule types to exclude\\
  \tt{--reverse}       & save only specified types instead of excluding them\\
  \tt{--join}          & join molecules by removing periodic boundary
                         conditions\\
  \tt{--wrap}          & wrap simulation box (i.e., apply periodic boundary
                         conditions)\\
  \tt{-n <int(s)>}     & save only specified timesteps\\
  \tt{--last}          & save only the last step\\
  \tt{-sc <float>}     & divide all coordinates by given value\\
  \tt{-m 3×<float>}    & add specified vector to all coordinates\\
  \midrule
  \multicolumn{2}{l}{Other options (see the beginning of
                     Chapter~\ref{chap:Utils})}\\
  \midrule
  \multicolumn{2}{p{0.948\textwidth}}{\tt{-st},
                                      \tt{-e},
                                      \tt{-sk},
                                      \tt{-i},
                                      \tt{--verbose},
                                      \tt{--silent},
                                      \tt{--help},
                                      \tt{--version}}\\
  \bottomrule
\end{longtable}
