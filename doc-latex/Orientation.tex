\section{Orientation} \label{sec:Orientation}

This utility calculates orientation order parameter for specified
molecules. It determines distribution of the values as well as average values.

Orientation order parameter $s_{ij}$ indicates how stretched molecules in,
e.g., membranes are. It is defined as
\begin{equation} \label{eq:Orient}
  s_{ij} = \frac{1}{2} \left(3 \cos \theta - 1 \right),
\end{equation}
where
\begin{equation} \label{eq:Cos}
  \cos \theta =
  \frac{\mathbf{r_{ij}\cdot\mathbf{n}}}{|\mathbf{r_{ij}}||\mathbf{n}|}.
\end{equation}
That is, the angle $\theta$ is between a normal vector to a plane,
\textbf{n}, and a vector connecting beads $i$ and $j$, $\mathbf{r_{ij}}$.
The value of $s$ is between -0.5 (parallel to that plane) and 1
(perpendicular to that plane).

The default normal is z-axis, but using \texttt{-a} option, a different
axis can be used. The \texttt{-a} option takes as an argument \texttt{x},
\texttt{y}, or \texttt{z}.

The two beads specifying the second vector can be defined via the
\texttt{-a} option (similarly to the \texttt{-n} option in
\texttt{AngleMolecules}, Section~\ref{sec:AngleMolecules}).  The default is
the first and second bead of the molecule, i.e., \texttt{-a 1 2}.  A
multiple of two indices can be provided to calculate different $s_{ij}$.
All provided index pairs are used for all molecule types, unless both
indices in a pair are higher than the number of beads in the molecule type.
If only one index is higher, then this index is reduced to the highest
index for the given molecule. For example, consider two molecule types to
be used: \texttt{mol1} with four beads and \texttt{mol2} with two beads.
Should \texttt{-a 1 2 1 3 3 4} be specified, \texttt{Orientation} would
calculate $s_{1,2}$, $s_{1,3}$, and $s_{3,4}$ for \texttt{mol1} and
$s_{1,2}$ and $s_{1,2}$ for \texttt{mol2} (it would literally calculate and
print the same result twice). The pair \texttt{3 4} would be ignored for
\texttt{mol2}.

The \texttt{<output>} file contains distributions of the orientation order
parameter with averages of the parameter appended at the end.

Usage:

\vspace{1em}
\noindent
\texttt{Orientation <input> <width> <output> <mol name(s)> <options>}

\noindent
\begin{longtable}{p{0.20\textwidth}p{0.744\textwidth}}
  \toprule
  \multicolumn{2}{l}{Mandatory arguments} \\
  \midrule
  \texttt{<input>} & input coordinate file (either \texttt{vcf} or
    \texttt{vtf} format) \\
  \texttt{<width>} & width of each bin of the distribution \\
  \texttt{<output>} & output file with distribution of the orientation
    order parameter \\
  \texttt{<mol name(s)>} & molecule name(s) used for calculation \\
  \toprule
  \multicolumn{2}{l}{Non-standard options} \\
  \midrule
  \texttt{-st <int>} & starting timestep for calculation (default: 1) \\
  \texttt{-e <int>} & ending timestep for calculation (default: none) \\
  \texttt{-a <axis>} & axis to use as the normal vector (default: z) \\
  \texttt{-n <ints>} & bead indices in the molecules (multiple of two;
    default: 1 2) \\
  \bottomrule
\end{longtable}

\noindent
Format of output files:
\begin{enumerate}[nosep,leftmargin=20pt]
  \item \texttt{<output>} -- distributions of orientation order parameters,
    $s_{ij}$
    \begin{itemize}[nosep,leftmargin=5pt]
      \item first line: command used to generate the file
      \item next lines: column headers (on line per specified molecule type)
        \begin{itemize}[nosep,leftmargin=10pt]
          \item first column is the centre of each bin (governed by
            \texttt{width}); i.e., if \texttt{<width>} is 0.1, then the
            centre of the first bin is -0.45 (because
            $s_{ij}\in\langle0.5,1\rangle$), centre of the second bin -0.35,
            etc.
          \item next, each line shows the bead $i$ and $j$ indices in the
            specified molecule (according to the provided \texttt{vsf}
            file) to be used for calculation of $s_{ij}$
        \end{itemize}
      \item next lines: calculated data
      \item last line: commented out (i.e., the line starts with
        \texttt{\#}) overall averages
      \item commented out lines right above the last line: column headers
        for the averages (same as at the beginning of the file)
    \end{itemize}
\end{enumerate}
