\section{Average} \label{sec:Average}

This utility uses the binning method to analyse data in a text file. It
does not use any of the standard options and prints the result only to
standard output (e.i., screen).

\texttt{Average} calculates average, statistical error, and estimate of the
autocorrelation time $\tau$. Empty lines and comments (lines beginning with
\texttt{\#}) are skipped. \texttt{Average} prints to standard output (i.e.,
the screen) four numbers: \texttt{<n\_blocks> <average> <std error> <tau
estimate>}:
\begin{longtable}{ll}
  \toprule
  \texttt{<n\_blocks>} & number of blocks used for the binning analysis \\
  \texttt{<average>} & simple arithmetic average \\
  \texttt{<std error>} & one-$\sigma$ statistical error \\
  \texttt{<tau estimate>} & estimate of autocorrelation time $\tau$ \\
  \bottomrule
\end{longtable}

A way to quickly estimate a `real' value of $\tau$ is to use a wide range
of \texttt{<n\_blocks>} value and plot the \texttt{<tau estimate>} values
as a function of \texttt{<n\_blocks>}. Because the number of data points in
one block of the binning analysis should be significantly larger than
$\tau$ (e.g., ten times larger), plotting $f(x)=$(number of data lines in
the file)$/(10x$) will produce an exponential function that intersects the
\texttt{<tau estimate>} line. A value of \texttt{<tau estimate>} near the
intersection (but to the left, where the exponential is above \texttt{<tau
average>}) can be used as a good estimate of $\tau$.

Usage:

\vspace{1em}
\noindent
\texttt{Average <input> <column> <discard> <n\_blocks>}

\noindent
\begin{longtable}{p{0.15\textwidth}p{0.794\textwidth}}
  \toprule
  \multicolumn{2}{l}{Mandatory arguments} \\
  \midrule
  \texttt{<input>} & input text file \\
  \texttt{<column>} & column number in \texttt{<input>} for data analysis \\
  \texttt{<discard>} & number of lines to discard from the beginning of
    \texttt{<input>} \\
  \texttt{<n\_blocks>} & number of blocks for binning analysis \\
  \bottomrule
\end{longtable}
