\section{Average} \label{sec:Average}

This utility uses the binning method to analyse data in a text file. It
does not use any of the standard options and prints the result only to
standard output (e.i., screen).

\texttt{Average} calculates average value, statistical error, and estimate of
autocorrelation time $\tau$. Empty lines and comments (lines beginning with
\texttt{\#}) are skipped. \texttt{Average} prints to standard output (i.e.,
the screen) four numbers: \texttt{<n\_blocks> <average> <std error> <tau
estimate>}:
\begin{longtable}{ll}
  \toprule
  \texttt{<n\_blocks>} & number of blocks used for the binning analysis \\
  \texttt{<average>} & simple arithmetic average \\
  \texttt{<std error>} & one-$\sigma$ statistical error \\
  \texttt{<tau estimate>} & estimate of autocorrelation time $\tau$ \\
  \bottomrule
\end{longtable}

The average value of an observable $\mathcal{O}$ is a simple arithmetic
mean:
\begin{equation} \label{eq:Average} %{{{
  \langle\mathcal{O}\rangle = \frac{1}{N} \sum^N_{i=1} \mathcal{O}_i,
\end{equation} %}}}
where $N$ is the number of measurements and the subscript $i$ denotes
individual measurements. If the measurements are independent (i.e.,
uncorrelated), the statistical error, $\epsilon$, is given by:
\begin{equation} \label{eq:IndependentError} %{{{
  \epsilon^2 =
    \frac{\sigma^2_{\mathcal{O}_i}}{N},
\end{equation} %}}}
where $\sigma^2_{\mathcal{O}_i}$
is the variance of the individual
measurements,
\begin{equation} %{{{
  \sigma^2_{\mathcal{O}_i} = \frac{1}{N-1} \sum^N_{i=1} (\mathcal{O}_i -
  \langle\mathcal{O}\rangle)^2.
\end{equation} %}}}

For correlated data, the autocorrelation time,
$\tau$, representing the number of steps between two uncorrelated
measurements must be determined. Every
$\tau$-th measurement is uncorrelated, so the
equation~\eqref{eq:IndependentError} can then be used to estimate the
error.

A commonly used method to estimate $\tau$
is the binning (or block) method. In this method, the correlated data are
divided into $N_{\mr{B}}$ non-overlapping blocks of size $k$ ($N=k
N_{\mr{B}}$) with per-block averages, $\mathcal{O}_{\mr{B},n}$,
defined as:
\begin{equation} \label{eq:BlockAverage} %{{{
  \mathcal{O}_{\mr{B},n} = \frac{1}{k}
    \sum^{k n}_{\substack{i=1+\\(n-1)k}} \mathcal{O}_i.
\end{equation} %}}}
If $k\gg\tau$, the blocks are assumed to be uncorrelated and
equation~\eqref{eq:IndependentError} can be used:
\begin{equation} \label{eq:Error} %{{{
  \epsilon^2 =
  \frac{\sigma^2_{\mr{B}} }{N_{\mr{B}} } = \frac{1}{N_{\mr{B}} (N_{\mr{B}}
  -1)} \sum^{N_{\mr{B}} }_{n=1} (\mathcal{O}_{\mr{B},n} -
  \overline{\mathcal{O}})^2.
\end{equation} %}}}
An estimate of the autocorrelation time can be obtained using the following
formula:
\begin{equation} \label{eq:tau} %{{{
  \tau_{\mathcal{O}} = \frac{k \sigma^2_{\mr{B}} }{2
  \sigma^2_{\mathcal{O}_i}}.
\end{equation} %}}}

A way to quickly get $\tau$ estimate is to use a wide range
of \texttt{<n\_blocks>} value and plot the \texttt{<tau estimate>} values
as a function of \texttt{<n\_blocks>}. Because the number of data points in
one block of the binning analysis should be significantly larger than
$\tau$ (e.g., ten times larger), plotting $f(x)=$(number of data lines in
the file)$/(10x$) will produce an exponential function that intersects the
\texttt{<tau estimate>} line. A value of \texttt{<tau estimate>} near the
intersection (but to the left, where the exponential is above \texttt{<tau
average>}) can be considere a safe estimate of $\tau$.

Usage:

\vspace{1em}
\noindent
\texttt{Average <input> <column> <discard> <n\_blocks>}

\noindent
\begin{longtable}{p{0.15\textwidth}p{0.794\textwidth}}
  \toprule
  \multicolumn{2}{l}{Mandatory arguments} \\
  \midrule
  \texttt{<input>} & input text file \\
  \texttt{<column>} & column number in \texttt{<input>} for data analysis \\
  \texttt{<discard>} & number of data lines to discard from the beginning of
    \texttt{<input>} \\
  \texttt{<n\_blocks>} & number of blocks for binning analysis \\
  \bottomrule
\end{longtable}
