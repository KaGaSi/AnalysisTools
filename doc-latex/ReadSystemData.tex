\section{Read system data} \label{sec:ReadSystemData}

\subsection[VtfReadStruct]{Read \vtf structure file}\label{ssec:VtfReadStruct}

This function reads an input \vsf file line by line, identifying all beads and
molecules present there. There are two modes that some utilities can use: bead
types are defined either by name only (the default behaviour), or by name, mass,
charge, and radius. In the latter case, the bead types that should share a name
are renamed by appending \tt{_\#} (where \tt{\#} is 1, 2,\ldots, etc.).

\TODO requires splitting into more functions---I guess\ldots

\CallInit{void VtfReadStruct}
\begin{longtable}{L{0.36\textwidth}Mp{0.49\textwidth}}
  \toprule
  parameter          & in/out & explanation \\
  \midrule
  \ttb{(char) struct_file[]}         & in  & name of the input \vsf file\\
  \ttb{(bool) detailed}              & in  & should the bead type determination
                                             be based on more than just name?\\
  \ttb{(COUNTS) Counts}              & out & basic system information
                                             (section \ref{ssec:Counts})\\
  \ttb{(BEADTYPE) BeadType[]}        & out & information about bead types
                                             (section \ref{ssec:BeadType})\\
  \ttb{(BEAD) Bead[]}                & out & information about individual beads
                                             (section \ref{ssec:Bead})\\
  \ttb{(int) Index[]}                & out & array connecting internal and  \vtf
                                             bead indices\\
  \ttb{(MOLECULETYPE) MoleculeType[]}& out & information about molecule types
                                             (section \ref{ssec:MoleculeType})\\
  \ttb{(MOLECULE) Molecule[]}        & out & information about individual
                                             molecules (section
                                             \ref{ssec:Molecule})\\
  \bottomrule
\end{longtable}
\begin{enumerate}
  \item read \ttb{struct_file} line by line
    \begin{itemize}
      \item save all information from \tt{atom} and \tt{bond} lines
        \begin{itemize}
          \item only the first \tt{atom default} line is saved
        \end{itemize}
      \item count
        \begin{itemize}
          \item beads (i.e., identify the highest index in \tt{atom <id>}
            lines), saving into \ttb{Counts.BeadsInVsf}
          \item molecules, saving into \ttb{Counts.Molecules}
          \item unique bead and molecule names (and save those names)
          \item \tt{atom} and \tt{bond} lines (\TODO necessary to mention here?)
        \end{itemize}
      \item stop reading when
        \begin{itemize}
          \item end of file or \tt{timestep} line encountered
          \item unrecognised line or coordinate line encounted and exit program
            with error
        \end{itemize}
        \begin{itemize}
          \item end of file or \tt{timestep} line encountered
          \item unrecognised line or coordinate line encounted and exit program
            with error
        \end{itemize}
    \end{itemize}
  \item save number of unbonded and bonded beads into \ttb{Counts.Unbonded} and
    \ttb{Counts.Bonded}
  \item identify bead types
    \begin{itemize}
      \item save the number of types to \ttb{Counts.TypesOfBeads}
      \item fill a temporary \nameref{ssec:BeadType} array
    \end{itemize}
    \begin{enumerate}
      \item if \ttb{detailed} is \tt{true}, identify the types based on names,
        mass, charge, and radius
        \begin{enumerate}
          \item create a new bead type for every \tt{atom} line that has a
            unique combination of name, mass, charge, and radius
          \item merge some of the bead types, specifically:
            \begin{itemize}[label=$-$]
              \item if a keyword is missing in one \tt{atom} line but present
                in another does not count as a different type; e.g. beads\\
                \tt{atom 0 n x m 1 q 1}\\
                \tt{atom 1 n x m 1}\\
                are of the same type (with charge $+1$)
              \item however, there can be ambiguities, so e.g., beads\\
                \tt{atom 0 n x m 1 q 1}\\
                \tt{atom 1 n x m 1}\\
                \tt{atom 2 n x m 1 q 0}\\
                remain as three distinct types (with charges 0, \tt{undefined},
                and $+1$)
              \item but only some \tt{atom} lines are ambiguous; e.g., beads\\
                \tt{atom 0 n x m 1 q 1}\\
                \tt{atom 1 n x m 1}\\
                \tt{atom 2 n x m 1 q 0}\\
                \tt{atom 3 n x q 0}\\
                are still of three types with the same charges as above and
                with mass 1 (there is no ambiguity as \tt{atom 3} is the same as
                \tt{atom 2} except for the undefined mass)
              \item note that sometimes the charge, mass, and/or radius can
                remain undefined even though there is exactly one well defined
                value; e.g., in case of \tt{atom} lines\\
                \tt{atom 0 n x m 1 q 1}\\
                \tt{atom 1 n x m 1}\\
                \tt{atom 2 n x m 1 q 0}\\
                \tt{atom 3 n x q 0}\\
                \tt{atom 4 n x m 1 q 0 r 1}\\
                beads sharing their type with \tt{atom 4} will have well
                defined radius of 1 (i.e., \tt{atom 2}, \tt{3}, and \tt{4}),
                while \tt{atom 0} and \tt{1} will have radius undefined as they
                have different charge to \tt{atom 4}; three bead types are
                recognized here:
                  \begin{enumerate}
                    \item type with $m=1$, $q=1$, and radius \tt{undefined}
                      (\tt{atom 0})
                    \item type with $m=1$ and charge and radius \tt{undefined}
                      (\tt{atom 1})
                    \item type with $m=1$, $q=0$, and $r=1$ (\tt{atom 2},
                      \tt{3}, and \tt{4})
                  \end{enumerate}
            \end{itemize}
        \end{enumerate}
      \item if \ttb{detailed} is \tt{false}, identify the types by name only
        \begin{itemize}
          \item mass, charge, and radius for each bead type is taken from the
            first \tt{atom} line of the given name where the respective
            parameter is defined
        \end{itemize}
    \end{enumerate}
  \item fill a temporary \nameref{ssec:Bead} array and a temporary associated
    index (i.e., \ttb{Index}) array
    \begin{itemize}
      \item put unbonded beads first and the bonded beads after them
      \item if \ttb{detailed} is \tt{false}, only bead name is checked to
        determine its type, otherwise its name, mass, charge, and radius are all
        checked
    \end{itemize}
  \item if \ttb{detailed} is \tt{true}, rename bead types with the same name
    \begin{itemize}
      \item when several bead types share a name, the name remains unchanged for
        the first one, but \tt{_\#} is added to all subsequent ones (\tt{\#}
        goes from 1 to $N$, where $N$ is the number of bead types with the same
        name)
      \item if the bead name would become too long (i.e., over 16 characters),
        it is shortened before the \tt{_\#} is added
    \end{itemize}
  \item identify molecules and molecule types
    \begin{itemize}
      \item save the number of types to \ttb{Counts.TypesOfMolecules}
      \item fill temporary \nameref{ssec:MoleculeType} and
        \nameref{ssec:Molecule} arrays
      \item molecules must share all information to be of the same type
        \begin{itemize}
          \item molecule name
          \item numbers of beads and bonds
          \item connectivity (i.e., the bonds must be identical)
          \item order of bead types (i.e., the order of the beads' indices in
            \ttb{struct_file})
        \end{itemize}
    \end{itemize}
  \item copy data from temporary arrays to \ttb{Bead}, \ttb{BeadType},
    \ttb{Index}, \ttb{Molecule}, and \ttb{MoleculeType}
\end{enumerate}

% old stuff with full pseudocode %{{{
\begin{comment}
This function reads an input \vsf file line by line, identifying all beads and
molecules present there.

\TODO requires splitting into more functions---I guess\ldots

\CallInit{void}{VtfReadStruct}
\begin{longtable}{L{0.36\textwidth}Mp{0.51\textwidth}}
  \toprule
  variable           & in/out & explanation \\
  \midrule
  \ttb{(char) struct_file[]}         & in  & name of the input \vsf file\\
  \ttb{(bool) detailed}              & in  & should the bead/molecule type
                                             determination be based on more than
                                             just name?\\
  \ttb{(COUNTS) Counts}              & out & basic system information \TODO add
                                             ref\\
  \ttb{(BEADTYPE) BeadType[]}        & out & information about bead types \TODO
                                             add ref\\
  \ttb{(BEAD) Bead[]}                & out & information about individual beads
                                             \TODO add ref\\
  \ttb{(int) Index[]}                & out & array connecting internal and  \vtf
                                             bead indices\\
  \ttb{(MOLECULETYPE) MoleculeType[]}& out & information about molecule types
                                             \TODO add ref\\
  \ttb{(MOLECULE) Molecule[]}       & out & information about individual
                                            molecules \TODO add ref\\
  \bottomrule
\end{longtable}
\vspace{-1em}
\begin{algorithmic}[1]
  \St initialize empty \ttb{Counts}
  \St open \ttb{struct_file}
  \St $file\_line\_count\gets0$\Comment{total number of lines in
    \ttb{struct_file}}
  \Stx $count\_atoms\gets0$\Comment{number of \hltt{atom} lines in
    \ttb{struct_file}}
  \Stx $default\_atom\gets0$\Comment{line number of the first
    \hltt{atom default} line (0 if no \hltt{atom default})}
  \Stx $count\_bonds\gets0$\Comment{number of \hltt{bond} lines in
    \ttb{struct_file}}
  \Stx $atom\_names\gets0$\Comment{number of unique bead names in
    \ttb{struct_file}}
  \Stx $res\_names\gets0$\Comment{number of unique molecule names in
    \ttb{struct_file}}
  \Stx $highest\_resid\gets-1$\Comment{highest molecule index (\hltt{resid
    <id>}) in the \ttb{struct_file}}
  \Stx $atom\_name[]$ and $resn\_name[]$\Comment{arrays for the unique
    names (memory allocated as needed)}
  \Stx $atom[]$, $atom\_def$, and $bond[]$\Comment{structures with information
    from \hltt{atom} and \hltt{bond} lines}
  \Stx $mol\_id[]$\Comment{array for \hltt{resid} numbers in \ttb{struct_file}
    (memory allocated as needed)}

  \Stx\vspace{-0.6em}\hspace{-17pt}\rule{\textwidth}{0.3pt}
  \Stx i) get all \hltt{atom} and \hltt{bond} information from
    \ttb{struct_file}
  \Stx\vspace{-0.6em}\hspace{-17pt}\rule{\textwidth}{0.3pt}
  \While{$line\gets$ get a line from \ttb{input_vcf}}
    \St $file\_line\_count\gets file\_line\_count+1$

    \Stx\vspace{-0.6em}\hspace{7pt}\rule{0.927\textwidth}{0.3pt}
    \Stx\hspace{6pt} reading \hltt{atom} line
    \Stx\vspace{-0.6em}\hspace{7pt}\rule{0.927\textwidth}{0.3pt}
    \If{$line$ is \hltt{atom} line}

      \Stx\vspace{-0.6em}\hspace{14pt}\rule{0.927\textwidth}{0.3pt}
      \Stx\hspace{12pt} detecting \hltt{atom default} bead
      \Stx\vspace{-0.6em}\hspace{14pt}\rule{0.927\textwidth}{0.3pt}
      \If{$line$ is \hltt{atom default} line}
        \If{not the first \hltt{atom default} line}
          \St warn that this \hltt{atom default} line is ignored
        \Else
          \St $default\_atom\gets file\_line\_count$
          \If{bead name not in $atom\_name[]$}
            \St $atom\_name[atom\_names]\gets$ bead name
            \St $atom\_names\gets atom\_names+1$
          \EndIf
          \St $atom\_def\gets$ mass, charge, and radius from the \hltt{atom}
            line
        \EndIf

      \Stx\vspace{-0.6em}\hspace{14pt}\rule{0.927\textwidth}{0.3pt}
      \Stx\hspace{12pt} reading \hltt{atom <id>} line and counting beads and
        molecules
      \Stx\vspace{-0.6em}\hspace{14pt}\rule{0.927\textwidth}{0.3pt}
      \Else
        \St $count\_atoms\gets count\_atoms+1$
        \St $id\gets$ bead index from the \hltt{atom} line
        \If{$id>$ \ttb{Counts.BeadsInVsf}}
          \For{$i=$ \ttb{Counts.BeadsInVsf}..$id-1$}
            \St $atom[i]\gets$ \hltt{default} bead\Comment{i.e., assume $i$ has
              no \hltt{atom} line; may change later}
          \EndFor
          \St \ttb{Counts.BeadsInVsf} $\gets$ $id+1$\Comment{$+1$ as
            \vtf bead indices start from 0}
        \EndIf
        \If{bead name not in $atom\_name[]$}
          \St $atom\_name[atom\_names]\gets$ bead name
          \St $atom\_names\gets atom\_names+1$
        \EndIf
        \St $atom[id]\gets$ mass, charge, and radius from the \hltt{atom} line
        \If{bead $id$ is in a molecule}
          \St $resid\gets$ molecule index
          \St $atom[id]\gets resid$
          \If{molecule name not in $res\_name[]$}
            \St $res\_name[res\_names]\gets$ molecule name
            \St $res\_names\gets res\_names+1$
          \EndIf
          \If{$resid > highest\_resid$}
            \For{$i=highest\_resid+1$..$resid$}
              \St $mol\_id[i]\gets-1$\Comment{marks index $i$ as not yet
                detected (discontinuous \hltt{resid} indices)}
            \EndFor
            \St $highest\_resid\gets resid$
          \EndIf
          \If{$mol\_id[resid]=-1$}\Comment{molecule index $resid$ detected for
            the first time}
            \St $mol\_id[resid]\gets$
              \ttb{Counts.Molecules}
            \St \ttb{Counts.Molecules} $\gets$ \ttb{Counts.Molecules}$+1$
          \EndIf
        \Else\Comment{bead $id$ is not in a molecule}
          \St $atom[id]\gets$ not in a molecule
        \EndIf
      \EndIf

    \Stx\vspace{-0.6em}\hspace{7pt}\rule{0.943\textwidth}{0.3pt}
    \Stx\hspace{6pt} reading \hltt{bond} line
    \Stx\vspace{-0.6em}\hspace{7pt}\rule{0.943\textwidth}{0.3pt}
    \ElsIf{$line$ is \hltt{bond} line}
      \St $bond[count\_bonds]\gets$ bead indices from the \hltt{bond} file
      \St $count\_bonds\gets count\_bonds+1$

    \Stx\vspace{-0.6em}\hspace{7pt}\rule{0.943\textwidth}{0.3pt}
    \Stx\hspace{6pt} \hltt{timestep} line constitutes end of \vsf part of
      \ttb{struct_file} (for full \vtf file)
    \Stx\vspace{-0.6em}\hspace{7pt}\rule{0.943\textwidth}{0.3pt}
    \ElsIf{$line$ is a \hltt{timestep} line}
      \Break

    \Stx\vspace{-0.6em}\hspace{7pt}\rule{0.943\textwidth}{0.3pt}
    \Stx\hspace{6pt} exit program with error when unrecognised or coordinate
      line is encountered
    \Stx\vspace{-0.6em}\hspace{7pt}\rule{0.943\textwidth}{0.3pt}
    \ElsIf{$line$ is not recognised}
      \Error unrecognised line
    \ElsIf{$line$ is coordinate line}
      \Error coordinate line encountered inside \vsf file block
    \EndIf
  \EndWhile
  \St close \ttb{struct_file}

  \Stx\vspace{-0.6em}\rule{0.96\textwidth}{0.3pt}
  \Stx check all beads are defined in \ttb{struct_file}
  \Stx\vspace{-0.6em}\rule{0.96\textwidth}{0.3pt}
  \If{$default\_atom=0$ \And $count\_atoms \neq$ \ttb{Counts.BeadsInVsf}}
    \Error \hltt{atom default} line is omitted, but not every bead is defined
      in an \hltt{atom <id>} line
  \EndIf

  \Stx\vspace{-0.6em}\rule{0.96\textwidth}{0.3pt}
  \Stx create and fill array connecting \ttb{struct_file} molecule indices
    (\hltt{resid <id>}) with internal ones
  \Stx\vspace{-0.6em}\rule{0.96\textwidth}{0.3pt}
  \St allocate memory for $mol\_id\_internal[]$
  \For{$i=0$..$highest\_resid-1$}
    \If{$mol\_id[i]\neq-1$}\Comment{is index $i$ defined in \ttb{struct_file}?}
      \St $mol\_id\_internal[mol\_id[i]]\gets i$
    \EndIf
  \EndFor

  \Stx\vspace{-0.6em}\rule{0.96\textwidth}{0.3pt}
  \Stx assign default values to \hltt{default} beads and count bonded/unbonded
    beads
  \Stx\vspace{-0.6em}\rule{0.96\textwidth}{0.3pt}
  \For{$i=0$..\ttb{Counts.BeadsInVsf}$-1$}
    \If{$atom[i]$ is \hltt{default} bead}
      \St $atom[i]\gets atom\_def$
    \EndIf
    \St count bead $i$ towards \ttb{Counts.Unbonded} or \ttb{Counts.Bonded}
  \EndFor

  \Stx\vspace{-0.6em}\hspace{-17pt}\rule{\textwidth}{0.3pt}
  \Stx ii) identify bead types
  \Stx\vspace{-0.6em}\hspace{-17pt}\rule{\textwidth}{0.3pt}
  \St define BEADTYPE (\TODO add ref) array $bt\_tmp[]$\Comment{memory allocated
    as needed}

  \Stx\vspace{-0.6em}\rule{0.96\textwidth}{0.3pt}
  \Stx identify bead types by name, mass, charge, and radius
  \Stx\vspace{-0.6em}\rule{0.96\textwidth}{0.3pt}
  \If{\ttb{detailed}}

    \Stx\vspace{-0.6em}\hspace{7pt}\rule{0.943\textwidth}{0.3pt}
    \Stx\hspace{6pt} create a bead type for each \hltt{atom} line with unique
      name, mass, charge, and radius
    \Stx\vspace{-0.6em}\hspace{7pt}\rule{0.943\textwidth}{0.3pt}
    \If{$default\_atom\neq0$}
      \St $bt\_tmp[0]\gets$ new bead type with \hltt{default} mass, charge, and
        radius
      \St \ttb{Counts.TypesOfBeads} $\gets$
        \ttb{Counts.TypesOfBeads}$+1$\Comment{in a function creating
        $bt\_tmp[0]$}
    \EndIf
    \For{$i=0$..$count\_atoms-1$}
      \If{$atom[i]$'s name, charge, mass, and radius do not match an existing
        bead type}
        \St $bt\_tmp[$\ttb{Counts.TypesOfBeads}$]\gets$ new bead type with
          $atom[i]$'s mass, charge, and radius
        \St \ttb{Counts.TypesOfBeads} $\gets$ \ttb{Counts.TypesOfBeads}$+1$
          \Comment{in a function creating $bt\_tmp[]$}
      \EndIf
      \St $bt\_tmp[atom[i]$'s type$]\gets bt\_tmp[atom[i]$'s
        type$]+1$\Comment{count beads of each type}
    \EndFor
    \If{$default\_atom\neq0$}
      \St $bt\_tmp[0].Number=$ \ttb{Counts.BeadsInVsf}
      \For{$i=0$..\ttb{Counts.TypesOfBeads}$-1$}
        \St $bt\_tmp[0].Number\gets bt\_tmp[0]-bt\_tmp[i].Number$
      \EndFor
    \EndIf

    \Stx\vspace{-0.6em}\hspace{7pt}\rule{0.943\textwidth}{0.3pt}
    \Stx\hspace{6pt} merge some of the created bead types (see above for
      details)
    \Stx\vspace{-0.6em}\hspace{7pt}\rule{0.943\textwidth}{0.3pt}
    \St $diff\_q[]\gets bt\_tmp[].Charge$\Comment{copy charge for all bead
      types}
    \St $diff\_m[]\gets bt\_tmp[].Mass$\Comment{copy mass for all bead types}
    \St $diff\_r[]\gets bt\_tmp[].Radius$\Comment{copy radius for all bead
      types}

    \Stx\vspace{-0.6em}\hspace{14pt}\rule{0.927\textwidth}{0.3pt}
    \Stx\hspace{12pt} find which bead types with unique names have ambiguous
      charge, mass, and radius
    \Stx\vspace{-0.6em}\hspace{14pt}\rule{0.927\textwidth}{0.3pt}
    \For{$i=0$..$atom\_names-1$}
      \For{$j=0$..\ttb{Counts.TypesOfBeads}$-1$}
        \If{$atom\_name[i]=bt\_tmp[j].Name$}
          \If{$diff\_q[i]\neq bt\_tmp[j].Charge$}
            \If{both $diff\_q[i]$ and $bt\_tmp[j].Charge$ are well defined}
              \St $diff\_q[i]\gets$1e6\Comment{more than one well defined value
                of charge exists}
            \ElsIf{$diff\_q[i]$ undefined}
              \St $diff\_q[i]\gets bt\_tmp[j].Charge$
            \EndIf
          \EndIf
          \St similarly for $diff\_m$ and $diff\_r$
        \EndIf
      \EndFor
    \EndFor

  \Stx\vspace{-0.6em}\rule{0.96\textwidth}{0.3pt}
  \Stx identify bead types only by name
  \Stx\vspace{-0.6em}\rule{0.96\textwidth}{0.3pt}
  \Else\Comment{identify bead types by name only}
    \St $bt\_tmp[]\gets$ new bead type for each name in $atom\_name[]$
    \For{$i=0$..$atom\_names-1$}
      \St $bt\_tmp[i]\gets$ new bead type with \hltt{default}/undefined
        mass, charge, and radius
      \St \ttb{Counts.TypesOfBeads} $\gets$
        \ttb{Counts.TypesOfBeads}$+1$\Comment{in a function creating $bt\_tmp[i]$}
    \EndFor
    \For{$i=0$..\ttb{Counts.BeadsInVsf}$-1$}
      \St $btype\gets$ bead type based on $i$'s name\Comment{error when
        $btype=-1$ (should be impossible)}
      \If{$bt\_tmp[btype].Mass$ is undefined \And $atom[i]$'s mass is defined}
        \St $bt\_tmp[btype].Mass\gets atom[i]$'s mass
      \EndIf
      \St similarly for charge and radius
    \EndFor
  \EndIf

% \St $file\_line\_count \gets 0$\Comment{count lines; line number printed in
%   case of an error}
% \While{$line\gets$ get a line from \ttb{input_vcf}}
%   \St $file\_line\_count\gets file\_line\_count+1$
%   \If{$line$ is \hltt{pbc} line}
%     \St \ttb{Box.Length}$\gets$ 1st to 3rd string from $line$ as box
%       dimensions
%     \St \ttb{Box}$\gets 90^\circ$ as all three angles\Comment{assume
%       orthogonal box}
%     \If{\hltt{pbc} line contains angles}
%       \St \ttb{Box} $\gets$ 4th to 6th string from $line$ as
%         angles\Comment{possibly triclinic box}
%     \EndIf
%     \Break\Comment{box dimensions found, so there's nothing more to do}
%   \ElsIf{$line$ is a coordinate line}
%     \Error missing box dimensions in \ttb{input_vcf}
%   \ElsIf{$line$ is unrecognised}
%     \Error invalid line in \ttb{input_vcf}
%   \EndIf
% \EndWhile
% \St close \ttb{input_vcf}
\end{algorithmic}
\algbottomrule
\end{comment}
 %}}}

\subsection[VtfReadTimestep]{Read single timestep from \vtf coordinate
file}\label{ssec:VtfReadTimestep}

This function reads a single timestep (its preamble and the coordinate block)
from an input \vcf file, identifying what beads are present via
\ttb{Bead[].InTimestep} flag.

\CallInit{void VtfReadTimestep}
\begin{longtable}{L{0.36\textwidth}Mp{0.49\textwidth}}
  \toprule
  parameter          & in/out & explanation \\
  \midrule
  \ttb{(FILE) vcf}                   & in  & pointer to open \vcf file\\
  \ttb{(char) vcf_file[]}            & in  & name of the input \vcf file\\
  \ttb{(BOX) Box}                    & out & box dimensions (section
                                             \ref{ssec:Box})\\
  \ttb{(COUNTS) Counts}              & out & basic system information
                                             (section \ref{ssec:Counts})\\
  \ttb{(BEADTYPE) BeadType[]}        & out & information about bead types
                                             (section \ref{ssec:BeadType})\\
  \ttb{(BEAD) Bead[]}                & out & information about individual beads
                                             (section \ref{ssec:Bead})\\
  \ttb{(int) Index[]}                & out & array connecting internal and  \vtf
                                             bead indices\\
  \ttb{(MOLECULETYPE) MoleculeType[]}& in \& out & information about molecule types
                                             (section \ref{ssec:MoleculeType})\\
  \ttb{(MOLECULE) Molecule[]}        & out & information about individual
                                             molecules (section
                                             \ref{ssec:Molecule})\\
  \ttb{(int) step_count}             & in \& out & counter of read timesteps\\
  \bottomrule
\end{longtable}
\begin{enumerate}
  \item read \ttb{struct_file} line by line
    \begin{itemize}
      \item save all information from \tt{atom} and \tt{bond} lines
        \begin{itemize}
          \item only the first \tt{atom default} line is saved
        \end{itemize}
      \item count
        \begin{itemize}
          \item beads (i.e., identify the highest index in \tt{atom <id>}
            lines), saving into \ttb{Counts.BeadsInVsf}
          \item molecules, saving into \ttb{Counts.Molecules}
          \item unique bead and molecule names (and save those names)
          \item \tt{atom} and \tt{bond} lines (\TODO necessary to mention here?)
        \end{itemize}
      \item stop reading when
        \begin{itemize}
          \item end of file or \tt{timestep} line encountered
          \item unrecognised line or coordinate line encounted and exit program
            with error
        \end{itemize}
        \begin{itemize}
          \item end of file or \tt{timestep} line encountered
          \item unrecognised line or coordinate line encounted and exit program
            with error
        \end{itemize}
    \end{itemize}
  \item save number of unbonded and bonded beads into \ttb{Counts.Unbonded} and
    \ttb{Counts.Bonded}
  \item identify bead types
    \begin{itemize}
      \item save the number of types to \ttb{Counts.TypesOfBeads}
      \item fill a temporary \nameref{ssec:BeadType} array
    \end{itemize}
    \begin{enumerate}
      \item if \ttb{detailed} is \tt{true}, identify the types based on names,
        mass, charge, and radius
        \begin{enumerate}
          \item create a new bead type for every \tt{atom} line that has a
            unique combination of name, mass, charge, and radius
          \item merge some of the bead types, specifically:
            \begin{itemize}[label=$-$]
              \item if a keyword is missing in one \tt{atom} line but present
                in another does not count as a different type; e.g. beads\\
                \tt{atom 0 n x m 1 q 1}\\
                \tt{atom 1 n x m 1}\\
                are of the same type (with charge $+1$)
              \item however, there can be ambiguities, so e.g., beads\\
                \tt{atom 0 n x m 1 q 1}\\
                \tt{atom 1 n x m 1}\\
                \tt{atom 2 n x m 1 q 0}\\
                remain as three distinct types (with charges 0, \tt{undefined},
                and $+1$)
              \item but only some \tt{atom} lines are ambiguous; e.g., beads\\
                \tt{atom 0 n x m 1 q 1}\\
                \tt{atom 1 n x m 1}\\
                \tt{atom 2 n x m 1 q 0}\\
                \tt{atom 3 n x q 0}\\
                are still of three types with the same charges as above and
                with mass 1 (there is no ambiguity as \tt{atom 3} is the same as
                \tt{atom 2} except for the undefined mass)
              \item note that sometimes the charge, mass, and/or radius can
                remain undefined even though there is exactly one well defined
                value; e.g., in case of \tt{atom} lines\\
                \tt{atom 0 n x m 1 q 1}\\
                \tt{atom 1 n x m 1}\\
                \tt{atom 2 n x m 1 q 0}\\
                \tt{atom 3 n x q 0}\\
                \tt{atom 4 n x m 1 q 0 r 1}\\
                beads sharing their type with \tt{atom 4} will have well
                defined radius of 1 (i.e., \tt{atom 2}, \tt{3}, and \tt{4}),
                while \tt{atom 0} and \tt{1} will have radius undefined as they
                have different charge to \tt{atom 4}; three bead types are
                recognized here:
                  \begin{enumerate}
                    \item type with $m=1$, $q=1$, and radius \tt{undefined}
                      (\tt{atom 0})
                    \item type with $m=1$ and charge and radius \tt{undefined}
                      (\tt{atom 1})
                    \item type with $m=1$, $q=0$, and $r=1$ (\tt{atom 2},
                      \tt{3}, and \tt{4})
                  \end{enumerate}
            \end{itemize}
        \end{enumerate}
      \item if \ttb{detailed} is \tt{false}, identify the types by name only
        \begin{itemize}
          \item mass, charge, and radius for each bead type is taken from the
            first \tt{atom} line of the given name where the respective
            parameter is defined
        \end{itemize}
    \end{enumerate}
  \item fill a temporary \nameref{ssec:Bead} array and a temporary associated
    index (i.e., \ttb{Index}) array
    \begin{itemize}
      \item put unbonded beads first and the bonded beads after them
      \item if \ttb{detailed} is \tt{false}, only bead name is checked to
        determine its type, otherwise its name, mass, charge, and radius are all
        checked
    \end{itemize}
  \item if \ttb{detailed} is \tt{true}, rename bead types with the same name
    \begin{itemize}
      \item when several bead types share a name, the name remains unchanged for
        the first one, but \tt{_\#} is added to all subsequent ones (\tt{\#}
        goes from 1 to $N$, where $N$ is the number of bead types with the same
        name)
      \item if the bead name would become too long (i.e., over 16 characters),
        it is shortened before the \tt{_\#} is added
    \end{itemize}
  \item identify molecules and molecule types
    \begin{itemize}
      \item save the number of types to \ttb{Counts.TypesOfMolecules}
      \item fill temporary \nameref{ssec:MoleculeType} and
        \nameref{ssec:Molecule} arrays
      \item molecules must share all information to be of the same type
        \begin{itemize}
          \item molecule name
          \item numbers of beads and bonds
          \item connectivity (i.e., the bonds must be identical)
          \item order of bead types (i.e., the order of the beads' indices in
            \ttb{struct_file})
        \end{itemize}
    \end{itemize}
  \item copy data from temporary arrays to \ttb{Bead}, \ttb{BeadType},
    \ttb{Index}, \ttb{Molecule}, and \ttb{MoleculeType}
\end{enumerate}

%\subsection[VtfReadPBC]{Get simulation box dimensions}\label{ssec:VtfReadPBC}

This function reads an input \vcf file line by line, searching for a valid
\tt{pbc} line that contains box dimensions and, for triclinic cells, angles.

Valid line: \tt{pbc <x> <y> <x> [<alpha> <beta> <gamme>]}.

\CallInit{void}{VtfReadPBC}
\begin{longtable}{m{0.23\textwidth}
                  >{\centering}m{0.2\textwidth}
                  m{0.57\textwidth}}
  \toprule
  variable           & in/out & explanation \\
  \midrule
  \ttb{(char) input_vcf[]} & in  & name of the input \vcf file \\
  \ttb{(BOX) Box}          & out & information about simulation box dimensions
    \TODO add ref\\
  \bottomrule
\end{longtable}
\vspace{-1em}
\begin{algorithmic}[1]
  \St \True
  \St \False
  \St open \ttb{input_vcf}
  \St open \ttb{input_vcf}
  \St open \ttb{input_vcf}
  \St open \ttb{input_vcf}
  \St open \ttb{input_vcf}
  \St open \ttb{input_vcf}
  \St open \ttb{input_vcf}
  \St open \ttb{input_vcf}
  \St open \ttb{input_vcf}
  \St open \ttb{input_vcf}
  \St open \ttb{input_vcf}
  \St open \ttb{input_vcf}
  \St open \ttb{input_vcf}
  \St open \ttb{input_vcf}
  \St open \ttb{input_vcf}
  \St open \ttb{input_vcf}
  \St open \ttb{input_vcf}
  \St open \ttb{input_vcf}
  \St open \ttb{input_vcf}
  \St open \ttb{input_vcf}
  \St open \ttb{input_vcf}
  \St open \ttb{input_vcf}
  \St open \ttb{input_vcf}
  \St open \ttb{input_vcf}
  \St open \ttb{input_vcf}
  \St open \ttb{input_vcf}
  \St open \ttb{input_vcf}
  \St open \ttb{input_vcf}
  \St open \ttb{input_vcf}
  \St open \ttb{input_vcf}
  \St open \ttb{input_vcf}
  \St open \ttb{input_vcf}
  \St open \ttb{input_vcf}
  \St open \ttb{input_vcf}
  \St open \ttb{input_vcf}
  \St open \ttb{input_vcf}
  \St open \ttb{input_vcf}
  \St open \ttb{input_vcf}
  \St open \ttb{input_vcf}
  \St open \ttb{input_vcf}
  \St open \ttb{input_vcf}
  \St open \ttb{input_vcf}
  \St open \ttb{input_vcf}
  \St open \ttb{input_vcf}
  \St open \ttb{input_vcf}
  \St open \ttb{input_vcf}
  \St open \ttb{input_vcf}
  \St open \ttb{input_vcf}
  \St open \ttb{input_vcf}
  \St open \ttb{input_vcf}
  \St open \ttb{input_vcf}
  \St open \ttb{input_vcf}
  \St open \ttb{input_vcf}
  \St open \ttb{input_vcf}
  \St open \ttb{input_vcf}
  \St open \ttb{input_vcf}
  \St open \ttb{input_vcf}
  \St open \ttb{input_vcf}
  \St open \ttb{input_vcf}
  \St open \ttb{input_vcf}
  \St open \ttb{input_vcf}
  \St open \ttb{input_vcf}
  \St open \ttb{input_vcf}
  \St open \ttb{input_vcf}
  \St open \ttb{input_vcf}
  \St open \ttb{input_vcf}
  \St open \ttb{input_vcf}
  \St open \ttb{input_vcf}
  \St open \ttb{input_vcf}
  \St open \ttb{input_vcf}
  \St open \ttb{input_vcf}
  \St open \ttb{input_vcf}
  \St open \ttb{input_vcf}
  \St open \ttb{input_vcf}
  \St open \ttb{input_vcf}
  \St open \ttb{input_vcf}
  \St open \ttb{input_vcf}
  \St open \ttb{input_vcf}
  \St open \ttb{input_vcf}
  \St open \ttb{input_vcf}
  \St open \ttb{input_vcf}
  \St open \ttb{input_vcf}
  \St open \ttb{input_vcf}
  \St open \ttb{input_vcf}
  \St open \ttb{input_vcf}
  \St open \ttb{input_vcf}
  \St open \ttb{input_vcf}
  \St open \ttb{input_vcf}
  \St open \ttb{input_vcf}
  \St open \ttb{input_vcf}
  \St open \ttb{input_vcf}
  \St open \ttb{input_vcf}
  \St open \ttb{input_vcf}
  \St open \ttb{input_vcf}
  \St open \ttb{input_vcf}
  \St open \ttb{input_vcf}
  \St open \ttb{input_vcf}
  \St open \ttb{input_vcf}
  \St open \ttb{input_vcf}
  \St open \ttb{input_vcf}
\end{algorithmic}
\vspace{0.5em}\hrule
test
test
test
test
test
test
test
test
test
test
test
test
test
test
test
test
test
test
test
test
test
test
test
test
test
test
test
test
test
test
test
test
test
test
test
test
test
test
test
test
test
test
test
test
test
test
test
test
test
test
test
test
test
test
test
test
test
test
test
test
test
test
test
test
test
test
test
test
test
test
test
test
test
test
test
test
test
test
test
test
test
test
test
test
test
test
test
test
test
test
test
test
test
test
test
test
test
test
test
test

 %not used anymore
%\subsection[VtfCheckTimestep]{Find what beads are in a \vtf
timestep}\label{ssec:VtfCheckTimestep}

This functions identifies what beads and molecules are in a single \vtf
timestep. \TODO make it check every timestep; but that will seriously slow down
the utilities, won't it?

\CallInit{bool VtfCheckTimestep}
\begin{longtable}{L{0.36\textwidth}Mp{0.51\textwidth}}
  \toprule
  parameter          & in/out & explanation \\
  \midrule
  \ttb{(char) vcf_file[]}            & in  & name of the input \vcf file\\
  \ttb{(COUNTS) Counts}              & out & basic system information
                                             (section \ref{ssec:Counts})\\
  \ttb{(BEADTYPE) BeadType[]}        & out & information about bead types
                                             (section \ref{ssec:BeadType})\\
  \ttb{(BEAD) Bead[]}                & out & information about individual beads
                                             (section \ref{ssec:Bead})\\
  \ttb{(int) Index[]}                & out & array connecting internal and  \vtf
                                             bead indices\\
  \ttb{(MOLECULETYPE) MoleculeType[]}& out & information about molecule types
                                             (section \ref{ssec:MoleculeType})\\
  \ttb{(MOLECULE) Molecule[]}        & out & information about individual
                                             molecules (section
                                             \ref{ssec:Molecule})\\
  \multicolumn{3}{p{\textwidth}}{\Ret \tt{true} for indexed timestep, \tt{false}
    for ordered one}\\
  \bottomrule
\end{longtable}
\begin{itemize}
  \item the \ttb{input_vcf} is read line by line until one of the following
    happens:
    \begin{itemize}
      \item valid \tt{pbc} line is encountered
      \item coordinate line is encountered; exit program with error
      \item invalid line is encountered, i.e., anything beside \tt{atom},
        \tt{bond}, comment (\tt{\#}-initiated line), or \tt{timestep} line; exit
        program with error
    \end{itemize}
\end{itemize}
 %not used anymore

\begin{comment} %{{{
In the pseudocodes, the following are used:
\begin{itemize}
  \item \ttb{blue} \ldots variable that is a parameter of the described
    function
  \item $italics$ \ldots \ variable defined within the described function
  \item square brackets with a variable means the variable is an array (with
    number of elements corresponding to what is inside the brackets)
  \item \hltt{high[light]} \ldots \ literal string (square brackets denote
    optional part of the string, i.e., only \hltt{high} would be looked for in
    this example)
  \item \textsc{SmallCaps} \ldots \ function call to a function not described
    within the manual
  \item \textsc{\textcolor{red}{RedSmallCaps}} \ldots \ function call
    referencing to a section with the called function
  \item \textbf{error} keyword \ldots \ exit a program with an error
  \item the triangle represent a short comment, while comments describing a
    whole section of the function are line-separated from the pseudocode
\end{itemize}
The following are rules for some common actions (unless stated otherwise in the
pseudocode):
\begin{itemize}
  \item reading a line from a file is usually done using
    \textsc{ReadAndSplitLine} function from \tt{General} library; this function:
    \begin{enumerate}
      \item reads at most 1023 characters of line, discarding the rest of the
        line
      \item splits the line into individual whitespace-separated strings using
        the \textsc{SplitLine} function from \tt{General} libreary; at most 31
        strings are created, each at most 63 characters long
    \end{enumerate}
  \item checking if a string is a number---generally written as $string
    \in$ (or $\notin$) \Real/\RealP/\Int---uses one of these functions from
    \tt{General} library:
    \begin{itemize}
      \item \textsc{IsReal} \ldots \ real number (\Real); that is, any number
      \item \textsc{IsPosReal} \ldots \ positive real number (\RealP)
      \item \textsc{IsInteger} \ldots \ integer (\Int)
    \end{itemize}
  \item checking what any given line is uses \textsc{CheckVtfLine} function from
    \tt{Read} library to differentiate between comment, blank, \tt{atom},
    \tt{bond}, \tt{pbc}, \tt{timestep} or coordinate lines in a \vtf file
\end{itemize}

\subsection[VtfReadPBC]{Get simulation box dimensions}\label{ssec:VtfReadPBC}

This function reads an input \vcf file line by line, searching for a valid
\tt{pbc} line that contains box dimensions and, for triclinic cells, angles.

Valid line: \tt{pbc <x> <y> <x> [<alpha> <beta> <gamme>]}.

\CallInit{void}{VtfReadPBC}
\begin{longtable}{m{0.23\textwidth}
                  >{\centering}m{0.2\textwidth}
                  m{0.57\textwidth}}
  \toprule
  variable           & in/out & explanation \\
  \midrule
  \ttb{(char) input_vcf[]} & in  & name of the input \vcf file \\
  \ttb{(BOX) Box}          & out & information about simulation box dimensions
    \TODO add ref\\
  \bottomrule
\end{longtable}
\vspace{-1em}
\begin{algorithmic}[1]
  \St \True
  \St \False
  \St open \ttb{input_vcf}
  \St open \ttb{input_vcf}
  \St open \ttb{input_vcf}
  \St open \ttb{input_vcf}
  \St open \ttb{input_vcf}
  \St open \ttb{input_vcf}
  \St open \ttb{input_vcf}
  \St open \ttb{input_vcf}
  \St open \ttb{input_vcf}
  \St open \ttb{input_vcf}
  \St open \ttb{input_vcf}
  \St open \ttb{input_vcf}
  \St open \ttb{input_vcf}
  \St open \ttb{input_vcf}
  \St open \ttb{input_vcf}
  \St open \ttb{input_vcf}
  \St open \ttb{input_vcf}
  \St open \ttb{input_vcf}
  \St open \ttb{input_vcf}
  \St open \ttb{input_vcf}
  \St open \ttb{input_vcf}
  \St open \ttb{input_vcf}
  \St open \ttb{input_vcf}
  \St open \ttb{input_vcf}
  \St open \ttb{input_vcf}
  \St open \ttb{input_vcf}
  \St open \ttb{input_vcf}
  \St open \ttb{input_vcf}
  \St open \ttb{input_vcf}
  \St open \ttb{input_vcf}
  \St open \ttb{input_vcf}
  \St open \ttb{input_vcf}
  \St open \ttb{input_vcf}
  \St open \ttb{input_vcf}
  \St open \ttb{input_vcf}
  \St open \ttb{input_vcf}
  \St open \ttb{input_vcf}
  \St open \ttb{input_vcf}
  \St open \ttb{input_vcf}
  \St open \ttb{input_vcf}
  \St open \ttb{input_vcf}
  \St open \ttb{input_vcf}
  \St open \ttb{input_vcf}
  \St open \ttb{input_vcf}
  \St open \ttb{input_vcf}
  \St open \ttb{input_vcf}
  \St open \ttb{input_vcf}
  \St open \ttb{input_vcf}
  \St open \ttb{input_vcf}
  \St open \ttb{input_vcf}
  \St open \ttb{input_vcf}
  \St open \ttb{input_vcf}
  \St open \ttb{input_vcf}
  \St open \ttb{input_vcf}
  \St open \ttb{input_vcf}
  \St open \ttb{input_vcf}
  \St open \ttb{input_vcf}
  \St open \ttb{input_vcf}
  \St open \ttb{input_vcf}
  \St open \ttb{input_vcf}
  \St open \ttb{input_vcf}
  \St open \ttb{input_vcf}
  \St open \ttb{input_vcf}
  \St open \ttb{input_vcf}
  \St open \ttb{input_vcf}
  \St open \ttb{input_vcf}
  \St open \ttb{input_vcf}
  \St open \ttb{input_vcf}
  \St open \ttb{input_vcf}
  \St open \ttb{input_vcf}
  \St open \ttb{input_vcf}
  \St open \ttb{input_vcf}
  \St open \ttb{input_vcf}
  \St open \ttb{input_vcf}
  \St open \ttb{input_vcf}
  \St open \ttb{input_vcf}
  \St open \ttb{input_vcf}
  \St open \ttb{input_vcf}
  \St open \ttb{input_vcf}
  \St open \ttb{input_vcf}
  \St open \ttb{input_vcf}
  \St open \ttb{input_vcf}
  \St open \ttb{input_vcf}
  \St open \ttb{input_vcf}
  \St open \ttb{input_vcf}
  \St open \ttb{input_vcf}
  \St open \ttb{input_vcf}
  \St open \ttb{input_vcf}
  \St open \ttb{input_vcf}
  \St open \ttb{input_vcf}
  \St open \ttb{input_vcf}
  \St open \ttb{input_vcf}
  \St open \ttb{input_vcf}
  \St open \ttb{input_vcf}
  \St open \ttb{input_vcf}
  \St open \ttb{input_vcf}
  \St open \ttb{input_vcf}
  \St open \ttb{input_vcf}
  \St open \ttb{input_vcf}
  \St open \ttb{input_vcf}
\end{algorithmic}
\vspace{0.5em}\hrule
test
test
test
test
test
test
test
test
test
test
test
test
test
test
test
test
test
test
test
test
test
test
test
test
test
test
test
test
test
test
test
test
test
test
test
test
test
test
test
test
test
test
test
test
test
test
test
test
test
test
test
test
test
test
test
test
test
test
test
test
test
test
test
test
test
test
test
test
test
test
test
test
test
test
test
test
test
test
test
test
test
test
test
test
test
test
test
test
test
test
test
test
test
test
test
test
test
test
test
test


\subsection[VtfReadStruct]{Read \vtf structure file}\label{ssec:VtfReadStruct}

This function reads an input \vsf file line by line, identifying all beads and
molecules present there. There are two modes that some utilities can use: bead
types are defined either by name only (the default behaviour), or by name, mass,
charge, and radius. In the latter case, the bead types that should share a name
are renamed by appending \tt{_\#} (where \tt{\#} is 1, 2,\ldots, etc.).

\TODO requires splitting into more functions---I guess\ldots

\CallInit{void VtfReadStruct}
\begin{longtable}{L{0.36\textwidth}Mp{0.49\textwidth}}
  \toprule
  parameter          & in/out & explanation \\
  \midrule
  \ttb{(char) struct_file[]}         & in  & name of the input \vsf file\\
  \ttb{(bool) detailed}              & in  & should the bead type determination
                                             be based on more than just name?\\
  \ttb{(COUNTS) Counts}              & out & basic system information
                                             (section \ref{ssec:Counts})\\
  \ttb{(BEADTYPE) BeadType[]}        & out & information about bead types
                                             (section \ref{ssec:BeadType})\\
  \ttb{(BEAD) Bead[]}                & out & information about individual beads
                                             (section \ref{ssec:Bead})\\
  \ttb{(int) Index[]}                & out & array connecting internal and  \vtf
                                             bead indices\\
  \ttb{(MOLECULETYPE) MoleculeType[]}& out & information about molecule types
                                             (section \ref{ssec:MoleculeType})\\
  \ttb{(MOLECULE) Molecule[]}        & out & information about individual
                                             molecules (section
                                             \ref{ssec:Molecule})\\
  \bottomrule
\end{longtable}
\begin{enumerate}
  \item read \ttb{struct_file} line by line
    \begin{itemize}
      \item save all information from \tt{atom} and \tt{bond} lines
        \begin{itemize}
          \item only the first \tt{atom default} line is saved
        \end{itemize}
      \item count
        \begin{itemize}
          \item beads (i.e., identify the highest index in \tt{atom <id>}
            lines), saving into \ttb{Counts.BeadsInVsf}
          \item molecules, saving into \ttb{Counts.Molecules}
          \item unique bead and molecule names (and save those names)
          \item \tt{atom} and \tt{bond} lines (\TODO necessary to mention here?)
        \end{itemize}
      \item stop reading when
        \begin{itemize}
          \item end of file or \tt{timestep} line encountered
          \item unrecognised line or coordinate line encounted and exit program
            with error
        \end{itemize}
        \begin{itemize}
          \item end of file or \tt{timestep} line encountered
          \item unrecognised line or coordinate line encounted and exit program
            with error
        \end{itemize}
    \end{itemize}
  \item save number of unbonded and bonded beads into \ttb{Counts.Unbonded} and
    \ttb{Counts.Bonded}
  \item identify bead types
    \begin{itemize}
      \item save the number of types to \ttb{Counts.TypesOfBeads}
      \item fill a temporary \nameref{ssec:BeadType} array
    \end{itemize}
    \begin{enumerate}
      \item if \ttb{detailed} is \tt{true}, identify the types based on names,
        mass, charge, and radius
        \begin{enumerate}
          \item create a new bead type for every \tt{atom} line that has a
            unique combination of name, mass, charge, and radius
          \item merge some of the bead types, specifically:
            \begin{itemize}[label=$-$]
              \item if a keyword is missing in one \tt{atom} line but present
                in another does not count as a different type; e.g. beads\\
                \tt{atom 0 n x m 1 q 1}\\
                \tt{atom 1 n x m 1}\\
                are of the same type (with charge $+1$)
              \item however, there can be ambiguities, so e.g., beads\\
                \tt{atom 0 n x m 1 q 1}\\
                \tt{atom 1 n x m 1}\\
                \tt{atom 2 n x m 1 q 0}\\
                remain as three distinct types (with charges 0, \tt{undefined},
                and $+1$)
              \item but only some \tt{atom} lines are ambiguous; e.g., beads\\
                \tt{atom 0 n x m 1 q 1}\\
                \tt{atom 1 n x m 1}\\
                \tt{atom 2 n x m 1 q 0}\\
                \tt{atom 3 n x q 0}\\
                are still of three types with the same charges as above and
                with mass 1 (there is no ambiguity as \tt{atom 3} is the same as
                \tt{atom 2} except for the undefined mass)
              \item note that sometimes the charge, mass, and/or radius can
                remain undefined even though there is exactly one well defined
                value; e.g., in case of \tt{atom} lines\\
                \tt{atom 0 n x m 1 q 1}\\
                \tt{atom 1 n x m 1}\\
                \tt{atom 2 n x m 1 q 0}\\
                \tt{atom 3 n x q 0}\\
                \tt{atom 4 n x m 1 q 0 r 1}\\
                beads sharing their type with \tt{atom 4} will have well
                defined radius of 1 (i.e., \tt{atom 2}, \tt{3}, and \tt{4}),
                while \tt{atom 0} and \tt{1} will have radius undefined as they
                have different charge to \tt{atom 4}; three bead types are
                recognized here:
                  \begin{enumerate}
                    \item type with $m=1$, $q=1$, and radius \tt{undefined}
                      (\tt{atom 0})
                    \item type with $m=1$ and charge and radius \tt{undefined}
                      (\tt{atom 1})
                    \item type with $m=1$, $q=0$, and $r=1$ (\tt{atom 2},
                      \tt{3}, and \tt{4})
                  \end{enumerate}
            \end{itemize}
        \end{enumerate}
      \item if \ttb{detailed} is \tt{false}, identify the types by name only
        \begin{itemize}
          \item mass, charge, and radius for each bead type is taken from the
            first \tt{atom} line of the given name where the respective
            parameter is defined
        \end{itemize}
    \end{enumerate}
  \item fill a temporary \nameref{ssec:Bead} array and a temporary associated
    index (i.e., \ttb{Index}) array
    \begin{itemize}
      \item put unbonded beads first and the bonded beads after them
      \item if \ttb{detailed} is \tt{false}, only bead name is checked to
        determine its type, otherwise its name, mass, charge, and radius are all
        checked
    \end{itemize}
  \item if \ttb{detailed} is \tt{true}, rename bead types with the same name
    \begin{itemize}
      \item when several bead types share a name, the name remains unchanged for
        the first one, but \tt{_\#} is added to all subsequent ones (\tt{\#}
        goes from 1 to $N$, where $N$ is the number of bead types with the same
        name)
      \item if the bead name would become too long (i.e., over 16 characters),
        it is shortened before the \tt{_\#} is added
    \end{itemize}
  \item identify molecules and molecule types
    \begin{itemize}
      \item save the number of types to \ttb{Counts.TypesOfMolecules}
      \item fill temporary \nameref{ssec:MoleculeType} and
        \nameref{ssec:Molecule} arrays
      \item molecules must share all information to be of the same type
        \begin{itemize}
          \item molecule name
          \item numbers of beads and bonds
          \item connectivity (i.e., the bonds must be identical)
          \item order of bead types (i.e., the order of the beads' indices in
            \ttb{struct_file})
        \end{itemize}
    \end{itemize}
  \item copy data from temporary arrays to \ttb{Bead}, \ttb{BeadType},
    \ttb{Index}, \ttb{Molecule}, and \ttb{MoleculeType}
\end{enumerate}

% old stuff with full pseudocode %{{{
\begin{comment}
This function reads an input \vsf file line by line, identifying all beads and
molecules present there.

\TODO requires splitting into more functions---I guess\ldots

\CallInit{void}{VtfReadStruct}
\begin{longtable}{L{0.36\textwidth}Mp{0.51\textwidth}}
  \toprule
  variable           & in/out & explanation \\
  \midrule
  \ttb{(char) struct_file[]}         & in  & name of the input \vsf file\\
  \ttb{(bool) detailed}              & in  & should the bead/molecule type
                                             determination be based on more than
                                             just name?\\
  \ttb{(COUNTS) Counts}              & out & basic system information \TODO add
                                             ref\\
  \ttb{(BEADTYPE) BeadType[]}        & out & information about bead types \TODO
                                             add ref\\
  \ttb{(BEAD) Bead[]}                & out & information about individual beads
                                             \TODO add ref\\
  \ttb{(int) Index[]}                & out & array connecting internal and  \vtf
                                             bead indices\\
  \ttb{(MOLECULETYPE) MoleculeType[]}& out & information about molecule types
                                             \TODO add ref\\
  \ttb{(MOLECULE) Molecule[]}       & out & information about individual
                                            molecules \TODO add ref\\
  \bottomrule
\end{longtable}
\vspace{-1em}
\begin{algorithmic}[1]
  \St initialize empty \ttb{Counts}
  \St open \ttb{struct_file}
  \St $file\_line\_count\gets0$\Comment{total number of lines in
    \ttb{struct_file}}
  \Stx $count\_atoms\gets0$\Comment{number of \hltt{atom} lines in
    \ttb{struct_file}}
  \Stx $default\_atom\gets0$\Comment{line number of the first
    \hltt{atom default} line (0 if no \hltt{atom default})}
  \Stx $count\_bonds\gets0$\Comment{number of \hltt{bond} lines in
    \ttb{struct_file}}
  \Stx $atom\_names\gets0$\Comment{number of unique bead names in
    \ttb{struct_file}}
  \Stx $res\_names\gets0$\Comment{number of unique molecule names in
    \ttb{struct_file}}
  \Stx $highest\_resid\gets-1$\Comment{highest molecule index (\hltt{resid
    <id>}) in the \ttb{struct_file}}
  \Stx $atom\_name[]$ and $resn\_name[]$\Comment{arrays for the unique
    names (memory allocated as needed)}
  \Stx $atom[]$, $atom\_def$, and $bond[]$\Comment{structures with information
    from \hltt{atom} and \hltt{bond} lines}
  \Stx $mol\_id[]$\Comment{array for \hltt{resid} numbers in \ttb{struct_file}
    (memory allocated as needed)}

  \Stx\vspace{-0.6em}\hspace{-17pt}\rule{\textwidth}{0.3pt}
  \Stx i) get all \hltt{atom} and \hltt{bond} information from
    \ttb{struct_file}
  \Stx\vspace{-0.6em}\hspace{-17pt}\rule{\textwidth}{0.3pt}
  \While{$line\gets$ get a line from \ttb{input_vcf}}
    \St $file\_line\_count\gets file\_line\_count+1$

    \Stx\vspace{-0.6em}\hspace{7pt}\rule{0.927\textwidth}{0.3pt}
    \Stx\hspace{6pt} reading \hltt{atom} line
    \Stx\vspace{-0.6em}\hspace{7pt}\rule{0.927\textwidth}{0.3pt}
    \If{$line$ is \hltt{atom} line}

      \Stx\vspace{-0.6em}\hspace{14pt}\rule{0.927\textwidth}{0.3pt}
      \Stx\hspace{12pt} detecting \hltt{atom default} bead
      \Stx\vspace{-0.6em}\hspace{14pt}\rule{0.927\textwidth}{0.3pt}
      \If{$line$ is \hltt{atom default} line}
        \If{not the first \hltt{atom default} line}
          \St warn that this \hltt{atom default} line is ignored
        \Else
          \St $default\_atom\gets file\_line\_count$
          \If{bead name not in $atom\_name[]$}
            \St $atom\_name[atom\_names]\gets$ bead name
            \St $atom\_names\gets atom\_names+1$
          \EndIf
          \St $atom\_def\gets$ mass, charge, and radius from the \hltt{atom}
            line
        \EndIf

      \Stx\vspace{-0.6em}\hspace{14pt}\rule{0.927\textwidth}{0.3pt}
      \Stx\hspace{12pt} reading \hltt{atom <id>} line and counting beads and
        molecules
      \Stx\vspace{-0.6em}\hspace{14pt}\rule{0.927\textwidth}{0.3pt}
      \Else
        \St $count\_atoms\gets count\_atoms+1$
        \St $id\gets$ bead index from the \hltt{atom} line
        \If{$id>$ \ttb{Counts.BeadsInVsf}}
          \For{$i=$ \ttb{Counts.BeadsInVsf}..$id-1$}
            \St $atom[i]\gets$ \hltt{default} bead\Comment{i.e., assume $i$ has
              no \hltt{atom} line; may change later}
          \EndFor
          \St \ttb{Counts.BeadsInVsf} $\gets$ $id+1$\Comment{$+1$ as
            \vtf bead indices start from 0}
        \EndIf
        \If{bead name not in $atom\_name[]$}
          \St $atom\_name[atom\_names]\gets$ bead name
          \St $atom\_names\gets atom\_names+1$
        \EndIf
        \St $atom[id]\gets$ mass, charge, and radius from the \hltt{atom} line
        \If{bead $id$ is in a molecule}
          \St $resid\gets$ molecule index
          \St $atom[id]\gets resid$
          \If{molecule name not in $res\_name[]$}
            \St $res\_name[res\_names]\gets$ molecule name
            \St $res\_names\gets res\_names+1$
          \EndIf
          \If{$resid > highest\_resid$}
            \For{$i=highest\_resid+1$..$resid$}
              \St $mol\_id[i]\gets-1$\Comment{marks index $i$ as not yet
                detected (discontinuous \hltt{resid} indices)}
            \EndFor
            \St $highest\_resid\gets resid$
          \EndIf
          \If{$mol\_id[resid]=-1$}\Comment{molecule index $resid$ detected for
            the first time}
            \St $mol\_id[resid]\gets$
              \ttb{Counts.Molecules}
            \St \ttb{Counts.Molecules} $\gets$ \ttb{Counts.Molecules}$+1$
          \EndIf
        \Else\Comment{bead $id$ is not in a molecule}
          \St $atom[id]\gets$ not in a molecule
        \EndIf
      \EndIf

    \Stx\vspace{-0.6em}\hspace{7pt}\rule{0.943\textwidth}{0.3pt}
    \Stx\hspace{6pt} reading \hltt{bond} line
    \Stx\vspace{-0.6em}\hspace{7pt}\rule{0.943\textwidth}{0.3pt}
    \ElsIf{$line$ is \hltt{bond} line}
      \St $bond[count\_bonds]\gets$ bead indices from the \hltt{bond} file
      \St $count\_bonds\gets count\_bonds+1$

    \Stx\vspace{-0.6em}\hspace{7pt}\rule{0.943\textwidth}{0.3pt}
    \Stx\hspace{6pt} \hltt{timestep} line constitutes end of \vsf part of
      \ttb{struct_file} (for full \vtf file)
    \Stx\vspace{-0.6em}\hspace{7pt}\rule{0.943\textwidth}{0.3pt}
    \ElsIf{$line$ is a \hltt{timestep} line}
      \Break

    \Stx\vspace{-0.6em}\hspace{7pt}\rule{0.943\textwidth}{0.3pt}
    \Stx\hspace{6pt} exit program with error when unrecognised or coordinate
      line is encountered
    \Stx\vspace{-0.6em}\hspace{7pt}\rule{0.943\textwidth}{0.3pt}
    \ElsIf{$line$ is not recognised}
      \Error unrecognised line
    \ElsIf{$line$ is coordinate line}
      \Error coordinate line encountered inside \vsf file block
    \EndIf
  \EndWhile
  \St close \ttb{struct_file}

  \Stx\vspace{-0.6em}\rule{0.96\textwidth}{0.3pt}
  \Stx check all beads are defined in \ttb{struct_file}
  \Stx\vspace{-0.6em}\rule{0.96\textwidth}{0.3pt}
  \If{$default\_atom=0$ \And $count\_atoms \neq$ \ttb{Counts.BeadsInVsf}}
    \Error \hltt{atom default} line is omitted, but not every bead is defined
      in an \hltt{atom <id>} line
  \EndIf

  \Stx\vspace{-0.6em}\rule{0.96\textwidth}{0.3pt}
  \Stx create and fill array connecting \ttb{struct_file} molecule indices
    (\hltt{resid <id>}) with internal ones
  \Stx\vspace{-0.6em}\rule{0.96\textwidth}{0.3pt}
  \St allocate memory for $mol\_id\_internal[]$
  \For{$i=0$..$highest\_resid-1$}
    \If{$mol\_id[i]\neq-1$}\Comment{is index $i$ defined in \ttb{struct_file}?}
      \St $mol\_id\_internal[mol\_id[i]]\gets i$
    \EndIf
  \EndFor

  \Stx\vspace{-0.6em}\rule{0.96\textwidth}{0.3pt}
  \Stx assign default values to \hltt{default} beads and count bonded/unbonded
    beads
  \Stx\vspace{-0.6em}\rule{0.96\textwidth}{0.3pt}
  \For{$i=0$..\ttb{Counts.BeadsInVsf}$-1$}
    \If{$atom[i]$ is \hltt{default} bead}
      \St $atom[i]\gets atom\_def$
    \EndIf
    \St count bead $i$ towards \ttb{Counts.Unbonded} or \ttb{Counts.Bonded}
  \EndFor

  \Stx\vspace{-0.6em}\hspace{-17pt}\rule{\textwidth}{0.3pt}
  \Stx ii) identify bead types
  \Stx\vspace{-0.6em}\hspace{-17pt}\rule{\textwidth}{0.3pt}
  \St define BEADTYPE (\TODO add ref) array $bt\_tmp[]$\Comment{memory allocated
    as needed}

  \Stx\vspace{-0.6em}\rule{0.96\textwidth}{0.3pt}
  \Stx identify bead types by name, mass, charge, and radius
  \Stx\vspace{-0.6em}\rule{0.96\textwidth}{0.3pt}
  \If{\ttb{detailed}}

    \Stx\vspace{-0.6em}\hspace{7pt}\rule{0.943\textwidth}{0.3pt}
    \Stx\hspace{6pt} create a bead type for each \hltt{atom} line with unique
      name, mass, charge, and radius
    \Stx\vspace{-0.6em}\hspace{7pt}\rule{0.943\textwidth}{0.3pt}
    \If{$default\_atom\neq0$}
      \St $bt\_tmp[0]\gets$ new bead type with \hltt{default} mass, charge, and
        radius
      \St \ttb{Counts.TypesOfBeads} $\gets$
        \ttb{Counts.TypesOfBeads}$+1$\Comment{in a function creating
        $bt\_tmp[0]$}
    \EndIf
    \For{$i=0$..$count\_atoms-1$}
      \If{$atom[i]$'s name, charge, mass, and radius do not match an existing
        bead type}
        \St $bt\_tmp[$\ttb{Counts.TypesOfBeads}$]\gets$ new bead type with
          $atom[i]$'s mass, charge, and radius
        \St \ttb{Counts.TypesOfBeads} $\gets$ \ttb{Counts.TypesOfBeads}$+1$
          \Comment{in a function creating $bt\_tmp[]$}
      \EndIf
      \St $bt\_tmp[atom[i]$'s type$]\gets bt\_tmp[atom[i]$'s
        type$]+1$\Comment{count beads of each type}
    \EndFor
    \If{$default\_atom\neq0$}
      \St $bt\_tmp[0].Number=$ \ttb{Counts.BeadsInVsf}
      \For{$i=0$..\ttb{Counts.TypesOfBeads}$-1$}
        \St $bt\_tmp[0].Number\gets bt\_tmp[0]-bt\_tmp[i].Number$
      \EndFor
    \EndIf

    \Stx\vspace{-0.6em}\hspace{7pt}\rule{0.943\textwidth}{0.3pt}
    \Stx\hspace{6pt} merge some of the created bead types (see above for
      details)
    \Stx\vspace{-0.6em}\hspace{7pt}\rule{0.943\textwidth}{0.3pt}
    \St $diff\_q[]\gets bt\_tmp[].Charge$\Comment{copy charge for all bead
      types}
    \St $diff\_m[]\gets bt\_tmp[].Mass$\Comment{copy mass for all bead types}
    \St $diff\_r[]\gets bt\_tmp[].Radius$\Comment{copy radius for all bead
      types}

    \Stx\vspace{-0.6em}\hspace{14pt}\rule{0.927\textwidth}{0.3pt}
    \Stx\hspace{12pt} find which bead types with unique names have ambiguous
      charge, mass, and radius
    \Stx\vspace{-0.6em}\hspace{14pt}\rule{0.927\textwidth}{0.3pt}
    \For{$i=0$..$atom\_names-1$}
      \For{$j=0$..\ttb{Counts.TypesOfBeads}$-1$}
        \If{$atom\_name[i]=bt\_tmp[j].Name$}
          \If{$diff\_q[i]\neq bt\_tmp[j].Charge$}
            \If{both $diff\_q[i]$ and $bt\_tmp[j].Charge$ are well defined}
              \St $diff\_q[i]\gets$1e6\Comment{more than one well defined value
                of charge exists}
            \ElsIf{$diff\_q[i]$ undefined}
              \St $diff\_q[i]\gets bt\_tmp[j].Charge$
            \EndIf
          \EndIf
          \St similarly for $diff\_m$ and $diff\_r$
        \EndIf
      \EndFor
    \EndFor

  \Stx\vspace{-0.6em}\rule{0.96\textwidth}{0.3pt}
  \Stx identify bead types only by name
  \Stx\vspace{-0.6em}\rule{0.96\textwidth}{0.3pt}
  \Else\Comment{identify bead types by name only}
    \St $bt\_tmp[]\gets$ new bead type for each name in $atom\_name[]$
    \For{$i=0$..$atom\_names-1$}
      \St $bt\_tmp[i]\gets$ new bead type with \hltt{default}/undefined
        mass, charge, and radius
      \St \ttb{Counts.TypesOfBeads} $\gets$
        \ttb{Counts.TypesOfBeads}$+1$\Comment{in a function creating $bt\_tmp[i]$}
    \EndFor
    \For{$i=0$..\ttb{Counts.BeadsInVsf}$-1$}
      \St $btype\gets$ bead type based on $i$'s name\Comment{error when
        $btype=-1$ (should be impossible)}
      \If{$bt\_tmp[btype].Mass$ is undefined \And $atom[i]$'s mass is defined}
        \St $bt\_tmp[btype].Mass\gets atom[i]$'s mass
      \EndIf
      \St similarly for charge and radius
    \EndFor
  \EndIf

% \St $file\_line\_count \gets 0$\Comment{count lines; line number printed in
%   case of an error}
% \While{$line\gets$ get a line from \ttb{input_vcf}}
%   \St $file\_line\_count\gets file\_line\_count+1$
%   \If{$line$ is \hltt{pbc} line}
%     \St \ttb{Box.Length}$\gets$ 1st to 3rd string from $line$ as box
%       dimensions
%     \St \ttb{Box}$\gets 90^\circ$ as all three angles\Comment{assume
%       orthogonal box}
%     \If{\hltt{pbc} line contains angles}
%       \St \ttb{Box} $\gets$ 4th to 6th string from $line$ as
%         angles\Comment{possibly triclinic box}
%     \EndIf
%     \Break\Comment{box dimensions found, so there's nothing more to do}
%   \ElsIf{$line$ is a coordinate line}
%     \Error missing box dimensions in \ttb{input_vcf}
%   \ElsIf{$line$ is unrecognised}
%     \Error invalid line in \ttb{input_vcf}
%   \EndIf
% \EndWhile
% \St close \ttb{input_vcf}
\end{algorithmic}
\algbottomrule
\end{comment}
 %}}}

\end{comment}
%}}}

\begin{comment} %{{{
In general, information about a system can be read from a \vtf file(s),
\tt{DL_MESO} \tt{FIELD} file, \tt{LAMMPS} \tt{data} file, or from a
combination of those (such as most of the data from a \vsf file
supplemented by angles and angle parameters and/or bond parameters from a
\tt{FIELD} and/or \tt{data} file(s)). Cartesian coordinates can then be
read from a \vcf, \tt{xyz}, or \tt{DL_MESO} \tt{CONFIG} file.

In principle, not all beads present in the structure file have to be
present in the coordinate file.

Subsection~\ref{ssec:ReadVtfStructure} describes reading system data from
\vtf file(s).

\subsection{Structure from a \vsf file} \label{ssec:ReadVtfStructure}
\noindent
First, box size is read from a \vcf file (if a \vcf is included in the
calculation), then structure of the complete system is read from a \vsf file, and
lastly, the system can be reduced only to beads present in an associated \vcf
file.

% read pbc from vcf - GetPBC %{{{
\subsubsection{Getting simulation box size}
\ttb{VECTOR GetPBC(char *vcf_file)}\\[-2em]
\begin{longtable}{lcp{85mm}}
  \toprule
  variable           & input/output? & explanation \\
  \midrule
  \ttb{vcf_file}     & in  & name of the \vcf file \\
  \multicolumn{3}{l}{\ttb{GetPBC} returns a \ttb{struct Vector} containing x,
    y, and z side length of the cuboid simulation box}\\
  \bottomrule
\end{longtable}
\noindent
This function simply reads the \vcf file until it encounter the \tt{pbc
<double> <double> <double>} line. %}}}

% read vsf file - ReadVtfStructure() %{{{
\subsubsection{Getting complete system information}
\ttb{void ReadVtfStructure(char *vsf_file, bool detailed, COUNTS *Counts,\\
BEADTYPE **BeadType, BEAD **Bead, int **Index,\\
MOLECULETYPE **MoleculeType, MOLECULE **Molecule)}\\[-2em]
\begin{longtable}{lcp{85mm}}
  \toprule
  variable           & input/output? & explanation \\
  \midrule
  \ttb{vsf_file}     & in  & name of the \tt{vsf} file \\
  \ttb{detailed}     & in  & mode for differentiating bead and molecule types \\
  \ttb{Counts}       & out & broad system information \\
  \ttb{BeadType}     & out & information about bead types \\
  \ttb{Bead}         & out & information about individual beads \\
  \ttb{Index}        & out & connection between in-code bead indices and \vsf indices \\
  \ttb{MoleculeType} & out & information about molecule types \\
  \ttb{Molecule}     & out & information about individual molecules \\
  \bottomrule
\end{longtable}
\noindent
The procedure to get all information is as follows:
\begin{enumerate}
  \item Go through the \vsf to get:
    \begin{itemize}
      \item number of \tt{a[tom]} lines (\ttb{(int)count_atom_lines})
      \item number of \tt{b[ond]} lines (\ttb{(int)count_bond_lines})
      \item if present, the line number of the first \tt{a[tom] default}
        line\\(\ttb{(int)default_atom_line})
      \item bead names into \ttb{(char **)atom_name} 2D array (size
        \ttb{atom_names}$\times$\tt{17} to save at most 16
        characters of each unique name)
      \item molecule names into \ttb{(char **)res_name} 2D array (size
        \ttb{res_names}$\times$\tt{9} to save at most 8
        characters of each unique name)
    \end{itemize}
  \item go through the \vsf again to count molecules and save all
    \tt{a[tom]} and \tt{b[ond]} lines into:
    \begin{itemize}
      \item \ttb{(struct)atom[i]} for $i$-th \tt{a[tom]} line,
        i.e., an array of structures (a \tt{default} line,
        if present, is the last element of the array), containing members:
    \end{itemize}
        \begin{longtable}{lp{100mm}}
          \toprule
          \ttb{(int)index}     & \tt{a[tom] <int>} keyword \\
          \ttb{(int)name}      & \tt{n[ame] <char(16)>} keyword; corresponds
            to an element in the \ttb{atom\_name} array holding the actual names \\
          \ttb{(int)resid}     & \tt{resid <int>} keyword \\
          \ttb{(int)resname}   & \tt{res[name] <char(8)>} keyword;
            corresponds to an element in the \ttb{res\_name} array holding
            the actual names \\
          \ttb{(double)charge} & \tt{charge|q <double>} keyword \\
          \ttb{(double)mass}   & \tt{m[ass] <double>} keyword \\
          \ttb{(double)radius} & \tt{r[adius] <double>} keyword \\
          \bottomrule
        \end{longtable}
    \begin{itemize}
      \item \ttb{(int)atom_id} array connecting \ttb{atom[i]} elements with \vsf
        bead indices; i.e., \ttb{atom\_id[<vsf id>]=i} (so that
        \ttb{atom[atom_id[<vsf id>]]} structure contains data from $i$-th
        \tt{a[tom]} line)
      \item \ttb{(struct)bond[j]} for the two connected beads in $j$-th
        bond (members \ttb{(int)index1} and \ttb{(int)index2} contain \vsf
        bead indices); i.e., an array of structures, so that the two
        \ttb{atom[atom_id[bond[j].<member>]]} structures contain data from
        the two connected beads' \tt{a[tom]} lines
    \end{itemize}
  \item go through the \ttb{atom} array to identify bead types and populate
    \ttb{(struct BeadType)bt} -- two modes
    \begin{itemize}
      \item \ttb{detailed=true}: bead types are distinguished based on
        their name, mass, charge, and radius -- if there's only one value
        of charge/mass/radius, even beads from \tt{a[tom]} lines missing
        that keyword are of the same type; conversely, if there are two
        values of charge/mass/radius and some \tt{a[tom]} lines are missing
        that keyword, three bead types are created with one bead type that
        has the charge/mass/radius undefined
      \item \ttb{detailed=false}: bead types are distinguished only
        according to their names and their charge/mass/radius is set
        according to the first \tt{a[tom]} line with that name that's not
        missing the appropriate keyword
    \end{itemize}
  \item go through the \ttb{atom} array to populate the \ttb{(struct
    Bead)bead_all} and construct \ttb{(int)index_all} array connecting
    in-code bead indices and \vsf indices, i.e., \ttb{index_all[<vsf
    id>]=<in-code id>} and thus \ttb{bead_all[index_all[<vsf
    id>]].Index=<vsf id>}
    \begin{itemize}
      \item internally, unbonded beads are placed before bonded beads,
        i.e., in all utilities, first \ttb{Counts.Unbonded} beads in a
        \ttb{(struct Bead)} array are unbonded and beads in molecules are
        behind those
      \item \tt{default} beads are assigned \vsf indices according to
        left-out numbers in the \vsf file
    \end{itemize}
  \item go through the \ttb{atom} and \ttb{bond} arrays to identify different
    molecule types and populate the \ttb{(struct MoleculeType)mt} -- two modes
    \begin{itemize}
      \item \ttb{detailed=true}: molecule types are distinguished based on
        their name, connectivity, numbers and order of beads in the molecule
      \item \ttb{detailed=false}: molecule types are distinguished only
        according to their names and their connectivity and bead order are
        determined according to the first molecule with that name
        encountered in the \vsf, i.e., if multiple molecules share a name,
        only the first molecule must have specified bonds
    \end{itemize}
  \item copy all in-function arrays and structure to output arrays and structures
\end{enumerate} %}}}

% reduce system based on vcf file - CheckVtfTimestep() %{{{
\subsubsection{Reducing the system}
\ttb{bool CheckVtfTimestep(FILE *vcf, char *vcf_file, COUNTS *Counts,\\
BEADTYPE **BeadType, BEAD **Bead, int **Index,\\
MOLECULETYPE **MoleculeType, MOLECULE **Molecule)}\\[-2em]
\begin{longtable}{lcp{85mm}}
  \toprule
  variable           & input/output? & explanation \\
  \midrule
  \ttb{vcf}          & in  & pointer to a \vcf file open at the beginning
                       of a timestep \\
  \ttb{vcf_file}     & in  & name of the open \vcf file \\
  \ttb{Counts}       & out & broad system information \\
  \ttb{BeadType}     & out & information about bead types \\
  \ttb{Bead}         & out & information about individual beads \\
  \ttb{Index}        & out & connection between in-code bead indices and \vsf indices \\
  \ttb{MoleculeType} & out & information about molecule types \\
  \ttb{Molecule}     & out & information about individual molecules \\
  \multicolumn{3}{l}{\ttb{CheckVtfTimestep} returns a \tt{true}/\tt{false}
    value whether the \vcf contains indexed timesteps}\\
  \bottomrule
\end{longtable}
\noindent
The procedure to get all information is as follows:
\begin{enumerate}
  \item skip timestep preamble (i.e., anything up to the first coordinate
    line) and determing the timestep type (ordered or indexed)
  \item count coordinate lines and save the bead indices in
    the timestep as \ttb{Bead[i].Flag=true}
    \begin{itemize}
      \item for an ordered timestep, \ttb{i} equals line number; for
        indexed timestep, \ttb{i=Index[<\vsf index>]} (\tt{<vsf index>} is
        the first number of the coordinate line)
    \end{itemize}
  \item for an ordered timestep and for an indexed timestep with all beads
    from the \vsf file present, do nothing more; otherwise, use the
    \ttb{Bead[i].Flag} values to reduce the system, copying the information
    about the new system into new structure arrays
    \begin{itemize}
      \item for \ttb{struct Counts}, all members except for
        \ttb{BeadsInVsf} are adjusted to contain only numbers relevant to
        the beads present in the \vcf file
      \item for \ttb{struct BeadType}, \ttb{struct Bead}, \ttb{struct
        MoleculeType}, and \ttb{struct Molecule}, only the entities present
        in the \vcf file are copied to the new system
      \item for molecule types, it is possible to remove only part(s) of the
        molecules (based on the in-molecule bead types); the numbers of
        bonds and angles are adjusted so that all those containing beads
        not in the \vcf file are removed, possibly creating disjointed
        molecules (or even molecules with no bonds)
    \end{itemize}
  \item copy the new structure arrays back into the original ones, reducing
    the sizes of those accordingly
\end{enumerate}
 %}}}
\end{comment}
%}}}
