\chapter{Introdution}

\section{General Information}
AnalysisTools is a set of utilities to analyse trajectories of
particle-based simulations. There are currently 37 utilities (including one
from the \href{https://www.scd.stfc.ac.uk/Pages/DL_MESO.aspx}{DL\_MESO
simulation package} that was slightly adapted).

Their functionality ranges from calculating system-wide properties such as
pair correlation functions or particle density in the simulation box to
determining per-molecule and per-aggregate properties (where aggregate
stands for any supramolecular structure) such as shape descriptors or
aggregation numbers.

There are also several utilities to generate/tweak initial configurations
for particle based simulations and create configuration files for the
DL\_MESO simulation package and \href{https://lammps.sandia.gov/}{LAMMPS
molecular dynamic simulator}.

\section{Installation}

All utilities can be compiled on linux using \texttt{cmake} and subsequently
running \texttt{make} on the generated \texttt{Makefile}. It requires
\texttt{C} and \texttt{FORTRAN} compilers.

The compilation should be done in a separate directory, such as
\texttt{\$ROOT\_DIR/build} (where \texttt{\$ROOT\_DIR} is the
\texttt{AnalysisTools} root directory). To create the \texttt{Makefile},
simply run \texttt{cmake \$ROOT\_DIR} in the \texttt{build} directory.  All
utilities are then compiled by running \texttt{make}.  The binaries will be
located in \texttt{build/bin} directory.

Following is a one line command to create the \texttt{build} directory and
compile all utilities (assuming the working directory is AnalysisTools root
directory):

\texttt{mkdir build; cd build; cmake ../; make}

To compile individual utilities, simply run \texttt{make <utility name>} in
the \texttt{build} directory.
