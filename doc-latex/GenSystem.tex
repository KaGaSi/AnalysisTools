\section{GenSystem} \label{sec:GenSystem}

\textit{This utility was not extensively tested and is not a good generator
of initial configuration.}

This simple utility uses modified \texttt{FIELD} file to create
\texttt{vsf} structure file and to generate coordinates that could be used
as an initial configuration. The utility assumes linear chains (no matter
the connectivity in the provided \texttt{FIELD} file; it looks at exactly
the first $n-1$ bonds to get bond lengths, where $n$ is the number of beads
in a molecule). For each molecule type, it uses equilibrium bond length (or
0.7 if the bond length is 0) to construct a prototype molecule that is
fully stretched in $z$-direction. The utility then creates layers of
molecules that are separated by layers of unbonded beads (if there are
any). The utility should fill the whole box with given beads.

The input \texttt{FIELD} file must contain \texttt{species} and
\texttt{molecule} sections, but the \texttt{interaction} section is ignored
(see \texttt{DL\_MESO} manual for details on the \texttt{FIELD} file). The
first line of \texttt{FIELD} that is ignored by \texttt{DL\_MESO} must
start with box dimensions, i.e., with three numbers.

This utility does not have error checking for the provided \texttt{FIELD}
file. If the file is not correct, \texttt{GenSystem} will exhibit
undefined behaviour, that is, it will either freeze, crash, or run without
errors, producing bad output files.

Usage (\texttt{GenSystem} does not use standard options):

\vspace{1em}
\noindent
\texttt{GenSystem <out.vsf> <out.vcf> <options>}

\noindent
\begin{longtable}{p{0.15\textwidth}p{0.794\textwidth}}
  \toprule
  \multicolumn{2}{l}{Mandatory arguments} \\
  \midrule
  \texttt{<out.vsf>} & output \texttt{vsf} structure file \\
  \texttt{<out.vcf>} & output \texttt{vcf} coordinate file \\
  \toprule
  \multicolumn{2}{l}{Options} \\
  \midrule
  \texttt{-f <name>} & FIELD-like file (default: FIELD)\\
  \texttt{-v}        & verbose output that provides information about all
    bead and molecule types \\
  \texttt{-h}        & print help and exit \\
  \bottomrule
\end{longtable}
