\section{GenSystem} \label{sec:GenSystem}

This simple utility uses modified \texttt{FIELD} file to create
\texttt{vsf} structure file and to generate coordinates that could be used
as a simulation's starting point. The utility assumes linear chains and
uses equilibrium bond length to construct a prototype molecule that is
fully stretched in one direction for each molecule type. The utility then
creates layers of molecules that are separated by layers of unbonded beads
(if there are any). The utility should fill the whole box with given beads.

The input \texttt{FIELD} file must contain \texttt{species} and
\texttt{molecule} sections, but the \texttt{interaction} section is ignored
(see \texttt{DL\_MESO} manual for details on the \texttt{FIELD} file). The
first line of \texttt{FIELD} that is ignored by \texttt{DL\_MESO} must
start with box dimensions, i.e., with three numbers (the rest of the file
is ignored).

Usage (\texttt{GenSystem} does not use standard options):

\vspace{1em}
\noindent
\texttt{GenSystem <out.vsf> <out.vcf> <options>}

\noindent
\begin{longtable}{p{0.15\textwidth}p{0.794\textwidth}}
  \toprule
  \multicolumn{2}{l}{Mandatory arguments} \\
  \midrule
  \texttt{<out.vsf>} & output \texttt{vsf} structure file \\
  \texttt{<out.vcf>} & output \texttt{vcf} coordinate file \\
  \toprule
  \multicolumn{2}{l}{Options} \\
  \midrule
  \texttt{-f <name>} & FIELD-like file (default: FIELD)\\
  \texttt{-v}        & verbose output that provides information about all
    bead and molecule types \\
  \texttt{-h}        & print help and exit \\
  \bottomrule
\end{longtable}
