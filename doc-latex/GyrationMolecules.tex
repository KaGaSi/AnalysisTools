\section{GyrationMolecules} \label{sec:GyrationMolecules}

This utility calculates shape descriptors similarly to
\texttt{GyrationAggregates} (Section~\ref{sec:GyrationAggregates}), but for
individual molecules instead for whole aggregates.

Usage:

\vspace{1em}
\noindent
\texttt{GyrationMolecules <input> <output> <mol name(s)> <options>}

\noindent
\begin{longtable}{p{0.265\textwidth}p{0.679\textwidth}}
  \toprule
  \multicolumn{2}{l}{Mandatory arguments} \\
  \midrule
  \texttt{<input>} & input coordinate file (either \texttt{vcf} or
    \texttt{vtf} format) \\
  \texttt{<output>} & output file(s) (one per molecule type) with
    automatic \texttt{-<mol\_name>.txt} ending \\
  \texttt{<mol name(s)>} & molecule name(s) to calculcate shape descriptors for \\
  \toprule
  \multicolumn{2}{l}{Non-standard options} \\
  \midrule
  \texttt{--joined} & specify that \texttt{<input>} contains joined
    coordinates (i.e., periodic boundary conditions for molecules do not
    have to be removed) \\
  \texttt{-bt <bead name(s)>} & bead type(s) to be used for calculation \\
  \texttt{-st <int>} & starting timestep for calculation (default: 1) \\
  \bottomrule
\end{longtable}

\begin{enumerate}[nosep,leftmargin=20pt]
  \item \texttt{<output>} -- per-timestep averages (one file per molecule
    type)
    \begin{itemize}[nosep,leftmargin=5pt]
      \item first line: command used to generate the file
      \item second line: name of molecule type
      \item third line: column headers
        \begin{itemize}[nosep,leftmargin=10pt]
          \item first is timestep
          \item rest are for calculated data: number, weight, and z
            averages (denoted by \texttt{\_n}, \texttt{\_w},
            and \texttt{\_z} respectively) of radius of gyration
            (\texttt{Rg} -- Equation~\eqref{eq:R_G}); number averages of
            relative shape anisotropy
            (\texttt{Anis} -- Equation~\eqref{eq:anis}), acylindricity
            and asphericity (\texttt{Acyl} and \texttt{Aspher},
            respectively -- Equation~\eqref{eq:b})
%           and all three
%           eigenvalues (\texttt{eigen.x}, \texttt{eigen.y}, and
%           \texttt{eigen.z} -- $\lambda_x^2$, $\lambda_y^2$, and
%           $\lambda_z^2$)
        \end{itemize}
    \end{itemize}
\end{enumerate}
