\chapter{Installation} \label{chap:Install}

All programs can be compiled using \texttt{cmake} which generates
\texttt{Makefile} and subsequently running \texttt{make}. It requires
\texttt{C} and \texttt{FORTRAN} compilers.
The compilation should be done in a separate directory, such as
\texttt{build}.

To create the \texttt{Makefule} and compile all utilities,
simply run the following command (assuming \texttt{build} is a subdirectory
of \texttt{AnalysisTools} root directory):

\vspace{1em}
\noindent
\texttt{cmake -DCMAKE\_BUILD\_TYPE=Release -G "Unix Makefiles" ../}
\vspace{1em}

The binaries will be in `bin` subdirectory of `build`.

Version for debugging can be compiled when
\texttt{-DCMAKE\_BUILD\_TYPE=Debug} is used instead of
\texttt{-DCMAKE\_BUILD\_TYPE=Release}.

To compile individual \texttt{C} programs using \texttt{gcc}, run (assuming
you are in a subdirectory of \texttt{AnalysisTools} root directory):

\vspace{1em}
\noindent
\texttt{gcc -O3 -lm ../AnalysisTools.c ../Error.c ../Options.c
PATH\_TO\_PRO}-

\noindent
\texttt{GRAM -o OUTPUT\_NAME}.
\vspace{1em}
