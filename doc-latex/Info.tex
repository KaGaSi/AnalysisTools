\section{Info} \label{sec:Info}

This simple utility just prints information about the system read from the
provided structure file. If \texttt{-c} option is used, only beads in the
provided coordinate file are printed. The information is the same as when
the verbose option (\texttt{-v}) is used with any other utility. Here, the
\texttt{-v} option prints detailed informations about every bead and every
molecule -- this information also contains indices as implemented inside
the \texttt{AnalysisTools} utilities (i.e., not necessarily consistent with
the structure file).

Usage (\texttt{Info} does not use standard options):

\vspace{1em}
\noindent
\texttt{Info <input> <options>}

\vspace{1em}
\noindent
\begin{longtable}{p{0.22\textwidth}p{0.724\textwidth}}
  \toprule
  \multicolumn{2}{l}{Mandatory argument} \\
  \midrule
  \texttt{<input>} & input structure file (either \texttt{.vsf} or
    \texttt{.vtf} format) \\
  \multicolumn{2}{l}{Options} \\
  \texttt{-c} & input coordinate file in either \texttt{.vcf} or
    \texttt{.vtf} format (default: None) \\
  \texttt{-h} & print this help and exit \\
  \bottomrule
\end{longtable}
