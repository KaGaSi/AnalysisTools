\section{Info}\label{sec:Info}

This utility prints system information read from an input structure file,
possibly augmenting it with information from another structure file (\tt{-i
<file>} option) and/or pruning the system to contain only beads from a provided
coordinate file (\tt{-c} option).

Note that using LAMMPS \data, \ltrj or \xyz input structure file automatically
reads coordinates in those files, but using \vtf does not. This is because a
\vtf timestep does not necessarily contain all beads defined in the \vtf
structure section. If coordinates should be read from the input \vtf structure
file (and the system potentially pruned), just use the \tt{-c} option with that
filename.

For \vtf structure files, the bead (and, consequently, molecule) type
identification can be enhanced via the \tt{--detailed} option. By default, bead
types are defined solely based on their name; the other properties (mass,
charge, and radius) are taken from the first \tt{a[tom]} line with the bead of
that name that contains the appropriate keyword. For example, lines

\noindent
\tt{atom default name A m 1}\\
\tt{atom 1 name B}\\
\tt{atom 2 name B m 2}\\
\tt{atom 3 name A m 2}\\
\tt{atom 4 name A}\\[-1em]

\noindent
would define two bead type called \tt{A} and \tt{B} with masses equal to 1 and
2, respectively. Using the \tt{--detailed} option would split the \tt{A} beads
into three types: two with masses equal to 1 and 2, respectively, and a third
one with undefined mass (it remains undefined because there is an
ambiguity---should the last bead have mass 1 or 2?).
% would define two bead type called \tt{A} and \tt{B} with masses equal to 1 and
% 2, respectively. Using the \tt{--detailed} option would split the \tt{A} beads
% into three types (and augment the names): \tt{A} and \tt{A_1} with masses
% equal to 1 and 2, respectively, and \tt{A_2} with undefined mass (it remains
% undefined because there is an ambiguity---should the last bead have mass 1 or
% 2?).
On the other hand, the \tt{B} beads would be assigned the same bead type with
mass equal to 2 (there is no ambiguity because only one mass is specified for
the \tt{B} beads).

Using the \tt{-i} option with an extra structure file assigns bead mass, charge,
and/or radius from the extra file if the bead types in the original file have
unspecified values; this is done only for bead types that share names. Also, if
the two structure files share molecules that have the same bead type order
(which can be initially checked via individual \tt{Info} runs on the two files),
bond types, angles, and angle types can be added to the molecule types in the
original file. Note that to add this extra information, the bead types in the
bonds/angles/dihedrals/impropers must be the same, i.e., all their parameters
(name, mass, charge, radius) must be the same.

\tt{Info} can also write this information into a new file (\tt{-o} option) of
any format. Note that only one timestep is read and, therefore, saved. Also, if
no coordinates are read (i.e., neither \tt{-c} option nor a structure file with
coordinates is used), all coordinates in an output file would be 0.
% Allowed output files are \vtsf file, LAMMPS \data file (either with the
% extension \tt{.data} or a filename without any recognizable
% extension), and DL\_MESO FIELD file (called \field or with the extension
% \tt{.FIELD}).

% Coordinates are printed to output files that support them, i.e., LAMMPS \data or
% \vtf file. These coordinates are 0 if no coordinate file is specified or the
% input structure file does not contain them.

There are a few options for the output structure file. For LAMMPS \data file,
\tt{--mass} can be used to specify LAMMPS atom types by their mass, but print
different charges in the \tt{Atoms} section (i.e., \tt{Info}-recognized bead
types which differ only in charge are aggregated into the same LAMMPS atom types
in the \tt{Mass} section). Furthermore, extra atom types (with mass 1) can be
added via the \tt{-ebt} option; these atoms do not appear in the \tt{Atoms}
section of the \data file and can be used as, e.g., extra atom type when srp
potential (i.e., non-crossing bonds created by adding ghost particles between
bonded beads) is used. For \vtsf file, \tt{atom default} bead type can be
specified via the \tt{-def} option; by default, the bead type with the most
unbonded beads is used as the \tt{atom default} type.

For examples of \tt{Info} usage, see the \tt{Examples/Info} folder.

\vspace{1em}
\noindent
Usage: \tt{Info <input> [options]}
\noindent
\begin{longtable}{p{0.19\textwidth}p{0.754\textwidth}}
  \toprule
  \multicolumn{2}{l}{Mandatory argument}\\
  \midrule
  \tt{<input>} & input structure file\\
  \midrule
  \multicolumn{2}{l}{Options for input files}\\
  \midrule
  \tt{-i <file>}    & extra input structure file (default: None)\\
  \tt{-c <file>}    & input coordinate file (default: None)\\
  \tt{--detailed}   & only \vtf structure file: detailed recognition for bead
    (and, consequently, molecule) types based on the bead names as well as
    charges, masses, and radii \\
  \midrule
  \multicolumn{2}{l}{Options for output file}\\
  \midrule
  \tt{-o <file>}    & output coordinate file (default: None)\\
  \tt{-def <bead>}  & only \vtf structure file: use bead type \tt{<bead>} for
    the \enquote{atom default} line (default:bead type with the most unbonded
    beads)\\
  \tt{--mass}       & only \data structure file: define LAMMPS atom types by
    mass but print per-atom charges in the \tt{Atoms} section (default: atom
    types are the same as \tt{Info}-recognized bead types)\\
  \tt{-ebt <int>}   & only \data structure file: number of extra atom types
                      (for, e.g., srp)\\
  \midrule
  \multicolumn{2}{l}{Other options (see the beginning of 
                     Chapter~\ref{chap:Utils})}\\
  \midrule
  \multicolumn{2}{l}{\tt{-st},
                     \tt{--verbose},
                     \tt{--silent},
                     \tt{--help},
                     \tt{--version}}\\
  \bottomrule
\end{longtable}
