\section{JoinRuns} \label{sec:JoinRuns}

{\it This utility may not work correctly.}

This utility joins two independent simulation runs of the same system.
That is, the system contains identical beads and molecules, but these beads
and molecules are numbered differently in the \texttt{vsf} and \texttt{vcf}
files from different simulations. The two input \texttt{vcf} files must
contain the same timestep type (i.e., both indexed or both ordered) and the
same number of beads (i.e., if one bead type is absent from one
\texttt{vcf} file, it must be absent from the second one as well).

The output is a \texttt{vcf} coordinate file with beads indexed according
to the first \texttt{vsf} structure file (i.e., \texttt{traject.vsf} or
provided by \texttt{-i} option).

Usage:

\vspace{1em}
\noindent
\texttt{JoinRuns <1st input> <2nd input> <2nd vsf> <output> <bead type(s)> \\ <options>}

\vspace{1em}
\noindent
\begin{longtable}{p{0.22\textwidth}p{0.724\textwidth}}
  \toprule
  \multicolumn{2}{l}{Mandatory arguments} \\
  \midrule
  \texttt{<1st input>} & input coordinate file from the first simulation (either \texttt{vcf} or
    \texttt{vtf} format) \\
  \texttt{<2nd input>} & input coordinate file from the second simulation (either \texttt{vcf} or
    \texttt{vtf} format) \\
  \texttt{<2nd vsf>} & input structure file from the second simulation
    (structure file from the first simulation is \texttt{traject.vsf};
    changeable via \texttt{-i} option) \\
  \texttt{<output>} & output \texttt{vcf} coordinate file with indexed
    coordinates \\
  \texttt{<bead type(s)>} & bead type names to save \\
  \toprule
  \multicolumn{2}{l}{Non-standard options} \\
  \midrule
  \texttt{--join} & join molecules by removing periodic boundary conditions \\
  \texttt{-st1 <int>} & starting timestep first run (default: 1) \\
  \texttt{-st2 <int>} & starting timestep second run (default: 1) \\
  \texttt{-sk1 <int>} & number of steps to skip per one used for first run (default: 0) \\
  \texttt{-sk2 <int>} & number of steps to skip per one used for second run (default: 0) \\
  \bottomrule
\end{longtable}
