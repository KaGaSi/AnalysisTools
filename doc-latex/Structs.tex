\section{Commonly used structures} \label{sec:Struct}

% Counts %{{{
\subsubsection{Basic system information}
\ttb{typedef struct Counts \{<members>\} COUNTS;}\\
\vspace{-2em}
\begin{longtable}{p{47mm}p{95mm}}
  \toprule
  member           & explanation \\
  \midrule
  \ttb{(int)TypesOfBeads}            & number of bead types  \\
  \ttb{(int)TypesOfMolecules}        & number of molecule types \\
  \ttb{(int)Beads}$^*$               & total number of beads in a coordinate file \\
  \ttb{(int)Bonded}$^*$              & total number of beads in all molecules \\
  \ttb{(int)Unbonded}$^*$            & total number of monomeric beads \\
  \ttb{(int)BeadsInVsf}$^*$          & total number of all beads in the system \\
  \ttb{(int)Molecules}$^*$           & total number of molecules \\
  \ttb{(int)Aggregates}              & total number of aggregates \\
  \ttb{(int)TypesOfBonds}$^\dagger$  & number of bond types \\
  \ttb{(int)TypesOfAngles}$^\dagger$ & number of bond types \\
  \bottomrule
\end{longtable}
\note{$^*$for \vtf, this number can correspond to a \vcf file, i.e., when
some beads are missing from the \vcf file, these beads do not have to be
included in this number\\
$^\dagger$if bond/angle parameters are not known (such as for a
\vsf file), this has a value of \tt{-1}}
  %}}}

% BeadType %{{{
\subsubsection{Information about bead types}
\ttb{typedef struct BeadType \{<members>\} BEADTYPE;} \\
\vspace{-2em}
\begin{longtable}{p{30mm}p{110mm}}
  \toprule
  member                & explanation \\
  \midrule
  \ttb{(char[17])Name}  & bead type name (at most 16 characters) \\
  \ttb{(int)Number}     & number of beads of the given type \\
  \ttb{(bool)Use}$^*$   & \tt{true}/\tt{false} value whether these beads
                          should be used in a calculation \\
  \ttb{(bool)Write}$^*$ & \tt{true}/\tt{false} value whether these beads
                          should be written to an output coordinate file \\
  \ttb{(double)Charge}  & electric charge of the bead type (\tt{1000} if undefined) \\
  \ttb{(double)Mass}    & mass of the bead type (\tt{0} if undefined) \\
  \ttb{(double)Radius}  & radius of the spherical bead (\tt{0} if undefined) \\
  \bottomrule
\end{longtable}
\note{$^*$\TODO think about usefulness of multiple \tt{bool}s}
\begin{itemize}
  \item array size: \tt{Counts.TypesOfBeads}
\end{itemize} %}}}

% Bead %{{{
\subsubsection{Information about individual beads}
\ttb{typedef struct Bead \{<members>\} BEAD;} \\
\vspace{-2em}
\begin{longtable}{p{43mm}p{97mm}}
  \toprule
  member                & explanation \\
  \midrule
  \ttb{(int)Type}               & bead type index corresponding to
                                  \ttb{struct BeadType} array index \\
  \ttb{(int)Molecule}           & index of a molecule the bead is in
                                  corresponding to \vsf's \tt{<resid>-1}
                                  value (or to \tt{-1} for monomeric bead) \\
  \ttb{(int)nAggregates}$^*$    & number of aggregates the bead is in \\
  \ttb{(int *)Aggregate}$^*$    & 1D array with aggregate indices \\
  \ttb{(int)Index}              & index corresponding to a \vsf file \\
  \ttb{(struct Vector)Position}$^\dagger\!\!$ & Cartesian coordinates \\
  \bottomrule
\end{longtable}
\note{$^*$\TODO think about usefullness of being in more aggregates \&
practical considerations\\
$^\dagger$ \ttb{struct Vector} contains members
\ttb{(double)x}, \ttb{(double)y}, and \ttb{(double)z}}
\begin{itemize}
  \item array size: \tt{Counts.Beads}
  \item array elements 0 to \tt{Counts.Unbonded} contain monomeric beads
  \item array elements \tt{Counts.Unbonded+1} to \tt{Counts.Beads} contain
    bonded beads
  \item \TODO add \ttb{(struct Vector)Velocity} for velocities
  \item usually accompanied by an \ttb{(int *)Index} array (with size of
    \tt{Counts.BeadsInVsf}) connecting in-code bead indices with \vsf bead
    indices, i.e.,\\
    \ttb{Bead[<in-code index>].Index = <vsf index>} and\\
    \ttb{Index[<vsf index>] = <in-code index>}
\end{itemize} %}}}

% MoleculeType %{{{
\subsubsection{Information about molecule types}
\ttb{typedef struct MoleculeType \{<members>\} MOLECULETYPE;} \\
\vspace{-2em}
\begin{longtable}{p{26mm}p{114mm}}
  \toprule
  member             & explanation \\
  \midrule
  \ttb{(char[17])Name}$^*$       & bead type name (at most 16 characters) \\
  \ttb{(int)Number}              & number of molecules of the given type \\
  \ttb{(int)nBeads}              & number of beads in these molecules \\
  \ttb{(int *)Bead}              & 1D array with bead in-code indices (i.e.,
                                   corresponding to the \ttb{struct Bead}
                                   array indices) \\
  \ttb{(int)nBonds}              & number of bonds in these molecules \\
  \ttb{(int **)Bond}             & 2D array with two bead indices of the
                                   connected beads and a bond type  \\
                                 & size of \ttb{nBonds}\tt{$\times$3}:
                                   \ttb{Bond[i][0]} and 1 hold in-code bead
                                   indices for bond $i$ and \ttb{Bond[i][2]}
                                   holds bond type (or \tt{-1} if bond type
                                   undefined) \\
  \ttb{(int)nAngles}             &  number of angles in these molecules \\
  \ttb{(int **)Angle}            &  2D array with three bead indices of the
                                    beads in the angle and an angle type  \\
                                 &  size of \ttb{nAngles}\tt{$\times$4}:
                                    \ttb{Angle[i][0]}, 1, and 2 hold in-code bead
                                    indices and \ttb{Angle[i][3]} holds angle
                                    type (or \tt{-1} if angle type undefined) \\
  \ttb{(int)nBTypes}             & number of bead types in these molecules \\
  \ttb{(int *)BType}             & 1D array with bead type indices (i.e.,
                                   corresponding to a \ttb{struct
                                   BeadType} array index) \\
  \ttb{(bool)InVcf}$^\dagger$    & \tt{true}/\tt{false} value whether these
                                   molecules are present in the \vcf file \\
  \ttb{(bool)Use}$^\dagger$      & \tt{true}/\tt{false} value whether these
                                   molecules should be used in a calculation \\
  \ttb{(bool)Write}$^\dagger$    & \tt{true}/\tt{false} value whether these molecules
                                   should be written to an output coordinate file \\
  \ttb{(double)Charge}$^\ddag$   & total electric charge of the molecule type \\
  \ttb{(double)Mass}$^\ddag$     & total mass of the molecule type \\
  \bottomrule
\end{longtable}
\note{$^*$\TODO change to \tt{char(9)} as per \vtf specification\\
$^\dagger$\TODO think about usefulness of multiple \tt{bool}s \\
$^\ddag$ undefined if any of the included bead types have undefined
charge/mass}
\begin{itemize}
  \item array size: \tt{Counts.TypesOfMolecules}
\end{itemize} %}}}

% Molecule %{{{
\subsubsection{Information about individual molecules}
\ttb{typedef struct Molecule \{<members>\} MOLECULE;} \\
\vspace{-2em}
\begin{longtable}{p{26mm}p{114mm}}
  \toprule
  member               & explanation \\
  \midrule
  \ttb{(int)Type}      & index of molecule type corresponding to a
                         \ttb{struct MoleculeType} array index \\
  \ttb{(int *)Bead}    & 1D array with in-code bead indices (i.e.,
                         corresponding to \ttb{struct Bead} array indices) \\
                       & size: \ttb{(int)nBeads} member of a \ttb{struct
                         MoleculeType} array element \\
  \ttb{(int)Aggregate} & index of an aggregate the molecule is in
                         corresponding to a \ttb{struct Aggregate} array
                         index \\
  \bottomrule
\end{longtable}
\begin{itemize}
  \item array size: \tt{Counts.Molecules}
  \item array indices correspond to \tt{<resid>-1} values in a \vsf file
    (because in \vsf, residue numbering starts at 1)
\end{itemize} %}}}

% Aggregate %{{{
\subsubsection{Information about individual aggregates}
\ttb{typedef struct Aggregate \{<members>\} AGGREGATE;} \\
\vspace{-2em}
\begin{longtable}{p{35mm}p{105mm}}
  \toprule
  member             & explanation \\
  \midrule
  \ttb{(int)nMolecules} & number of molecules in an aggregate \\
  \ttb{(int *)Molecule} & 1D array with molecule indices (i.e.,
                          corresponding to \ttb{struct Molecule} array
                          indices and to \tt{<resid>-1} values in a \vsf
                          file) \\
  \ttb{(int)nBeads}     & number of bonded beads in an aggregate \\
  \ttb{(int *)Bead}     & 1D array with in-code bead indices (i.e.,
                          corresponding to \ttb{struct Bead} array indices)
                          of bonded beads in an aggregate \\
  \ttb{(int)nMonomers}  & number of monomeric beads in an aggregate \\
  \ttb{(int *)Monomer}  & 1D array with in-code bead indices (i.e.,
                          corresponding to \ttb{struct Bead} array indices)
                          of monomeric beads in an aggregate \\
  \ttb{(double)Mass}    & total mass of an aggregate (undefined if any of
                          the molecules have undefined mass) \\
  \ttb{(bool)Use}$^*$   & \tt{true}/\tt{false} value whether this aggregate
                          should be used in a calculation \\
  \bottomrule
\end{longtable}
\begin{itemize}
  \item array size: \tt{Counts.Aggregates}
\end{itemize} %}}}
