\section{TransformVsf} \label{sec:TransformVsf}

This utility reads \texttt{FIELD} and \texttt{vsf} files to create a new
structure file. Generally, this is useful only if \texttt{traject-v2\_5} or
\texttt{traject-v2\_6} were used to generate the \texttt{dl\_meso.vsf}
file. This file does not contain mass and charge of particles, whereas the
one generated with \texttt{TransformVsf} (or with \texttt{traject.exe} from
the dl\_meso version 2.7) does and therefore the original \texttt{FIELD}
file is not needed for further analysis.

Usage (\texttt{TransformVsf} does not use standard options):

\vspace{1em}
\noindent
\texttt{TransformVsf <output.vsf> <options>}

\vspace{1em}
\noindent
\begin{longtable}{p{0.22\textwidth}p{0.724\textwidth}}
  \toprule
  \multicolumn{2}{l}{Mandatory argument} \\
  \midrule
  \texttt{<output.vsf>} & output structure file \\
  \multicolumn{2}{l}{Options} \\
  \texttt{-i <name>} & use custom structure file instead of
    \texttt{traject.vsf} (\texttt{vsf} or \texttt{vtf} format) \\
  \texttt{-v} & verbose output \\
  \texttt{-h} & print this help and exit \\
  \bottomrule
\end{longtable}
