\section{Surface} \label{sec:Surface}

Surface utility determines the first bead (either from the box's centre or
edges) in each square prism defined by the given <width> parameter, thus
defining a surface of, e.g., polymer brush or lipid bilayer. The <width>
slices the box into square prisms along the chosen axis (i.e., if z is the
chosen axis, the xy plane is chopped into squares, creating
<width>*<width>*<box length in z> prisms). In each such prism, two beads
are found corresponding to the two surfaces (e.g., polymer brush on both
box edges or the two surfaces of a lipid bilayer inside the box). The
output file contains 3D data in the format expected by the gnuplot program.

Usage:

\vspace{1em}
\noindent
\texttt{Surface <input> <width> <output> <axis> <options>}

\noindent
\begin{longtable}{p{0.24\textwidth}p{0.704\textwidth}}
  \toprule
  \multicolumn{2}{l}{Mandatory arguments} \\
  \midrule
  \texttt{<input>} & input coordinate file (either \texttt{vcf} or
    \texttt{vtf} format) \\
  \texttt{<width>} & side length of each square \\
  \texttt{<output>} & output file \\
  \texttt{<axis>} & direction in which to determine the surface: \texttt{x},
    \texttt{y}, or \texttt{z} \\
  \toprule
  \multicolumn{2}{l}{Non-standard options} \\
  \midrule
  \texttt{-in} & start from the box centre instead of from its edges \\
  \texttt{-mol <name(s)>} & molecule type(s) to use (default: all) \\
  \texttt{-bt <name(s)>} & bead type(s) to use (default: all) \\
  \texttt{-st <int>} & starting timestep for calculation (default: 1) \\
  \bottomrule
\end{longtable}

\noindent
Format of output files:
\begin{enumerate}[nosep,leftmargin=20pt]
  \item \texttt{<output>} -- bead densities
    \begin{itemize}[nosep,leftmargin=5pt]
      \item first line: command used to generate the file
      \item second line: column headers
        \begin{itemize}[nosep,leftmargin=5pt]
          \item first is the centre of each bin (governed by
            \texttt{<width>}); i.e., if \texttt{<width>} is 0.1,
            then the centre of bin 0 to 0.1 is 0.05, centre of bin 0.1 to
            0.2 is 0.15, etc.
          \item the rest are for the calculated data: each number
            corresponds to the density of the specified bead type
        \end{itemize}
    \end{itemize}
\end{enumerate}
