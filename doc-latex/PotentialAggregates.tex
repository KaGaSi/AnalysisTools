\section{PotentialAggregates} \label{sec:PotentialAggregates}

{\em This utility should be working, but it needs more testing.}

\texttt{PotentialAggregates} calculates electrostatic potential as a
function of distance from the centre of mass of specified aggregate
size(s). It places a virtual particle with charge $q=1$ at several places
on the surface of an ever increasing sphere and calculates electrostatic
potential acting on that virtual particle.

At long range, the potential is calculated using Coulomb potential,
\begin{equation} \label{eq:Coulomb}
  U_{ij}^{\mathrm{long}} = \frac{l_{\mathrm{B}} q_i q_j}{r_{ij}},
\end{equation}
where $l_{\mathrm{B}}$ is the Bjerrum length, $q_i$ and $q_j$ are charges
of particles $i$ and $j$, and $r_{ij}$ is interparticle distance. At short
range, the potential is for now calculated using potential between two
charges with exponentially decreasing charge density,
\begin{equation} \label{eq:Slater}
  U_{ij}^{\mathrm{short}} = U_{ij}^{\mathrm{long}}\left[1 - \left(1 + \beta
  r_{ij}\right) \mathrm{e}^{-2\beta r_{ij}}\right],
\end{equation}
where $\beta=\frac{5r_{\mathrm{c}}}{8\lambda}$ ($r_{\mathrm{c}}$ is cut off
distance and $\lambda$ is smearing constant). The utility takes into
account periodic images of the simulation box.

For now, parameters for the potential are hard coded in the source code:
the Bjerrum length is \texttt{bjerrum=1.1} (aqueous solution), cut-off is
\texttt{r\_c=3}, charge smearing constant \texttt{lambda=0.2}, and number
of periodic images of the simulation box is \texttt{images=5}. The
parameters can be changed but the utility must then be recompiled.

The aggregate size can be modified using \texttt{-m} option similarly to
\texttt{Density}-\texttt{Aggregates}
(Section~\ref{sec:DensityAggregates}).

Be aware that the utility does not use any Ewald sum, so the calculation is
extremely slow (depending on the number of charged particles in the system).

Usage:

\vspace{1em}
\noindent
\texttt{PotentialAggregates <input> <input.agg> <width> <output> <agg size(s)> <options>}

\noindent
\begin{longtable}{p{0.24\textwidth}p{0.704\textwidth}}
  \toprule
  \multicolumn{2}{l}{Mandatory arguments} \\
  \midrule
  \texttt{<input>} & input coordinate file (either \texttt{vcf} or
    \texttt{vtf} format) \\
  \texttt{<input.agg>} & input \texttt{agg} file \\
  \texttt{<width>} & width of each bin of the distribution \\
  \texttt{<output>} & output file with electrostatic potential \\
  \texttt{<agg size(s)>} & aggregate size(s) for calculation of
  electrostatic potential \\
  \toprule
  \multicolumn{2}{l}{Non-standard options} \\
  \midrule
  \texttt{--joined} & specify that \texttt{<input>} contains joined
    coordinates (i.e., periodic boundary conditions for aggregates do not
    have to be removed) \\
  \texttt{-st <int>} & starting timestep for calculation (default: 1) \\
  \texttt{-m <mol name(s)>} & instead of `true' aggregate size, use the number
    of specified molecule type(s) in an aggregate \\
  \bottomrule
\end{longtable}
