\section{DensityBox} \label{sec:DensityBox}

This utility calculates number density for all bead types in the specified
direction of the simulation box. The density is calculated from 0 to box
length in the given direction, that is, the box is \enquote{sliced} into
blocks with width \texttt{<width>} and numbers of different bead types are
counted in each of the slices.

The utility does not distinguish between beads with the same name in
different molecules, so if one bead type is in more than one molecule type,
it's density will be averaged over all molecule types it appears in. In
this case, however, the \texttt{-x} option can be used to first run the
utility and exclude one molecule type and then rerun it, excluding the
other molecule type. Thus, two output files are generated, each missing
densities from the specified molecule types.

Usage:

\vspace{1em}
\noindent
\texttt{DensityBox <input> <width> <output> <axis> <options>}

\noindent
\begin{longtable}{p{0.24\textwidth}p{0.704\textwidth}}
  \toprule
  \multicolumn{2}{l}{Mandatory arguments} \\
  \midrule
  \texttt{<input>} & input coordinate file (either \texttt{vcf} or
    \texttt{vtf} format) \\
  \texttt{<width>} & width of each bin of the distribution \\
  \texttt{<output>} & output file with automatic \texttt{<axis>.rho} ending \\
  \texttt{<axis>} & direction in which to calculate density: \texttt{x},
    \texttt{y}, or \texttt{z} \\
  \toprule
  \multicolumn{2}{l}{Non-standard options} \\
  \midrule
  \texttt{-n <int>} & number of bins to average to get smoother density
    (default: 1) \\
  \texttt{-st <int>} & starting timestep for calculation (default: 1) \\
  \texttt{-x <mol name(s)>} & exclude specified molecule type(s) (i.e., do
    not calculate density for beads in molecules \texttt{<mol name(s)>}) \\
  \bottomrule
\end{longtable}

\noindent
Format of output files:
\begin{enumerate}[nosep,leftmargin=20pt]
  \item \texttt{<output>} -- bead densities
    \begin{itemize}[nosep,leftmargin=5pt]
      \item first line: command used to generate the file
      \item second line: column headers
        \begin{itemize}[nosep,leftmargin=5pt]
          \item first is the centre of each bin (governed by
            \texttt{<width>}); i.e., if \texttt{<width>} is 0.1,
            then the centre of bin 0 to 0.1 is 0.05, centre of bin 0.1 to
            0.2 is 0.15, etc.
          \item the rest are for the calculated data: each number
            corresponds to the density of the specified bead type
        \end{itemize}
    \end{itemize}
\end{enumerate}
