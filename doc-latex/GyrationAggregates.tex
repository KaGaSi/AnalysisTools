\section{GyrationAggregates} \label{sec:GyrationAggregates}

This utility calculates the gyration tensor and its eigenvalues (or the
roots of the tensor's characteristic polynomial) for all aggregates. Using
the eigenvalues, the utility determines shape descriptors:
radius of gyration, asphericity, acylindricity, and relative shape
anisotropy.

The eigenvalues, $\lambda_x^2$, $\lambda_y^2$, and $\lambda_z^2$, (sorted
so that $\lambda_x^2\leq\lambda_y^2\leq\lambda_z^2$) are also written to
output file(s), because their square roots represent half-axes of an
equivalent ellipsoid.

The radius of gyration, $R_{\mathrm{G}}$, is defined as
\begin{equation} \label{eq:R_G}
  R_{\mathrm{G}}^2 = \lambda_x^2 + \lambda_y^2 + \lambda_z^2.
\end{equation}
The asphericity, $b$, and the acylindricity, $c$, are defined,
respectively, as
\begin{equation} \label{eq:b}
  b= \lambda_z^2 - \frac{1}{2}(\lambda_x^2 + \lambda_y^2) =
    \frac{3}{2}\lambda_z^2 - \frac{R_{\mathrm{G}}^2}{2}
  \mbox{ \ \ and \ \ }
  c = \lambda_y^2 - \lambda_x^2.
\end{equation}
The relative shape anisotropy is defined in terms of the other descriptors
as
\begin{equation} \label{eq:anis}
  \kappa^2 = \frac{b^2 + 0.75 c^2}{R_{\mathrm{G}}^4}
\end{equation}

Number average of all properties and, additionally, weight and z averages
for the radius of gyration are calculated (see Equation~\eqref{eq:Avg} in
Section~\ref{sec:DistrAgg} for general definitions of averages).
Per-timestep averages (i.e., time evolution) are written to the
\texttt{<output>} file. Per-size averages can be saved using the
\texttt{-ps} option.

The gyration tensor is by default calculated for whole aggregates, but
\texttt{-bt} option can be used to specify which bead types to consider.

Similarly to \texttt{DistrAgg}, the definition of aggregate size is
flexible -- see Section~\ref{sec:DistrAgg} for explanations of the
\texttt{-m} and \texttt{-x} options.

The starting step (\texttt{-st} option) and ending step (\texttt{-e}
option) are used only for averages (both overall averages and per-size
averages with \texttt{-ps} option). Per-timestep averages in the
\texttt{<output>} file disregard \texttt{-st} and \texttt{-e} options.

Usage:

\vspace{1em}
\noindent
\texttt{GyrationAggregates <input> <input.agg> <output> <options>}

\noindent
\begin{longtable}{p{0.265\textwidth}p{0.679\textwidth}}
  \toprule
  \multicolumn{2}{l}{Mandatory arguments} \\
  \midrule
  \texttt{<input>} & input coordinate file (either \texttt{vcf} or
    \texttt{vtf} format) \\
  \texttt{<input.agg>} & input \texttt{agg} file \\
  \texttt{<output>} & output file with per-timestep averages \\
  \toprule
  \multicolumn{2}{l}{Non-standard options} \\
  \midrule
  \texttt{--joined} & specify that \texttt{<input>} contains joined
    coordinates (i.e., periodic boundary conditions for aggregates do not
    have to be removed) \\
  \texttt{-bt <bead name(s)>} & bead type(s) used for calculation \\
  \texttt{-ps <name>} & output file with per-size averages \\
  \texttt{-m <mol name(s)>} & instead of `true' aggregate size, use the number
    of specified molecule type(s) in an aggregate \\
  \texttt{-x <mol name(s)>} & exclude aggregates containing only specified
    molecule type(s) \\
  \texttt{-n <int> <int>} & use only aggregate sizes in given range \\
  \texttt{-st <int>} & starting timestep for calculation (default: 1) \\
  \texttt{-e <int>} & ending timestep for calculation (default: none) \\
  \bottomrule
\end{longtable}

\begin{enumerate}[nosep,leftmargin=20pt]
  \item \texttt{<output>} -- per-timestep averages
    \begin{itemize}[nosep,leftmargin=5pt]
      \item first line: command used to generate the file
      \item second line: column headers
        \begin{itemize}[nosep,leftmargin=10pt]
          \item first is timestep
          \item rest are for calculated data: number, weight, and z
            averages (denoted by \texttt{\_n}, \texttt{\_w}, and
            \texttt{\_z} respectively) of radius of gyration and square of
            radius of gyration (\texttt{Rg} and \texttt{Rg\^{}2},
            respectively -- Equation~\eqref{eq:R_G}); number averages of
            relative shape anisotropy (\texttt{Anis} --
            Equation~\eqref{eq:anis}), acylindricity, and asphericity
            (\texttt{Acyl} and \texttt{Aspher}, respectively --
            Equation~\eqref{eq:b}), and all three eigenvalues
            (\texttt{eigen.x}, \texttt{eigen.y}, and \texttt{eigen.z} --
            $\lambda_x^2$, $\lambda_y^2$, and $\lambda_z^2$)
        \end{itemize}
      \item second to last line: column headers for overall averages
        written in the last line
        \begin{itemize}[nosep,leftmargin=10pt]
          \item \texttt{<M>\_n} and \texttt{<M>\_w} are
            number and weight, respectively, averages of aggregate
            masses (the averages are defined in Equation~\eqref{eq:Avg});
            aggregate mass here is the mass of all beads of the chosen
            type(s) in the aggregate
          \item average numbers of molecules of each type in an aggregate
            are shown in columns denoted \texttt{<mol names>}
          \item the remaining column represent overall averages of the
            per-timestep quantities described above
        \end{itemize}
    \end{itemize}
  \item \texttt{-ps <name>} -- per-size averages
  \begin{itemize}[nosep,leftmargin=5pt]
    \item first line: command used to generate the file
    \item second line: column headers
      \begin{itemize}[nosep,leftmargin=10pt]
        \item first is aggregate size
        \item last column is the total number of aggregates of the given size
        \item the columns in between are for the calculated data: simple
          averages of the numbers of molecules of each type in the
          aggregate, of radius of gyration and its square, of relative
          shape anisotropy, acylindricity, and asphericity, and of all
          three eigenvalues (all are denoted by the above-described symbols)
      \end{itemize}
  \end{itemize}
\end{enumerate}
