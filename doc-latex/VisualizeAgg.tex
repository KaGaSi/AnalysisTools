\section{VisualizeAgg} \label{sec:VisualizeAgg}

This utility prints aggregates of given size to output \texttt{vcf} files
(one file per aggregate size). The output files contain one aggregate per
timestep regardless of how many aggregates of the given size are in the
initial coordinate file.

The definition of aggregate size is fairly flexible (similarly to
\texttt{DensityAggregates} utility -- Section~\ref{sec:DensityAggregates}).

As the output file(s) do not necessarily contain all beads of a bead
type, these \texttt{vcf} files cannot be used for further analysis via
\texttt{AnalysisTools}.

Usage:

\vspace{1em}
\noindent
\texttt{VisualizeAgg <input> <input.agg> <output> <agg \\
size(s)> <options>}

\noindent
\begin{longtable}{p{0.24\textwidth}p{0.704\textwidth}}
  \toprule
  \multicolumn{2}{l}{Mandatory arguments} \\
  \midrule
  \texttt{<input>} & input coordinate file (either \texttt{vcf} or
    \texttt{vtf} format) \\
  \texttt{<input.agg>} & input \texttt{.agg} file \\
  \texttt{<output>} & output file(s) (one per aggregate size) with
    automatic \texttt{\#.vcf} ending (\texttt{\#} is aggregate size) \\
  \texttt{<agg size(s)>} & aggregate size(s) to save \\
  \toprule
  \multicolumn{2}{l}{Non-standard options} \\
  \midrule
  \texttt{--joined} & specify that \texttt{<input>} contains joined
    coordinates (i.e., periodic boundary conditions for aggregates do not
    have to be removed) \\
  \texttt{-st <int>} & starting timestep for calculation (default: 1) \\
  \texttt{-e <int>} & ending timestep for calculation (default: none) \\
  \texttt{-m <mol name(s)>} & instead of \enquote{true} aggregate size, use
    the number of specified molecule type(s) in an aggregate \\
  \texttt{-x <mol name(s)>} & exclude aggregates containing only specified
    molecule type(s) \\
  \bottomrule
\end{longtable}
