\section{PairCorrel} \label{sec:PairCorrel}

This utility calculates pair correlation function (pcf) between specified
bead types. All bead type pairs are used -- if \texttt{A} and \texttt{B}
bead types, \texttt{A-A}, \texttt{A-B}, and \texttt{B-B} bead type pairs
are used. Right now, the pcfs are not correctly normalised.

The utility do not recognise between beads of the same type that are in
different molecules, so a pcf will be a sum of the beads from different
molecule types.

Usage:

\vspace{1em}
\noindent
\texttt{PairCorrel <input> <width> <output> <bead name(s)> <options>}

\noindent
\begin{longtable}{p{0.205\textwidth}p{0.739\textwidth}}
  \toprule
  \multicolumn{2}{l}{Mandatory arguments} \\
  \midrule
  \texttt{<input>} & input coordinate file (either \texttt{vcf} or
    \texttt{vtf} format) \\
  \texttt{<width>} & width of each bin of the pair correlation functions \\
  \texttt{<output>} & output file with pair correlation functions \\
  \texttt{<bead name(s)>} & bead type(s) used for calculation \\
  \toprule
  \multicolumn{2}{l}{Non-standard options} \\
  \midrule
  \texttt{-n <int>} & number of bins to average to get smoother pair
    correlation function (default: 1) \\
  \texttt{-st <int>} & starting timestep for calculation (default: 1) \\
  \bottomrule
\end{longtable}

