\section{PairCorrel} \label{sec:PairCorrel}

This utility calculates pair correlation function (pcf), or the radial
distribution function, between specified bead types. All bead type pairs
are used -- if \texttt{A} and \texttt{B} are the bead types, \texttt{A-A},
\texttt{A-B}, and \texttt{B-B} bead pairs are used.

The utility do not differentiate beads of the same type that are in
different molecules, so a pcf will be a sum of the beads from different
molecule types (similar the \texttt{DensityAggregates} utility -- see
Section \ref{sec:DensityAggregates}).

Usage:

\vspace{1em}
\noindent
\texttt{PairCorrel <input> <width> <output> <bead name(s)> <options>}

\noindent
\begin{longtable}{p{0.205\textwidth}p{0.739\textwidth}}
  \toprule
  \multicolumn{2}{l}{Mandatory arguments} \\
  \midrule
  \texttt{<input>} & input coordinate file (either \texttt{vcf} or
    \texttt{vtf} format) \\
  \texttt{<width>} & width of each bin of the pair correlation functions \\
  \texttt{<output>} & output file with pair correlation functions \\
  \texttt{<bead name(s)>} & bead type(s) used for calculation \\
  \toprule
  \multicolumn{2}{l}{Non-standard options} \\
  \midrule
  \texttt{-n <int>} & number of bins to average to get smoother pair
    correlation function (default: 1) \\
  \texttt{-st <int>} & starting timestep for calculation (default: 1) \\
  \texttt{-e <int>} & ending timestep for calculation (default: none) \\
  \bottomrule
\end{longtable}

\noindent
Format of output files:
\begin{enumerate}[nosep,leftmargin=20pt]
  \item \texttt{<output>} -- pair correlation functions between all bead types
    \begin{itemize}[nosep,leftmargin=5pt]
      \item first line: command used to generate the file
      \item second line: column headers
        \begin{itemize}[nosep,leftmargin=5pt]
          \item first is the centre of each bin (governed by
            \texttt{<width>}); i.e., if \texttt{<width>} is 0.1,
            then the centre of bin 0 to 0.1 is 0.05, centre of bin 0.1 to
            0.2 is 0.15, etc.
          \item the rest are for the calculated data: each column
            corresponds to one bead types pair
        \end{itemize}
    \end{itemize}
\end{enumerate}
