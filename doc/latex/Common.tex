Utilies that are not specific to any given system and are used for all simulations.\hypertarget{Common_traject}{}\subsection{traject utility}\label{Common_traject}
This utility is from the \href{http://www.scd.stfc.ac.uk//research/app/ccg/software/DL_MESO/40694.aspx}{\tt D\+L\+\_\+\+M\+E\+SO simulation package}. While originally it creates a {\ttfamily .vtf} file containing both structure and coordinates, I have changed it to create a separate {\ttfamily dl\+\_\+meso.\+vsf} structure file and {\ttfamily All.\+vcf} coordinate file containing ordered timesteps.

Usage\+:

{\ttfamily traject $<$cores$>$}

\begin{quote}
{\ttfamily $<$cores$>$} \begin{quote}
number of computer cores used for the simulation run (or the number of {\ttfamily H\+I\+S\+T\+O\+RY} file) \end{quote}
\end{quote}


The standard options cannot be used with this utility.\hypertarget{Common_SelectedVcf}{}\subsection{Selected\+Vcf utility}\label{Common_SelectedVcf}
This utility takes {\ttfamily .vcf} file containing either \hyperlink{InputFiles_OrderedCoorFile}{ordered timesteps} (such as {\ttfamily All.\+vcf} created by D\+L\+\_\+\+M\+E\+SO {\ttfamily traject} utility which was modified by me) or \hyperlink{InputFiles_IndexedCoorFile}{indexed timesteps} and creates a new {\ttfamily .vcf} coordinate file containing only beads of selected types with an option of removing periodic boundary condition and thus joining molecules. The otput {\ttfamily .vcf} file therefore contains i ndexed timesteps.

Specified molecules can be excluded which is useful when the same bead type is shared between more molecule types.

Usage\+:

{\ttfamily Selected\+Vcf $<$input.\+vcf$>$ $<$start$>$ $<$skip$>$ $<$output.\+vcf$>$ $<$type names$>$ $<$options$>$}

\begin{quote}
{\ttfamily $<$input.\+vcf$>$} \begin{quote}
input coordinate filename (must end with {\ttfamily .vcf}) containing either ordered or indexed timesteps \end{quote}
{\ttfamily $<$start$>$} \begin{quote}
number of timestep to start from \end{quote}
{\ttfamily $<$skip$>$} \begin{quote}
leave out every {\ttfamily skip} steps \end{quote}
{\ttfamily $<$output.\+vcf$>$} \begin{quote}
output filename with indexed coordinates (must end with {\ttfamily .vcf}) \end{quote}
{\ttfamily $<$type names$>$} \begin{quote}
names of bead types to save \end{quote}
{\ttfamily $<$options$>$} \begin{quote}
{\ttfamily -\/j} \begin{quote}
join individual molecules by removing periodic boundary conditions \end{quote}
{\ttfamily -\/st $<$int$>$} \begin{quote}
starting timestep for calculation \end{quote}
{\ttfamily -\/sk $<$int$>$} \begin{quote}
number of steps to skip per one used \end{quote}
{\ttfamily -\/x $<$name(s)$>$} \begin{quote}
exclude specified molecule(s) \end{quote}
\end{quote}
\end{quote}
\hypertarget{Common_Config}{}\subsection{Config utility}\label{Common_Config}
This utility takes {\ttfamily .vcf} file containing either \hyperlink{InputFiles_OrderedCoorFile}{ordered timesteps} (such as {\ttfamily All.\+vcf} created by D\+L\+\_\+\+M\+E\+SO {\ttfamily traject} utility which was modified by me) or \hyperlink{InputFiles_IndexedCoorFile}{indexed timesteps} and creates {\ttfamily C\+O\+N\+F\+IG} file (file containing initial coordinates for a simulation via \href{http://www.scd.stfc.ac.uk//research/app/ccg/software/DL_MESO/40694.aspx}{\tt D\+L\+\_\+\+M\+E\+SO simulation package}).

Usage\+:

{\ttfamily Config $<$input.\+vcf$>$ $<$options$>$}

\begin{quote}
{\ttfamily $<$input.\+vcf$>$} \begin{quote}
input coordinate filename (must end with {\ttfamily .vcf}) containing either ordered or indexed timesteps \end{quote}
\end{quote}
\hypertarget{Common_TransformVsf}{}\subsection{Transform\+Vsf utility}\label{Common_TransformVsf}
This utility takes {\ttfamily .vsf} structure file and D\+L\+\_\+\+M\+E\+SO input file {\ttfamily F\+I\+E\+LD} and transforms them into a different {\ttfamily .vsf} structure file that is well suited for visualisation using V\+MD software.

Usage\+:

{\ttfamily Transform\+Vsf $<$output.\+vsf$>$ $<$options$>$}

\begin{quote}
{\ttfamily $<$output.\+vsf$>$} \begin{quote}
output structure file that must end with {\ttfamily .vsf} \end{quote}
\end{quote}
\hypertarget{Common_TransformVsf-output}{}\subsubsection{Format of output structure file}\label{Common_TransformVsf-output}
Every atom line in the generated structure file contains bead\textquotesingle{}s index number, mass, charge and name. Atom lines for beads in molecules also contain molecule\textquotesingle{}s id number and the name of the type of molecule. The bond section of {\ttfamily output.\+vsf} lists all bonds one by one (i.\+e. no chains of bonds in the format {\ttfamily $<$id1$>$\+:\+: $<$id2$>$} are used). Information about which bonds belong to which molecule is provided as a comment. The file has the following format\+: \begin{quote}
{\ttfamily atom default name $<$name$>$ mass $<$m$>$ charge $<$q$>$}

{\ttfamily ...}

{\ttfamily atom $<$id$>$ name $<$name$>$ mass $<$m$>$ charge $<$q$>$}

{\ttfamily ...}

{\ttfamily atom $<$id$>$ name $<$name$>$ mass $<$m$>$ charge $<$q$>$ segid $<$name$>$ resid $<$id$>$}

{\ttfamily ...}

{\ttfamily \# resid $<$id$>$}

{\ttfamily $<$bonded bead id1$>$\+: $<$bonded bead id2$>$}

{\ttfamily ...} \end{quote}


For V\+MD atom selection\+: \begin{quote}
{\ttfamily segid $<$name$>$} \begin{quote}
selects all molecules with given name(s) \end{quote}
{\ttfamily resid $<$id$>$} \begin{quote}
selects molecule(s) with given index number(s) \end{quote}
{\ttfamily charge $<$q$>$} \begin{quote}
selects all beads with given charge(s) (double quotes are required for negative charge) \end{quote}
{\ttfamily mass $<$m$>$} \begin{quote}
selects all beads with given mass(es) \end{quote}
\end{quote}


\begin{DoxyRefDesc}{Todo}
\item[\hyperlink{todo__todo000001}{Todo}]Optional bond file format\+: somehow avoid the need to use the special optional bond file, where the ids of beads must strictly adhere to F\+I\+E\+LD. Possibly require use of a vcf file in conjunction with bond file 

Join\+Runs\+: implement wholy {\ttfamily -\/-\/script} common option 

Completely change this -\/ either implement {\ttfamily -\/x} option or remove function {\ttfamily Write\+Coor\+Indexed} and hard code the writing to file\end{DoxyRefDesc}
\hypertarget{Common_BondLength}{}\subsection{Bond\+Length utility}\label{Common_BondLength}
Bond\+Length utility calculates normalized distribution of bond length for specified molecule types.

Usage\+:

{\ttfamily Bond\+Length $<$input.\+vcf$>$ $<$output file$>$ $<$width$>$ $<$molecule names$>$ $<$options$>$}

\begin{quote}
{\ttfamily $<$input.\+vcf$>$} \begin{quote}
input coordinate filename (must end with {\ttfamily .vcf}) containing either ordered or indexed timesteps \end{quote}
{\ttfamily $<$output file$>$} \begin{quote}
output filename containing distribution of bond lengths \end{quote}
{\ttfamily $<$width$>$} \begin{quote}
width of each bin for the distribution \end{quote}
{\ttfamily $<$molecule names$>$} \begin{quote}
names of molecule types to calculate the distribution for \end{quote}
\end{quote}
\hypertarget{Common_Aggregates}{}\subsection{Aggregates \& Aggregates-\/\+Not\+Same\+Beads utility}\label{Common_Aggregates}
These utilities determine which molecules belong to which aggregates according to a simple criterion\+: two molecules belong to the same aggregate if they share at least a specified number of contact pairs. A contact pair is a pair of two beads belonging to different molecules which are closer than certain distance. Both the distance and the number of needed contact pairs are arguments of the command as well as bead types to consider. Specified molecule(s) can be excluded from aggregate calculation (both from aggregate calculation and the output {\ttfamily .agg} file).

While the Aggregates utility uses all possible pairs of given bead types, Aggregates-\/\+Not\+Same\+Beads does not use same-\/type pairs. That is, if bead types {\ttfamily A}, {\ttfamily B} and {\ttfamily C} are given, Aggregates utility will use all six bead type pairs, that is {\ttfamily A-\/A}, {\ttfamily A-\/B}, {\ttfamily A-\/C}, {\ttfamily B-\/B}, {\ttfamily B-\/C} and {\ttfamily C-\/C} (provided the beads are in different molecules), but Aggregates-\/\+Not\+Same\+Beads will not use {\ttfamily A-\/A}, {\ttfamily B-\/B} or {\ttfamily C-\/C} contacts. Therefore at least two bead types must be provided for {\ttfamily $<$type names$>$} argument in Aggregates-\/\+Not\+Same\+Beads.

Usage\+:

{\ttfamily Aggregates $<$input.\+vcf$>$ $<$distance$>$ $<$contacts$>$ $<$output.\+agg$>$ $<$type names$>$ $<$options$>$}

\begin{quote}
{\ttfamily $<$input.\+vcf$>$} \begin{quote}
input coordinate filename (must end with {\ttfamily .vcf}) containing either ordered or indexed timesteps \end{quote}
{\ttfamily $<$distance$>$} \begin{quote}
minimum distance for two beads to be in contact (constituting one contact pair) \end{quote}
{\ttfamily $<$contacts$>$} \begin{quote}
minimum number of contact pairs to consider two molecules to be in one aggregate \end{quote}
{\ttfamily $<$output.\+agg$>$} \begin{quote}
output filename (must end with {\ttfamily .agg}) containing information about aggregates \end{quote}
{\ttfamily $<$type names$>$} \begin{quote}
names of bead types to use for calculating contact pairs \end{quote}
{\ttfamily $<$options$>$} \begin{quote}
{\ttfamily -\/x $<$name(s)$>$} \begin{quote}
exclude specified molecule(s) from calculation of aggregates \end{quote}
{\ttfamily -\/j $<$joined.\+vcf$>$} \begin{quote}
filename for coordinates of joined aggregates (must end with {\ttfamily .vcf}) \end{quote}
\end{quote}
\end{quote}
\hypertarget{Common_JoinAggregates}{}\subsection{Join\+Aggregates utility}\label{Common_JoinAggregates}
This utility reads input {\ttfamily .vcf} and {\ttfamily .agg} files and removes periodic boundary conditions from aggregates -\/ e.\+i. it joins the aggregates. The distance and the bead types for closeness check are read from the first line of {\ttfamily .agg} file with contains full Aggregates command used to generate the file. Join\+Aggregates is meant for cases, where {\ttfamily -\/j} flag was omitted in Aggregates utility.

Usage\+:

{\ttfamily Aggregates $<$input.\+vcf$>$ $<$input.\+agg$>$ $<$output.\+vcf$>$ $<$options$>$}

\begin{quote}
{\ttfamily $<$input.\+vcf$>$} \begin{quote}
input coordinate filename (must end with {\ttfamily .vcf}) containing either ordered or indexed timesteps \end{quote}
{\ttfamily $<$input.\+agg$>$} \begin{quote}
input filename (must end with {\ttfamily .agg}) containing information about aggregates \end{quote}
{\ttfamily $<$output.\+vcf$>$} \begin{quote}
output filename (must end with {\ttfamily .vcf}) with joined coordinates \end{quote}
\end{quote}
\hypertarget{Common_DistrAgg}{}\subsection{Distr\+Agg utility}\label{Common_DistrAgg}
Distr\+Agg calculates number and weight average aggregation numbers for each timestep (time evolution).  
The number average aggregation number, $\langle A_{\mathrm{s}} \rangle_n$
is defined as:

\begin{equation}
\langle A_{\mathrm{s}} \rangle_n = \frac{\sum_{i=1}^N m_i}{N} \mbox{,}
\end{equation}

where $m_i$ is weight (aggregation number) of aggregate $i$ and $N$ is
total number of aggregates. The weight average aggregation number, $\langle
A_{\mathrm{s}} \rangle_w$ is then defined as:

\begin{equation}
\langle A_{\mathrm{s}} \rangle_w = \frac{\sum_{i=1}^N m_i^2}{\sum_{i=1}^N m_i} \mbox{.}
\end{equation}


It also calculates overall number and weight distribution function.  
The number distribution function, $F_n (A_{\mathrm{s}})$ is defined as:

\begin{equation}
F_n (A_{\mathrm{s}}) = \frac{N_{A_{\mathrm{s}}}}{\sum_{i=1}^N N_i} \mbox{,}
\end{equation}

where $N_i$ is the number of aggregates with aggregation number
$A_{\mathrm{s}} = i$.  The weight distribution function, $F_w
(A_{\mathrm{s}})$ is then defined as:

\begin{equation}
F_w(A_{\mathrm{s}}) = \frac{m_{A_{\mathrm{s}} }
N_{A_{\mathrm{s}}}}{\sum_{i=1}^N
m_i N_i} \mbox{,}
\end{equation}

where $m_{A_{\mathrm{s}} }$ and $m_i$ are again the weight, that is the
aggregation number.


Lastly, the utility calculates volume fractions of all aggregates, where it (for now) assumes that all beads have reduced mass of 1.  
Volume fraction of an aggregate with aggregation number $A_{\mathrm{s}}$ is
defined as:

\begin{equation}
\phi(A_{\mathrm{s}}) = \frac{n_{A_{\mathrm{s}}} N_{A_{\mathrm{s}}}}{\sum_{i=1}^N n_i N_i} \mbox{,}
\end{equation}

where $n_i$ is volume of an aggregate with $A_{\mathrm{s}} = i$ -- that is
the number of beads in the aggregate.

It should be noted that weight average aggregation number and weight
distribution function do not take into account the actual weight of an
associates -- it is weighted via the aggregation number itself.


The utility reads information about aggregate from input file with \hyperlink{InputFiles_AggregateFile}{Aggregate format}. This file can be generated using \hyperlink{Common_Aggregates}{Aggregates utility}.

Usage\+:

{\ttfamily Distr\+Agg $<$input$>$ $<$output distr file$>$ $<$output avg file$>$ $<$options$>$}

\begin{quote}
{\ttfamily $<$input$>$} \begin{quote}
input filename with information about aggregates \end{quote}
{\ttfamily $<$output distr file$>$} \begin{quote}
output filename with weight and number distribution functions \end{quote}
{\ttfamily $<$output avg file$>$} \begin{quote}
output filename with weight and number average aggregation number in each timestep \end{quote}
{\ttfamily $<$options$>$} \begin{quote}
{\ttfamily -\/n $<$int$>$} \begin{quote}
starting timestep for calculation (does not affect calculation of time evolution) \end{quote}
{\ttfamily -\/-\/no-\/unimers} \begin{quote}
free chains shouldn\textquotesingle{}t be used to calcalute average aggregation numbers \end{quote}
\end{quote}
\end{quote}
\hypertarget{Common_AggDensity}{}\subsection{Density\+Aggregates}\label{Common_AggDensity}
This utility calculates number bead density for aggregates of specified size from their center of mass. During the calculation, only the current aggregate is taken into account, so there is no possibility of getting \textquotesingle{}false\textquotesingle{} densities from adjacent aggregates. Therefore if some bead type is never present in an aggregate of specified size (but is in the {\ttfamily .vcf} file), its density will always be 0.

Instead of true aggregate size, a number of molecules of specified name can be used, i.\+e. an aggregate with 1 {\ttfamily A} molecule and 2 {\ttfamily B} molecules can be specified with {\ttfamily $<$agg sizes$>$} of 3 without {\ttfamily -\/m} option or 1 if {\ttfamily -\/m A} is used (or 2 if {\ttfamily -\/m B} is used).

Also specified molecule type(s) can be excluded via the {\ttfamily -\/x} option. This is useful in case of several molecules sharing the same bead type. Calculated densities take into account only name of a bead type, not in which molecule(s) it occurs. The density from the bead type in different molecule types will therefore be the sum of the densities from those molecules.

Usage\+:

{\ttfamily Density\+Aggregates $<$input.\+vcf$>$ $<$input.\+agg$>$ $<$width$>$ $<$output.\+rho$>$ $<$agg sizes$>$ $<$options$>$}

\begin{quote}
{\ttfamily $<$input.\+vcf$>$} \begin{quote}
input coordinate filename (must end with {\ttfamily .vcf}) containing either ordered or indexed timesteps \end{quote}
{\ttfamily $<$input.\+agg$>$} \begin{quote}
input filename (must end with {\ttfamily .agg}) containing information about aggregates \end{quote}
{\ttfamily $<$width$>$} \begin{quote}
width of each bin for the distribution \end{quote}
{\ttfamily $<$output.\+rho$>$} \begin{quote}
output density file (automatic ending {\ttfamily agg\#.rho} added) \end{quote}
{\ttfamily $<$agg sizes$>$} \begin{quote}
aggregate sizes for density calculation \end{quote}
{\ttfamily $<$options$>$} \begin{quote}
{\ttfamily -\/j} \begin{quote}
specify that the {\ttfamily $<$input.\+vcf$>$} contains aggregates with joined coordinates \end{quote}
{\ttfamily -\/n $<$int$>$} \begin{quote}
number of bins to average \end{quote}
{\ttfamily -\/st $<$int$>$} \begin{quote}
starting timestep for calculation \end{quote}
{\ttfamily -\/m $<$molecule type name$>$} \begin{quote}
instead of aggregate size, use number of molecules of specified molecule types \end{quote}
{\ttfamily -\/x $<$name(s)$>$} \begin{quote}
exclude specified molecule(s) \end{quote}
\end{quote}
\end{quote}


\begin{DoxyRefDesc}{Todo}
\item[\hyperlink{todo__todo000002}{Todo}]Density\+Aggregates\+: check if only chains in one aggregate are used -- anomalies in Van\+Der\+Burgh/\+Added\+Pol/\end{DoxyRefDesc}
\hypertarget{Common_DensityMolecules}{}\subsection{Density\+Molecules}\label{Common_DensityMolecules}
Density\+Molecules works in similar way as the Density\+Aggregates, only instead of aggregates, the densities are calculated for specified molecule types. Care must be taken with beadtype names in various molecules types, because if one beadtype appears in more molecule types, the resulting density for that beadtype will be averaged without regard for the various types of molecule it appears in.

It is possible to use specified bead instead of the center of mass for the coordinates to calculate densities from. Care must be taken, because the order of molecule types is taken from {\ttfamily F\+I\+E\+LD} rather then from {\ttfamily Density\+Molcules} arguments. For example\+: whether bead 1 will be connected with {\ttfamily NameA} or {\ttfamily NameB} in {\ttfamily Density\+Molecules ... NameA NameB -\/c 1 2} depends on molecules\textquotesingle{} order in {\ttfamily F\+I\+E\+LD} file; that is if {\ttfamily NameA} is first in {\ttfamily F\+I\+E\+LD}, 1 will be associated with {\ttfamily NameA} and 2 with {\ttfamily NameB}, but if {\ttfamily NameB} is first, the associations are reverse, regardless of the order of names in the command\textquotesingle{}s arguments. If the center of mass should be used, {\ttfamily x} is given as argument. In the above example (assuming {\ttfamily NameA} is first in {\ttfamily F\+I\+E\+LD}) if bead 1 is intended to be used for {\ttfamily NameB}, but center of mass for {\ttfamily NameA}, then an argument of the form {\ttfamily -\/c x 1} must be used.

Usage\+:

{\ttfamily Density\+Molecules $<$input.\+vcf$>$ $<$input.\+agg$>$ $<$width$>$ $<$output.\+rho$>$ $<$agg sizes$>$ $<$options$>$}

\begin{quote}
{\ttfamily $<$input.\+vcf$>$} \begin{quote}
input coordinate filename (must end with {\ttfamily .vcf}) containing either ordered or indexed timesteps \end{quote}
{\ttfamily $<$input.\+agg$>$} \begin{quote}
input filename (must end with {\ttfamily .agg}) containing information about aggregates \end{quote}
{\ttfamily $<$width$>$} \begin{quote}
width of each bin for the distribution \end{quote}
{\ttfamily $<$output.\+rho$>$} \begin{quote}
output density file (automatic ending {\ttfamily agg\#.rho} added) \end{quote}
{\ttfamily $<$agg sizes$>$} \begin{quote}
aggregate sizes for density calculation \end{quote}
{\ttfamily $<$options$>$} \begin{quote}
{\ttfamily -\/j} \begin{quote}
specify that the {\ttfamily $<$input.\+vcf$>$} contains aggregates with joined coordinates \end{quote}
{\ttfamily -\/n $<$average$>$} \begin{quote}
number of bins to average \end{quote}
{\ttfamily -\/c $<$int$>$} \begin{quote}
use specified molecule bead instead of center of mass \end{quote}
\end{quote}
\end{quote}
\hypertarget{Common_GyrationAggregates}{}\subsection{Gyration\+Aggregates utility}\label{Common_GyrationAggregates}
This utility calculates a gyration tensor and its eigenvalues (using Jacobi transformations) for aggregates of given size. It then determines various shape descriptors. It saves averages during the simulation (time evolution) to an output file and prints overall averages to standard output.

It calculates radius of gyration, $R_{\mathrm{G}}$:

\begin{equation}
  R_{\mathrm{G}}^2 = \lambda_x^2 + \lambda_y^2 + \lambda_z^2 \mbox{,}
\end{equation} where $\lambda_i^2$ is the $i$-th principle moment of the tensor
of gyration ($\lambda_x^2 \leq \lambda_y^2 \leq \lambda_z^2$).
Then it calculates the asphericity, $b$:

\begin{equation}
  b = \lambda_z^2 - \frac{1}{2} \left( \lambda_x^2 + \lambda_y^2 \right) = \frac{3}{2} \lambda_z^2 - \frac{R_{\mathrm{G}}^2}{2} \mbox{,}
\end{equation}
 the acylindricity, $c$:

\begin{equation}
  c = \lambda_y^2 - \lambda_x^2
\end{equation}
 and the relative shape anisotropy , $\kappa$:
\begin{equation}
  \kappa^2 = \frac{b^2 + 0.75 c^2}{R_{\mathrm{G}}^4} = \frac{3}{2}
  \frac{\lambda_x^4 + \lambda_y^4 + \lambda_z^4}{\left( \lambda_x^2 +
  \lambda_y^2 + \lambda_z^2 \right)^2}
\end{equation}


Usage\+:

{\ttfamily Gyration\+Aggregates $<$input.\+vcf$>$ $<$input.\+agg$>$ $<$output$>$ $<$agg sizes$>$ $<$options$>$}

\begin{quote}
{\ttfamily $<$input.\+vcf$>$} \begin{quote}
input coordinate filename (must end with {\ttfamily .vcf}) containing either ordered or indexed timesteps \end{quote}
{\ttfamily $<$input.\+agg$>$} \begin{quote}
input filename (must end with {\ttfamily .agg}) containing information about aggregates \end{quote}
{\ttfamily $<$output.\+vcf$>$} \begin{quote}
output filename with shape descriptors for chosen sizes throughout simulation \end{quote}
{\ttfamily $<$agg sizes$>$} \begin{quote}
aggregate sizes for gyration calculation \end{quote}
{\ttfamily $<$options$>$} \begin{quote}
{\ttfamily -\/j} \begin{quote}
specify that the {\ttfamily $<$input.\+vcf$>$} contains aggregates with joined coordinates \end{quote}
{\ttfamily -\/t} \begin{quote}
specify bead types to be used for calculation (default is all) \end{quote}
{\ttfamily -\/m $<$name$>$} \begin{quote}
take as an aggregate size the number of {\ttfamily $<$name$>$} molecules in aggregates instead of the number of all molecules \end{quote}
\end{quote}
\end{quote}


\begin{DoxyRefDesc}{Todo}
\item[\hyperlink{todo__todo000003}{Todo}]Gyration\+Aggregates\+: understand {\ttfamily jacobi} function\end{DoxyRefDesc}
\hypertarget{Common_GyrationMolecules}{}\subsection{Gyration\+Molecules utility}\label{Common_GyrationMolecules}
This utility function in the same way as Gyration\+Aggregates, but it calculates radii of gyration for specified molecule names instead of aggregate sizes.

Right now it calculates gyration for all beads in the specified molecule types.

Usage\+:

{\ttfamily Gyration\+Molecules $<$input.\+vcf$>$ $<$input.\+agg$>$ $<$output$>$ $<$molecule names$>$ $<$options$>$}

\begin{quote}
{\ttfamily $<$input.\+vcf$>$} \begin{quote}
input coordinate filename (must end with {\ttfamily .vcf}) containing either ordered or indexed timesteps \end{quote}
{\ttfamily $<$input.\+agg$>$} \begin{quote}
input filename (must end with {\ttfamily .agg}) containing information about aggregates \end{quote}
{\ttfamily $<$output.\+vcf$>$} \begin{quote}
output filename with radii of gyration throughout simulation (automatic ending \#.txt) \end{quote}
{\ttfamily $<$molecule names$>$} \begin{quote}
molecule types for gyration calculation \end{quote}
{\ttfamily $<$options$>$} \begin{quote}
{\ttfamily -\/j} \begin{quote}
specify that the {\ttfamily $<$input.\+vcf$>$} contains joined coordinates \end{quote}
\end{quote}
\end{quote}


\begin{DoxyRefDesc}{Todo}
\item[\hyperlink{todo__todo000004}{Todo}]Gyration\+Molecules\+: implement using only selected bead types for calculation\end{DoxyRefDesc}


\begin{DoxyRefDesc}{Todo}
\item[\hyperlink{todo__todo000005}{Todo}]Gyration\+Molecules\+: understand {\ttfamily jacobi} function\end{DoxyRefDesc}


\begin{DoxyRefDesc}{Todo}
\item[\hyperlink{todo__todo000006}{Todo}]Gyration\+: move function from Gyration\+Aggregates and Gyration\+Molecules to a separate header file\end{DoxyRefDesc}


\subsection*{Pair\+Correl utility (\#\+Pair\+Correl)}

This utility calculates pair correlation function between specified bead types. All pairs of bead types (including same pair) are calculated -\/ given A and B types, pcf between A-\/A, A-\/B and B-\/B are calculated.

Usage\+:

{\ttfamily Pair\+Correl $<$input.\+vcf$>$ $<$width$>$ $<$output.\+pcf$>$ $<$bead type(s)$>$ $<$options$>$}

\begin{quote}
{\ttfamily $<$input.\+vcf$>$} \begin{quote}
input coordinate filename (must end with {\ttfamily .vcf}) containing either ordered or indexed timesteps \end{quote}
{\ttfamily $<$width$>$} \begin{quote}
width of each bin for the distribution \end{quote}
{\ttfamily $<$output.\+pcf$>$} \begin{quote}
output file with pair correlation function(s) \end{quote}
{\ttfamily $<$bead type(s)$>$} \begin{quote}
bead type name(s) for pcf calculation \end{quote}
{\ttfamily $<$options$>$} \begin{quote}
{\ttfamily -\/n $<$int$>$} \begin{quote}
number of bins to average \end{quote}
{\ttfamily -\/st $<$int$>$} \begin{quote}
starting timestep for calculation \end{quote}
\end{quote}
\end{quote}


\subsection*{Pair\+Correl\+Per\+Agg utility (\#\+Pair\+Correl)}

Pair\+Correl\+Per\+Agg utility calculates pair correlation function per aggregates -\/ that is only beads in the same aggregate are used. If aggregate size(s) is not specified, average pcf is calculated (that is, regardless of aggregate size). In all probability the utility is working, but since it is not really useful, it has never been thouroughly tested.

Usage\+:

{\ttfamily Pair\+Correl $<$input.\+vcf$>$ $<$input.\+agg$>$ $<$width$>$ $<$output.\+pcf$>$ $<$bead type(s)$>$ $<$options$>$}

\begin{quote}
{\ttfamily $<$input.\+vcf$>$} \begin{quote}
input coordinate filename (must end with {\ttfamily .vcf}) containing either ordered or indexed timesteps \end{quote}
{\ttfamily $<$input.\+agg$>$} \begin{quote}
input filename (must end with {\ttfamily .agg}) containing information about \end{quote}
{\ttfamily $<$width$>$} \begin{quote}
width of each bin for the distribution \end{quote}
{\ttfamily $<$output.\+pcf$>$} \begin{quote}
output file with pair correlation function(s) \end{quote}
{\ttfamily $<$bead type(s)$>$} \begin{quote}
bead type name(s) for pcf calculation \end{quote}
{\ttfamily $<$options$>$} \begin{quote}
{\ttfamily -\/n $<$int$>$} \begin{quote}
number of bins to average \end{quote}
{\ttfamily -\/st $<$int$>$} \begin{quote}
starting timestep for calculation \end{quote}
\end{quote}
\end{quote}
\hypertarget{Common_JoinRuns}{}\subsection{Join\+Runs utility}\label{Common_JoinRuns}
M\+O\+ST L\+I\+K\+E\+LY N\+OT W\+O\+R\+K\+I\+NG -- IT\textquotesingle{}S N\+OT U\+S\+ED.

This program is to be used if two simulation runs with different initial seeds (that is, two simulations with different bead id numbers {\ttfamily .vsf} files, but identical {\ttfamily F\+I\+E\+LD} files) should be joined. Two {\ttfamily .vcf} files that contain the same bead types must be provided as well as the {\ttfamily .vsf} structure file for the second simulation. The output {\ttfamily .vcf} coordinate files has bead ids according to the structure file of the first simulation. The program is, however, extremely inefficient with unbonded beads, while bonded beads are always sorted in the same way by D\+L\+\_\+\+M\+E\+SO simulation software. The usefullness of such utility is confined to cases with more then one type of unbonded beads and under those conditions the utility may take around 1 minute per step (of the second simulation run) for system in box of side length 40.

Usage\+:

{\ttfamily Join\+Runs $<$1st input.\+vcf$>$ $<$2nd input.\+vcf$>$ $<$2nd input.\+vsf$>$ $<$output.\+vcf$>$ $<$type names$>$ $<$options$>$}

\begin{quote}
{\ttfamily $<$1st input.\+vcf$>$} \begin{quote}
input coordinate filename (must end with {\ttfamily .vcf}) containing either ordered or indexed timesteps for the first simulation \end{quote}
{\ttfamily $<$2nd input.\+vcf$>$} \begin{quote}
input coordinate filename for the second sumation in the same format as the first coordinate file \end{quote}
{\ttfamily $<$2nd input.\+vsf$>$} \begin{quote}
{\ttfamily .vsf} structure file for the second simulation (must end with {\ttfamily .vsf}) \end{quote}
{\ttfamily $<$output.\+vcf$>$} \begin{quote}
output filename with indexed coordinates (must end with {\ttfamily .vcf}) \end{quote}
{\ttfamily $<$type names$>$} \begin{quote}
names of bead types to save \end{quote}
{\ttfamily $<$options$>$} \begin{quote}
{\ttfamily -\/j} \begin{quote}
join individual molecules by removing periodic boundary conditions \end{quote}
{\ttfamily -\/n1 $<$int$>$} \begin{quote}
number of timestep to start the first simulation from \end{quote}
{\ttfamily -\/n2 $<$int$>$} \begin{quote}
number of timestep to start the second simulation from \end{quote}
{\ttfamily -\/u1 $<$int$>$} \begin{quote}
leave out every {\ttfamily skip} steps in the first simulation \end{quote}
{\ttfamily -\/u2 $<$int$>$} \begin{quote}
leave out every {\ttfamily skip} steps in the second simulation \end{quote}
\end{quote}
\end{quote}


\begin{DoxyRefDesc}{Todo}
\item[\hyperlink{todo__todo000007}{Todo}]Join\+Runs\+: base reindexing of beads in the second simulation on comparison between the two {\ttfamily .vsf} files \end{DoxyRefDesc}
\hypertarget{Common_Average}{}\subsection{Average utility}\label{Common_Average}
Utility calculating average values with standard deviation and autocorrelation time from values contained in a text file. The first line of the file has to contain the number of data lines and no comments are allowed.

Usage\+:

{\ttfamily Average $<$filename$>$ $<$column$>$ $<$discard$>$ $<$n\+\_\+blocks$>$}

\begin{quote}
{\ttfamily $<$filename$>$} \begin{quote}
name of data filel \end{quote}
{\ttfamily $<$column$>$} \begin{quote}
column number in the file containing the data to analyze \end{quote}
{\ttfamily $<$discard$>$} \begin{quote}
number of data values considered as equilibrium \end{quote}
{\ttfamily $<$n\+\_\+blocks$>$} \begin{quote}
number of blocks for binning analysis \end{quote}
\end{quote}


\begin{DoxyRefDesc}{Todo}
\item[\hyperlink{todo__todo000008}{Todo}]Average\+: completely rewrite -\/ especially remove requirement for number of lines on the first line of input file \end{DoxyRefDesc}
